% !TEX Program = xelatex

\documentclass[print, master, vlined, timesmath]{DissertUESTC}
\usepackage{minted}
\usepackage{tabularx}
\usepackage{ragged2e}
\usepackage{diagbox}
\usepackage{needspace}

\begin{document}



 \uestccover{几种特殊性质的智能合约漏洞检测技术研究}
 {网络空间安全}
 {202221080235}
 {张天义}
 {夏琦}
 {教授}
 {计算机科学与工程学院(网络空间安全学院)} 

 % 中文扉页,仅研究生用
 \DegLv{}  % 清空文档类选项设置的<申请学位级别>
 %\uestczhtitlepage  % 空白中文扉页
 \ClsNum{TP399}  % \ClsNum{<分类号>}
 \ClsLv{公开}  % \ClsLv{<密级>}
 \UDC{004.92}  % \UDC{<UDC号>}
 \DissertationTitle{几种特殊性质的智能合约漏洞检测技术研究}  % \DissertationTitle{<题名>}
 \Author{张天义}  % \Author{<作者姓名>}
 \Supervisor{夏琦}{教授}{电子科技大学}{成都}  % \Supervisor{<指导教师>}{<职称>}{<单位名称>}{<单位地址>}


% % 英文扉页,仅研究生用
 % \uestcentitlepage{<文题>}{<专业>}{<学号>}{<作者>}{<导师>}{<副导师>}{<学院>},若无副导师,则将“<副导师>”参数留空即可
 % \uestcentitlepage{}{}{}{}{}{}{}  % 空白英文扉页
 \uestcentitlepage{Research on Vulnerability Detection Techniques for Smart Contracts with Special Properties}
 {Cyber Security}
 {202221080235}
 {Tianyi Zhang}
 {Qi Xia}{Professor}
 {School of Computer Science and Engineering (School of Cyber Security)}


\zhabstract
随着区块链技术的广泛应用,智能合约已成为去中心化应用的核心,但其安全问题日益突出。由于智能合约承载高价值且部署后通常不可更改,漏洞可能导致巨额经济损失。现有的漏洞检测方法在处理 EVM 字节码层面分析时,面临精度、效率和自动化程度不足的挑战。

针对这些问题,本文聚焦于 EVM 字节码,提出了一种结合轻量级静态分析与符号执行的混合式智能合约漏洞检测方法。首先,本文深入分析了重入、整数溢出、时间依赖和拒绝服务四种典型漏洞在字节码层面的指令序列和状态转换特征。基于这些特征,本文提出了一种新型静态分析引导的符号执行策略:利用静态分析(模式匹配与污点分析)快速识别潜在漏洞候选点,并以此引导后续的动态符号执行聚焦于高风险代码路径,有效缓解路径爆炸问题,提升分析效率。

基于该混合方法,本文设计并实现了一个名为 ByteChecker 的原型系统。详细阐述了系统的整体架构、关键模块(包括预处理、核心分析引擎、约束求解器交互)、系统关键数据结构和算法。

最后,通过在标准数据集和真实世界合约数据集上的实验,对 ByteChecker 进行了全面评估。结果表明,该系统在检测目标漏洞方面具有良好的准确性(总体 F1 分数达到 74.6\%),且分析效率相较于纯符号执行工具有显著提升(平均分析时间减少 27.2\%)。实验验证了静态分析引导机制在平衡检测深度与效率方面的有效性。本研究为字节码层面的智能合约自动化安全审计提供了一种有效可行的新方法。

% 中文关键字
\zhkeywords{智能合约安全,漏洞检测,EVM 字节码,静态分析,符号执行}


% 开启英文摘要
\enabstract
With the widespread adoption of blockchain technology, smart contracts have become central to decentralized applications, yet their security vulnerabilities are increasingly prominent. Due to the high value they often manage and their typical immutability after deployment, vulnerabilities can lead to substantial financial losses. Existing vulnerability detection methods face challenges in precision, efficiency, and automation when analyzing Ethereum Virtual Machine (EVM) bytecode.

To address these issues, this thesis focuses on EVM bytecode and proposes a hybrid smart contract vulnerability detection method that combines lightweight static analysis with symbolic execution. Firstly, the instruction sequences and state transition characteristics of four typical vulnerabilities—reentrancy, integer overflow, timestamp dependency, and denial of service—at the bytecode level were thoroughly analyzed. Based on these characteristics, a novel static analysis-guided symbolic execution strategy was designed: static analysis (utilizing pattern matching and taint analysis) rapidly identifies potential vulnerability candidates, which then guides subsequent dynamic symbolic execution to focus on high-risk code paths. This approach effectively mitigates the path explosion problem and enhances analysis efficiency.

Based on this hybrid method, a prototype system named ByteChecker was designed and implemented. The thesis elaborates on the system's overall architecture, key modules (including preprocessing, core analysis engine, and constraint solver interaction), and the specific detection logic for the four targeted vulnerabilities.

Finally, ByteChecker was comprehensively evaluated through experiments on standard datasets and real-world contract datasets. The results demonstrate that the system achieves good accuracy in detecting the target vulnerabilities (overall F1 score of 74.6\%) and shows a significant improvement in analysis efficiency compared to pure symbolic execution tools (average analysis time reduced by 27.2\%). The experiments validate the effectiveness of the static analysis guidance mechanism in balancing detection depth and efficiency. This research provides an effective and feasible new approach for the automated security auditing of smart contracts at the bytecode level.

% 英文关键字
\enkeywords{Smart Contract Security, Vulnerability Detection, EVM Bytecode, Static Analysis, Symbolic Execution}

% 主目录,必要
\tableofcontents  
% 图多则放,反之不放
% \listoffigures 

\chapter{绪论}

\section{研究工作的背景与意义}

自 2008 年中本聪发布比特币白皮书\textsuperscript{\cite{whitebook}}以来,区块链技术以其分布式账本、密码学加密、共识机制等核心要素,构建了一种去中心化、不可篡改、公开透明的信任机器,迅速从数字货币领域扩展到众多行业,被认为是继互联网之后又一项颠覆性的技术创新\textsuperscript{\cite{BlockchainRevolution_2016}}。其核心价值在于通过技术手段在去中心化环境中建立信任关系。

智能合约是部署在区块链上、能根据预设规则自动执行条款的计算机程序。它将合约条款代码化,利用区块链特性保证执行的确定性和不可篡改性,允许在无须信任第三方的情况下进行可信交互,提高了效率并降低了成本。其中,去中心化金融(Decentralized Finance,DeFi)\textsuperscript{\cite{DecentralizedFinance_2020}}是其最引人瞩目的应用领域之一。此外,非同质化通证(Non-Fungible Token,NFT)\textsuperscript{\cite{MappingNFT_2021}}利用智能合约实现了数字资产的唯一性认证。去中心化自治组织(Decentralized Autonomous Organization,DAO)则通过智能合约编码组织规则。智能合约还在供应链溯源、数字身份验证、去中心化存储与计算\textsuperscript{\cite{ComprehensiveSurvey_2023}}等领域展现出巨大潜力。

由于智能合约承载价值且部署后通常不可更改,它们成为了网络攻击者的目标。近年来,针对智能合约的安全事件频发,造成了巨大的经济损失和信任危机。一些标志性事件揭示了其脆弱性。例如,2016 年的 The DAO 事件中,攻击者利用重入漏洞\textsuperscript{\cite{sendner2024large}}窃取了价值约 5000 万美元的以太币,最终导致以太坊硬分叉\textsuperscript{\cite{Stickfork_2017}}。2017 年 Parity 钱包两次遭受攻击,分别因初始化缺陷被盗约 3000 万美元\textsuperscript{\cite{wu2024comprehensive}},以及因另一漏洞导致价值超 1.5 亿美元的以太币被永久冻结,暴露了库复用和访问控制风险。2021 年 Poly Network 事件中,攻击者利用逻辑漏洞盗取了超 6 亿美元资产\textsuperscript{\cite{PolyNetwork}},凸显了跨链安全挑战。

随着 DeFi 生态的增长,攻击变得更加频繁和复杂。攻击者常利用闪电贷结合价格预言机操纵、重入、逻辑漏洞等手段获利\textsuperscript{\cite{AttackingDeFi_2021a}},据慢雾科技报告,仅 2024 年上半年 Web3 领域损失就超 10 亿美元\textsuperscript{\cite{slowmist}},其中智能合约漏洞是主因,DeFi 仍是重灾区。

这些安全事件的负面影响远超经济损失,严重打击了用户信心,阻碍了技术普及,并引发了监管关注\textsuperscript{\cite{SmartContract_2021}}。对项目方而言,安全事件可能导致项目失败和声誉受损。因此,智能合约安全已成为阻碍区块链大规模应用的重要障碍。

为应对此形势,部署前进行彻底的安全审计和漏洞检测已成行业共识\textsuperscript{\cite{FormalMethods_2022}}。然而,实现高效全面的检测面临多重挑战:现代合约逻辑极其复杂;EVM 的特殊执行环境如指令集、状态、Gas等为系统引入新风险;漏洞类型多样且隐蔽\textsuperscript{\cite{Blockchainsmart_2020}},许多隐藏在业务逻辑中;现有检测工具如静态分析、动态分析、符号执行、形式化验证各有局限,如误报、路径爆炸或扩展性差,缺乏完美方案\textsuperscript{\cite{Howeffective_2020}};且攻防持续演化,要求检测技术不断迭代。

综上所述,智能合约潜力巨大,但安全问题严峻。频发的安全事件造成巨额损失,侵蚀信任。尽管业界已努力,但现有检测技术在应对合约复杂性、环境特殊性、漏洞多样性及持续演化方面仍面临挑战。因此,研究更先进、有效的自动化漏洞检测技术具有重要的理论价值和迫切的现实意义。


\section{国内外研究现状}

随着区块链技术,特别是以太坊平台的兴起,智能合约已成为构建去中心化应用的核心组件。然而,由于其承载高价值资产且部署后通常不可更改的特性,智能合约的安全性问题备受关注。近年来,针对智能合约的攻击事件频发,造成了数十亿美元的经济损失,严重阻碍了区块链技术的健康发展和广泛应用\textsuperscript{\cite{jiao2024survey}}。因此,对智能合约进行有效的漏洞检测已成为学术界和工业界共同关注的研究热点。目前,国内外学者已提出多种检测方法和工具,主要可以从采用的技术角度进行归纳,包括模糊测试、符号执行、机器学习以及其他多种技术路线。

模糊测试作为一种广泛应用的动态分析技术,在智能合约领域也得到了深入研究。早期的工作如 ContractFuzzer\textsuperscript{\cite{jiang2018contractfuzzer}}通过解析合约的 ABI(Application Binary Interface)来自动生成函数调用序列,并在执行过程中监控运行时异常。为了提升测试的有效性和目标性,后续研究引入了更复杂的策略。Echidna\textsuperscript{\cite{grieco2020echidna}}采用了基于属性的模糊测试方法,允许开发者或审计人员定义合约应满足的安全属性,模糊器则尝试生成违反这些属性的交易序列。ILF\textsuperscript{\cite{he2019learning}}提出了一种基于不变量推断的模糊测试框架,通过静态分析推导合约的状态不变量,并利用这些不变量指导测试用例的生成和筛选。为了解决传统模糊测试在路径覆盖上的随机性问题,覆盖率引导的模糊测试被广泛应用,例如 sFuzz\textsuperscript{\cite{nguyen2020sfuzz}}结合轻量级静态分析来识别关键代码分支,优先探索可能包含漏洞的路径,显著提高了代码覆盖率。考虑到智能合约执行的上下文环境,特别是与其他合约的交互,ConFuzzius\textsuperscript{\cite{jiang2018contractfuzzer}}探索了上下文敏感的输入生成策略,能够更有效地检测涉及复杂跨合约调用的漏洞。模糊测试的优势在于能够发现与运行时状态和复杂交互相关的漏洞,且误报率相对较低,但其主要挑战在于难以保证路径覆盖的完备性,可能遗漏隐藏较深的漏洞。

符号执行是另一种在智能合约漏洞检测中发挥核心作用的技术,通常归类为高级静态分析。它通过使用符号值代替具体输入来模拟程序执行,探索多条执行路径,并将路径条件表达为符号约束。通过结合 SMT 求解器(如 Z3\textsuperscript{\cite{Z3}})求解这些约束与预定义漏洞模式的组合,可以判断是否存在触发漏洞的输入。Oyente\textsuperscript{\cite{luu2016making}}是该领域的开创性工作之一,首次将符号执行应用于以太坊智能合约,能够检测交易顺序依赖、时间戳依赖和部分重入等漏洞。Mythril在 Oyente 的基础上进行了改进,采用了更精确的 EVM 模拟和约束求解策略。Manticore\textsuperscript{\cite{Mossberg2019ma}}则提供了更灵活的分析框架,支持并行符号执行和自定义检测逻辑。为了缓解符号执行固有的路径爆炸问题,研究者们提出了多种优化策略。ETHBMC\textsuperscript{\cite{frank2020ethbmc}}借鉴了有界模型检验的思想,通过限制符号执行的深度或循环次数来控制分析范围。Seraph\textsuperscript{\cite{yang2020seraph}}则引入了基于抽象解释的符号执行方法,通过在抽象域上进行分析,有效降低了状态空间的复杂度和分析开销。尽管符号执行在深度路径分析方面能力强大,但路径爆炸和精确环境模拟仍然是其面临的主要挑战\textsuperscript{\cite{perez2023survey}}。

近年来,机器学习技术也被引入智能合约漏洞检测领域,为自动化分析提供了新的视角。研究者尝试利用机器学习模型从大量合约代码中学习漏洞模式。例如,ContractWard\textsuperscript{\cite{wang2020contractward}}使用深度学习模型(如 LSTM)直接从合约字节码序列中提取特征,用于快速识别多种类型的漏洞。为了更好地捕捉代码的结构信息,基于图神经网络(GNN)的方法受到关注。Dr.Contract将合约代码表示为抽象语法树或控制流图,并利用 GNN 和注意力机制来学习代码的语义和结构特征。CGE\textsuperscript{\cite{zhai2024smart}}提出将合约代码转换为代码属性图,并应用图嵌入技术进行漏洞模式学习。为了提高检测结果的可解释性和准确性,一些研究开始探索将机器学习与传统程序分析技术相结合。例如,ESCORT\textsuperscript{\cite{gao2020escort}}取的特征和深度学习模型,取得了较好的效果。机器学习方法的优势在于能够处理大规模数据集并可能发现未知的漏洞模式,但其效果很大程度上依赖于训练数据的质量和特征工程的设计,且模型的可解释性仍是一个挑战。


除了上述主流技术外,其他多种技术路线也为智能合约安全分析做出了贡献。形式化验证(Formal Verification)致力于提供最高级别的安全保证。研究者利用定理证明器或模型检测器对合约进行形式化建模和验证\textsuperscript{\cite{grishchenko2018semantic}}\textsuperscript{\cite{bhargavan2016formal}}。例如,ZEUS\textsuperscript{\cite{kalra2018zeus}}通过将 Solidity 代码编译为 LLVM 中间表示,并结合抽象解释和符号执行进行验证。EtherTrust\textsuperscript{\cite{albert2018ethertrust}}则为 EVM 字节码建立了形式化的小步操作语义,为精确分析奠定了基础。然而,形式化验证通常需要大量的人工介入来编写规范和证明,应用门槛较高。动态运行时监控也是一种补充方法,例如 SODA\textsuperscript{\cite{chen2020soda}}提出了一种轻量级的在线异常检测框架,通过监控合约执行过程中的关键指标来实时发现可疑行为。此外,混合分析方法尝试结合不同技术的优点,例如 Hybroid\textsuperscript{\cite{qian2022hybroid}}通过静态分析识别潜在风险点,再利用动态验证进行确认,以平衡检出率和误报率。


总的来看,智能合约漏洞检测的研究已经取得了显著进展,形成了多元化的技术格局。模糊测试擅长状态探索,符号执行擅长深度路径分析,机器学习提供了模式学习的新途径,而形式化验证则追求严格的正确性保证。然而,这些方法都存在各自的局限性。例如,纯粹的模糊测试难以保证对深层逻辑路径的覆盖,可能遗漏隐藏较深的漏洞;传统的符号执行虽然路径分析能力强,但易陷入路径爆炸的困境,且对复杂环境的精确模拟仍具挑战,导致分析效率低下或无法完成;机器学习方法则面临训练数据依赖性强、模型可解释性不足以及对零日漏洞检测能力有限等问题;形式化验证的应用门槛和人力成本较高,难以广泛应用于快速迭代的开发场景。单一技术往往难以完美应对智能合约的复杂性和多样化的漏洞类型,未来的研究趋势可能更加侧重于多技术的深度融合与协同,特别是在 EVM 字节码层面实现更紧密的交互,以期克服各自的局限性,实现更高效、更准确、更全面的自动化漏洞检测。同时,如何快速适应不断演化的合约模式和新型攻击手法,以及如何提升检测工具的易用性和在实际开发流程中的集成度,也是该领域持续面临的重要挑战。


\section{本文主要研究内容}

本研究聚焦于 EVM 字节码层面,提出一种结合轻量级静态分析引导下的符号执行的漏洞检测方法。其核心思想是通过研究重入漏洞、溢出漏洞、时间依赖漏洞和拒绝服务漏洞四种典型漏洞的字节码特征,设计高效的静态分析算法,结合污点分析追踪关键数据流向,快速识别潜在的漏洞点。然后通过静态分析结果引导符号执行,使其能够更聚焦于高风险的代码路径和状态空间,从而实现对特定漏洞更精确、更高效的自动化检测。

本文主要的内容和创新点如下:

(1)通过在字节码层面对重入漏洞、整数溢出漏洞、时间依赖漏洞和拒绝服务漏洞进行深入分析,总结出四种漏洞在 EVM 字节码层面的特征表征,针对四种漏洞分别提出了字节码层面检测方法。

(2)基于四种漏洞的字节码特征,针对性设计了一种新型静态分析引导的符号执行方法,在静态分析模块设计了一种结合了模式匹配和污点分析的静态分析方法,快速识别潜在的漏洞点,并追踪关键数据流向。在符号执行模块,基于四种漏洞的字节码特征,针对性设计了符号执行框架,优化了符号执行从符号化建模,到深度优先探索,再到路径约束求解的过程,提升了符号执行的效率和准确性。

(3) 设实现了一个基于字节码层面的智能合约漏洞检测系统 ByteChecker,系统包括静态分析模块和符号执行模块,能够高效地检测重入漏洞、整数溢出漏洞、时间依赖漏洞和拒绝服务漏洞。通过实验验证了系统的有效性和性能,并与现有工具进行了对比分析,展示了本方法在检测精度和效率上的优势。







\section{论文结构安排}


本论文共分为六章,各章节的主要内容安排如下:

第一章 绪论: 介绍研究背景、意义,分析国内外研究现状,总结现有研究的不足,引出本文的研究动机,明确本文的研究内容,并概述论文的整体结构。

第二章 背景知识介绍: 详细介绍智能合约、以太坊虚拟机(Ethereum Virtual Machine,EVM)、EVM 字节码以及符号执行技术的基本概念、原理和相关工作,为后续章节奠定理论基础。

第三章 漏洞原理分析: 深入分析重入、整数溢出、时间依赖、拒绝服务四种典型智能合约漏洞的产生原理和攻击实例,重点阐述这些漏洞在 EVM 字节码层面的具体指令序列和状态转换特征,为后续检测方法的设计提供依据。

第四章 系统设计与实现: 详细阐述所提出的结合轻量级静态分析与符号执行的漏洞检测方法的总体架构。重点介绍关键模块的设计,包括用于快速筛选的静态分析模块和用于深度验证的符号执行模块。阐述静态分析结果如何有效引导符号执行聚焦于高风险路径,以及整个检测流程和原型系统 ByteChecker 的具体实现细节。

第五章 实验与结果分析: 设计实验方案,选择合适的智能合约数据集和评价指标。对实现的原型系统 ByteChecker 进行全面的实验评估,展示并详细分析实验结果,将其与现有的代表性漏洞检测工具进行对比,以验证本文所提方法的有效性、性能优势,并讨论其存在的局限性。

第六章 总结与展望: 总结全文的主要研究工作和核心贡献,再次强调研究的创新点。同时,指出当前研究存在的不足之处,并对未来可能的研究方向和改进空间进行展望。


\chapter{背景知识介绍}
本章将系统性地介绍智能合约的基本概念、主流的编程语言 Solidity、开发流程及其面临的安全挑战;深入阐述作为智能合约执行核心的以太坊虚拟机(EVM)的架构、字节码结构、执行模型和 Gas 机制;并详细介绍本研究采用的核心分析技术——符号执行技术的原理、关键问题、约束求解及其在智能合约安全分析中的应用。理解这些背景知识对于把握本文所提检测方法的动机、设计原理和技术细节至关重要。

\section{智能合约}


\subsection{智能合约概念}
智能合约的出现,旨在利用区块链的去中心化、不可篡改和透明性等特点,将传统合约的执行过程自动化、程序化,从而降低信任成本,提高协作效率\textsuperscript{\cite{AttackingDeFi_2021a}}。它通过代码强制执行协议条款,为构建无需中介的复杂应用奠定了基础。在区块链语境下,智能合约通常指部署在区块链网络上、由代码定义的、能够根据预设规则自动执行、验证或强制执行协议条款的计算机程序。本质上,它是“代码即法律”思想的一种体现。

智能合约的核心在于将传统合约的条款和执行逻辑编码为计算机程序,并利用区块链的特性来保证其可靠执行。一个典型的智能合约主要包含几个核心组成部分:用于存储合约持久化数据的状态变量,它们代表合约的当前状态并记录在分布式账本中,例如代币合约中的账户余额;定义了合约可执行操作的函数,这些函数能够读取或修改状态变量,响应外部或内部调用,并可通过可见性修饰符控制访问权限;允许合约向外部世界发出信号的事件,用于通知重要操作的发生,其日志记录在链上供外部查询;以及一个可选的特殊构造函数,它仅在合约首次部署时执行一次,通常负责初始化合约状态。

当用户或其他合约向智能合约地址发送一笔交易,指定要调用的函数及相关参数时,区块链网络中的节点会执行该合约的对应代码。执行过程在 EVM 等虚拟机环境中进行,根据代码逻辑读取或修改状态变量。如果执行成功且交易被共识机制确认,状态的改变将永久记录在区块链上。这种基于代码、由网络共识保证执行的机制,使得智能合约能够在无需信任第三方的情况下,自动、透明且可靠地执行协议。

这些特性使得智能合约能够极大地提高效率、降低交易成本、增强透明度,并在金融、游戏、数字收藏品、组织治理、供应链管理等众多领域催生了革命性的应用。以太坊是目前最活跃、应用最广泛的智能合约平台,其上运行着数百万个智能合约,锁定了巨额的数字资产。

\subsection{Solidity编程语言}

Solidity 是一种专为以太坊平台上智能合约开发而设计的高级合约语言,由以太坊核心团队于 2014 年首次提出并实现。其语法风格受 JavaScript、C++ 和 Python 等语言的影响,采用静态类型系统,支持面向对象编程、继承、多态、库引用等特性,便于开发者构建结构化、可复用的智能合约模块。

作为以太坊虚拟机的主要合约语言,Solidity 的主要目标是描述合约在链上行为的逻辑,包括账户之间的状态转移、函数调用、事件触发与数据存储等。Solidity 源代码在编译后会被转换为 EVM 字节码部署至链上,在链上执行时由每个节点通过虚拟机逐指令解释运行,从而确保整个网络中合约行为的一致性与可验证性。

Solidity 在语言层面提供了丰富的原语,允许开发者访问调用上下文和链上状态信息。但同时,Solidity 也暴露出诸多安全风险,如重入攻击、整数溢出、未初始化存储引用、委托调用滥用等。由于智能合约一经部署后难以修改,这些安全漏洞可能导致不可逆的资产损失,因而 Solidity 的语言设计、安全分析与验证手段长期以来是学术界与工业界关注的研究重点之一。

\subsection{智能合约开发流程}

智能合约的开发通常遵循“编写—测试—部署—验证”的标准流程,其每一阶段均直接关系到合约最终的安全性与正确性。

首先,开发者使用 Solidity 等智能合约语言编写合约源代码,定义合约的状态变量、函数接口以及与以太坊虚拟机交互的逻辑行为。在此阶段,开发者需结合业务需求与链上环境限制,合理设计合约的控制流与数据结构,以减少冗余逻辑并预防已知的安全风险。

在编写完成后,合约需经过本地测试与模拟执行,通常使用如 Hardhat、Truffle 或 Foundry 等开发框架,配合 Ganache 等本地区块链模拟器,实现对合约在不同交易场景下行为的单元测试与集成测试。测试过程中可以使用断言、模拟攻击者行为等手段,验证合约在异常条件下的鲁棒性和边界处理能力。

测试通过后,合约源代码会被编译为以太坊虚拟机可识别的 EVM 字节码,并通过部署交易提交至以太坊主网或测试网。部署成功后,合约地址即被永久记录在链上,并对所有用户开放交互接口。

最后,在合约上线运行后,开发者通常会进行形式化验证、安全审计或部署监控机制,以保障合约运行的正确性与抗攻击能力。其中,形式化验证通过模型检查或符号执行等方法,对合约逻辑进行全面覆盖的可达性与安全属性分析;而安全审计则由专业团队基于手动分析与自动工具,对潜在的逻辑漏洞、权限管理失误等问题进行识别与修复建议。

需要强调的是,由于链上合约的不可更改性,一旦部署后发现安全缺陷,将极难修复。因此,完善的开发与验证流程在智能合约生命周期中具有关键意义,特别是在涉及高价值资产管理的 DeFi 合约中,更是保障系统安全的第一道防线。

\subsection{智能合约安全挑战}

随着区块链技术的迅猛发展,智能合约作为自动化执行合约逻辑的重要组件,广泛应用于金融、供应链管理、身份验证等多个领域。然而,由于其不可篡改和自动执行的特性,一旦智能合约部署在区块链上,其代码将无法更改,任何潜在的漏洞都可能被恶意利用,造成严重的安全后果。因此,智能合约的安全性问题日益受到研究人员和开发者的高度关注。

智能合约安全面临诸多挑战。首先,智能合约通常使用图灵完备的编程语言编写,这使得其逻辑表达能力强大但也更容易引入复杂的编程错误。其次,区块链环境的执行模型与传统程序运行环境存在显著差异,例如 Gas 消耗、交易顺序依赖性、链上数据的不可变性等因素,都可能成为漏洞利用的切入点。此外,智能合约之间的交互复杂,攻击者可能通过构造精巧的调用链实现重入攻击、拒绝服务攻击、逻辑劫持等行为。最后,由于当前缺乏标准化的开发和测试流程,开发者对安全实践掌握不足,也进一步加剧了安全风险。

近年来发生的大量安全事件,如 The DAO 被盗、Parity 钱包冻结、众多 DeFi 协议被攻击都凸显了智能合约安全挑战的严峻性。因此,在部署前对智能合约进行彻底的安全分析和漏洞检测至关重要。

\section{以太坊虚拟机}
以太坊虚拟机(Ethereum Virtual Machine, EVM)是理解智能合约执行和漏洞的关键。它是以太坊网络中执行智能合约字节码的运行时环境。

\subsection{EVM架构}

EVM 是一个基于栈 (Stack-based) 的虚拟机\textsuperscript{\cite{yellowbook}}。这意味着大多数操作都是通过操作栈顶的数据来完成的,而不是使用寄存器。它被设计为一个确定性的沙盒环境\textsuperscript{\cite{jamwal2024survey}},保证相同的输入和状态在任何兼容的客户端上执行都会产生完全相同的结果,这是区块链共识的基础。EVM 是准图灵完备的\textsuperscript{\cite{wang2025tsgn}},其计算能力理论上是图灵完备的,但实际执行受到 Gas 总量的限制,以防止无限循环等滥用资源的行为。

EVM 的核心架构组件包括:

\begin{figure}[htbp]
    \centering
    \includegraphics[width=0.8\textwidth]{EVM架构}
    \caption{EVM架构}
    \label{fig:EVM架构}
\end{figure}

\textbf{栈(Stack)}

以太坊虚拟机(EVM)使用一个后进先出(LIFO)结构的栈作为其基本操作数存储机制。该栈用于保存指令执行过程中临时使用的操作数与中间计算结果。EVM 栈的每个槽(slot)存储一个 256 位长度的字,最多可容纳 1024 个元素,超过此限制将触发运行时异常。指令如 PUSHx、POP、DUPx、SWAPx、ADD、SUB 等均以栈为操作目标,是实现算术逻辑、数据调度与控制流程的重要基础。

\textbf{内存(Memory)}

EVM 提供一个线性、按字节寻址的临时内存空间,用于存储函数参数、局部变量、返回值及各种计算中间结果。该内存为易失性,每次消息调用(包括外部交易或合约间调用)都会分配一块全新的、内容清零的内存区域,调用结束后立即被丢弃。内存空间支持动态扩展,但其扩容代价与内存使用量成二次方关系,旨在限制过度内存消耗。指令如 SHA3、CALL、CODECOPY 等依赖内存作为输入输出数据的缓冲区,因此内存性能对执行效率影响显著。

\textbf{存储(Storage)}

合约存储是 EVM 中唯一持久化的数据区域,用于维护链上合约的状态信息。其本质上是一个 256 位键到 256 位值的键值映射结构,每个合约拥有独立的存储空间。Solidity 编译器会将高级语言中的状态变量(如结构体、数组、映射等)映射到具体的存储槽地址中。由于对全局状态的修改需被全网节点共识确认,SLOAD 和 SSTORE 等指令的 Gas 成本极高。特别是 SSTORE,其消耗量取决于存储值的变化类型,例如从零变为非零或从非零变为零。此外,合约的存储内容在交易之间保持不变,直到被显式覆盖或合约被销毁。

\textbf{合约代码区(Code ROM)}

每个合约账户在链上都有一段只读代码区域,存放其部署时写入的 EVM 字节码。该区域在执行过程中无法修改,仅可读取。EVM 通过顺序扫描此代码区的指令来执行程序流程,程序计数器(PC)记录当前指令的字节偏移位置。指令 CODESIZE 可获取代码长度,CODECOPY 可将指定段的代码复制到内存中,用于如代理合约、元交易等动态代码操作场景。

\textbf{程序计数器(Program Counter, PC)}

程序计数器是 EVM 的控制流核心组件之一,其值指向当前合约代码中即将被执行的字节码位置。执行非跳转类指令后,PC 自动递增以继续顺序执行;而遇到 JUMP 或 JUMPI 等跳转类指令时,PC 的值将被替换为目标跳转地址。这一机制支持合约中的条件判断、循环结构和函数调度。

\textbf{可用 Gas(Gas Available)}

Gas 是以太坊对计算资源消耗的度量单位。每个执行上下文在启动时都会分配一定的 Gas 预算,随着每条指令的执行而持续扣减。若执行过程中剩余 Gas 耗尽,EVM 将强制终止执行并回滚当前交易。因此,Gas 剩余量不仅影响程序能否顺利完成执行,也间接限制了路径长度、内存扩展和复杂计算操作的可行性。

\textbf{调用数据(Calldata)}

Calldata 是当前合约执行上下文的只读输入数据,通常由外部交易或合约调用附带而来。其内容包括函数选择器(前四字节)和 ABI 编码的参数信息。由于 Calldata 不会被写入区块链全局状态,且其访问成本远低于存储或内存,因此特别适用于大批量只读数据的传输。EVM 通过指令 CALLDATALOAD、CALLDATASIZE 和 CALLDATACOPY 提供对 Calldata 的读取与拷贝操作,常用于函数参数解析与输入预处理等环节。


\subsection{EVM字节码结构}

智能合约经由 Solidity 等高级语言编写后,最终被编译为以太坊虚拟机能够识别和执行的字节码。这一字节码并非杂乱无章的指令集合,而是具有明确语义和结构划分的程序表示形式。对字节码结构的理解是开展静态与动态漏洞分析的前提,尤其是在无源代码可用的情形下,字节码是唯一的分析对象。

EVM 字节码的基本构成单位是操作码与参数。它由一系列操作码组成,每个操作码占用 1 个字节,表示一种原子操作,例如 ADD、SLOAD、JUMP 等。部分操作码(尤其是 PUSH1 至 PUSH32)之后会跟随 1 到 32 字节的参数数据,用于向栈顶压入常量。例如,字节码 0x6080 表示 PUSH1 0x80,其语义为将常数 0x80 压入操作数栈顶。此外,大多数操作码依赖操作数栈来传递操作对象,执行过程中自动从栈顶取值并将计算结果重新压栈。这种基于栈的执行模型决定了 EVM 在语义提取与控制流还原方面具有显著的挑战性。

EVM 字节码通常可分为部署代码与运行时代码两部分\textsuperscript{\cite{albert2025secure}}。部署代码是智能合约首次部署时随交易发送的代码,其作用是初始化合约状态并产出最终的运行时代码。部署代码会执行合约构造函数的逻辑,并在内存中构造完整的运行时代码,最后通过 RETURN 指令将该代码片段返回。以太坊节点接收到这段返回值后,将其作为合约的“主体代码”持久化至链上,从而在后续交易中作为该合约的实际执行逻辑。因此,在漏洞检测与程序分析中,分析对象通常聚焦于运行时代码。

运行时代码本身具备一定的结构化模式,尤其在由现代 Solidity 编译器生成的合约中更为显著\textsuperscript{\cite{xiang2025automating}}。其主要结构包括函数分发器、函数体、辅助逻辑与内置函数代码以及元数据哈希。字节码起始位置通常包含一段函数分发器逻辑,用于处理外部调用传入的 calldata。该逻辑会提取前 4 字节作为函数选择器,并与所有已定义的 public 或 external 函数选择器进行匹配。一旦匹配成功,程序通过条件跳转(JUMPI)跳转至对应函数的入口代码块;若无匹配项,则会跳转至 fallback 或 receive 函数,或执行 REVERT 操作中止执行。匹配成功后,控制流进入目标函数对应的基本块序列,即函数体。这些基本块之间通过 JUMP 和 JUMPI 连接,函数体内部可能还包含对其他函数的调用、异常处理逻辑等。函数返回通常由 RETURN 或 REVERT 实现。运行时代码中还可能嵌入由编译器自动插入的安全检查代码、内存操作代码、ABI 解码与编码逻辑等辅助性片段\textsuperscript{\cite{li2024varlifter}}。最后,在字节码末尾,现代 Solidity 编译器(自 0.6.0 起)会附加一段非可执行的元数据哈希。该哈希采用 CBOR 编码,包含编译器版本、源码哈希、依赖库信息等元信息,对于漏洞检测无直接语义影响,但在合约归属与版本识别中具有参考价值。

\subsection{EVM执行模型}

以太坊虚拟机是智能合约在以太坊区块链上运行的执行环境,其设计目标在于保障合约执行的确定性、原子性与安全性\textsuperscript{\cite{fu2024evmfuzz}}。EVM 采用一种基于程序计数器驱动的循环执行模型,以解释方式逐条处理合约字节码中的操作码。深入理解这一执行机制,对于智能合约漏洞检测中的符号执行、模糊测试等分析技术具有重要的理论支撑意义。

在执行过程中,EVM 按照固定的指令循环对每条字节码进行处理。首先,虚拟机根据当前程序计数器所指的位置,从合约字节码中获取当前待执行的操作码。随后,系统将验证该指令的合法性,并检查剩余的 Gas 是否足以承担其基础执行成本。若操作码无效或资源不足,则立即触发异常中止,通常表现为 REVERT 操作,回滚当前执行状态。

在通过上述验证后,EVM 根据操作码的语义执行相应操作。这些操作涵盖从操作数栈中弹出参数、执行算术或逻辑运算、访问内存或存储区域、读取环境上下文信息(如区块高度、交易发送者地址、当前时间戳等),以及修改程序控制流(例如通过 JUMP、JUMPI 实现分支跳转)。某些指令还可发起外部合约调用,例如 CALL、DELEGATECALL 等,实现合约间的互操作性。在完成指令语义后,虚拟机会更新自身的内部状态,包括程序计数器位置、栈与内存的内容、存储区域的状态,以及当前剩余的 Gas。除非遇到明确的终止指令或运行时错误,EVM 将进入下一轮指令循环,继续执行后续字节码。

整个执行过程持续进行,直到遇到显式终止指令(如 STOP、RETURN、REVERT、SELFDESTRUCT)或在执行过程中发生错误(如栈溢出、无效跳转、Gas 耗尽等)。此种循环式的解释执行模型不仅保证了合约逻辑的完整性,也为静态与动态分析方法提供了结构清晰、便于建模的执行框架。

EVM 在其执行机制中还体现出若干关键的语义特性,使其特别适用于区块链这一去中心化的执行环境。其中最为核心的特性之一是原子性(Atomicity)。每一笔交易及其触发的所有内部合约调用均被视作一个不可分割的执行单元,要么全部成功并将状态更改永久写入区块链,要么在出现异常时整体回滚,不留下任何副作用。这一机制有效防止了部分执行结果被不一致记录的问题,是保障跨合约交互安全的基础。

其次,EVM 的确定性(Determinism)确保了在任意节点上,对相同初始状态与输入的执行结果完全一致。这一特性对于维护链上全网共识至关重要。因此,EVM 明确禁止使用非确定性的外部依赖,如系统时间、随机源、网络状态等,仅允许访问严格受控的链上数据,确保所有执行路径可验证、可复现。

最后,EVM 的沙盒化设计\textsuperscript{\cite{lagouvardos2025precise}}(Sandboxing)提供了执行环境的隔离性。智能合约无法直接访问底层节点的操作系统资源,亦不可发起外部网络连接或读取本地文件系统,只能通过栈、内存、存储和少量环境变量进行有限的信息交互。此种封闭的执行上下文不仅提升了平台的安全性,也大大增强了合约行为的可预测性与可分析性。

\subsection{Gas机制}

在以太坊中,Gas 是用于衡量计算资源消耗的基本单位,它不仅是 EVM 运转的经济基础,也是一种资源管理与安全保障机制。每当智能合约执行时,EVM 会基于操作指令的复杂度和资源消耗量,对其分配相应的 Gas 成本。简而言之,Gas 是链上计算的“燃料”,控制着合约执行的经济边界与行为约束。

EVM 中的每一条操作码都有其对应的固定 Gas 成本。例如,诸如 PUSH、ADD 这类简单的堆栈操作具有较低的 Gas 消耗;而像 SSTORE、SHA3、CAL等复杂或资源密集型操作则具有较高的 Gas 成本。此外,诸如内存扩展、调用数据的传输等因素也会进一步增加交易的总 Gas 消耗。因此,交易最终所消耗的 Gas 数量,是合约执行路径上所有指令 Gas 成本的累计总和。

Gas 机制的设计初衷之一是防止资源滥用与拒绝服务攻击(Denial-of-Service, DoS)。由于智能合约中可能存在复杂的控制流甚至是死循环结构,Gas 系统通过为每一步运算引入成本,使得无止境的代码执行不可持续。一旦合约陷入过度计算或无限循环,系统将因 Gas 耗尽而被迫中止执行,从而有效避免对节点资源的持续占用。此外,高昂的存储与计算费用也抑制了用户对链上状态的不必要扩张,为以太坊网络提供了天然的抗滥用机制。

从经济激励角度出发,Gas 机制还承担着激励矿工的关键职能。用户在发起交易时需设定两个与 Gas 相关的参数:其一为 Gas Limit,表示用户愿意为此次交易支付的最大 Gas 数量;其二为 Gas Price,表示用户为每单位 Gas 所愿意支付的价格,单位通常为 Gwei。随着 EIP-1559 的引入,Gas 定价机制发生变化,现阶段交易费用包含基础费用(Base Fee)与优先费用(Priority Fee)两部分。交易执行完成后,用户需按实际消耗的 Gas(Gas Used)乘以相应的价格支付费用,该费用作为对矿工处理交易、提供计算资源的经济补偿。

在合约执行完成后,EVM 根据执行过程中的 Gas 消耗情况,区分出不同的执行结果。若交易在 Gas Limit 范围内成功执行完毕,则相关状态更改被写入区块链,剩余的未使用 Gas 会返还给交易发起者;但已使用部分的费用仍需支付。而当执行过程超出设定的 Gas Limit 时,EVM 将立即中止,抛出“Out of Gas”异常,所有状态更改被回滚,但消耗的 Gas 无法返还。此外,在执行过程中若遇到 REVERT 指令或其他运行时错误,EVM 同样中止执行并回滚状态,但不同于 Gas 耗尽的是,此时未消耗的 Gas 会被退还。

Gas 机制贯穿以太坊的执行逻辑、经济激励与安全控制,是智能合约生命周期中的关键一环。对开发者而言,Gas 不仅是资源定价单位,也是一种设计约束,要求其编写更加高效、节能的代码逻辑;而对于攻击者而言,Gas 也可能成为攻击媒介,特别是在设计 DoS 攻击时,借由构造高 Gas 消耗路径诱导合约执行失败。因此,Gas 分析在智能合约的安全审计工作中具有不可忽视的地位,是评估合约可用性与鲁棒性的核心维度之一。


% ...existing code...


\section{符号执行技术}

符号执行(Symbolic Execution)是一种用于程序分析与漏洞检测的重要技术。与传统动态测试(Dynamic Testing)依赖具体输入值、每次仅探索一条执行路径的策略不同,符号执行以符号变量替代实际输入,通过系统性路径探索实现对程序行为的更全面覆盖。其核心思想在于对程序输入建模为抽象的符号,并在程序执行过程中以符号表达式形式跟踪变量值和路径条件,从而推导出不同输入所引发的多种执行路径。

在符号执行中,程序的输入被表示为一组符号变量(Symbolic Inputs),而非具体的数值。例如,函数参数、用户输入、环境变量等均可视为符号变量的来源。程序在处理这些输入时,不再直接计算出具体值,而是对变量进行符号操作,形成数学或逻辑表达式。由此,符号执行构建出一个“符号状态”(Symbolic State),其中每个程序变量的当前值都是由输入符号变量组合而成的表达式。举例而言,若输入为符号变量 x,执行语句 y = x + 5 后,变量 y 的值将表示为符号表达式 x + 5,而非实际的数值。

为正确记录程序执行路径中的分支选择,符号执行维护一个称为路径条件(Path Condition, PC)的逻辑公式。该路径条件是输入符号变量所需满足的布尔约束集合,用于描述当前路径的可达性条件。初始路径条件为真(True),表示没有任何输入限制。每当程序执行进入条件分支语句(如 if (condition)),且该条件表达式中含有符号变量时,符号执行引擎会同时探索两个逻辑路径。

通过对所有分支路径进行上述分析,符号执行逐步构建出一棵完整的执行树(Execution Tree)。树的每一条从根节点到叶节点的路径代表程序在某类输入下的一条可能执行路径;而每条路径对应的最终路径条件则刻画了触发该路径所需满足的输入约束。通过求解这些路径条件(通常使用约束求解器,如 SMT Solver),可以生成能够覆盖特定路径甚至触发潜在漏洞的具体输入。

综上所述,符号执行不仅提供了一种系统性探索程序行为的能力,还为漏洞检测、路径覆盖分析以及自动化测试生成等场景提供了强有力的理论与实践支撑。尤其在智能合约等对安全性要求极高的场景中,符号执行能够有效识别复杂条件下才会暴露的逻辑漏洞,成为当前主流安全分析工具的关键组成部分。

\subsection{符号执行原理}
符号执行(Symbolic Execution)是一种用于程序分析与漏洞检测的重要技术。与传统动态测试(Dynamic Testing)依赖具体输入值、每次仅探索一条执行路径的策略不同,符号执行以符号变量替代实际输入,通过系统性路径探索实现对程序行为的更全面覆盖。其核心思想在于对程序输入建模为抽象的符号,并在程序执行过程中以符号表达式形式跟踪变量值和路径条件,从而推导出不同输入所引发的多种执行路径。

在符号执行中,程序的输入被表示为一组符号变量(Symbolic Inputs),而非具体的数值。例如,函数参数、用户输入、环境变量等均可视为符号变量的来源。程序在处理这些输入时,不再直接计算出具体值,而是对变量进行符号操作,形成数学或逻辑表达式。由此,符号执行构建出一个符号状态,其中每个程序变量的当前值都是由输入符号变量组合而成的表达式。举例而言,若输入为符号变量 x,执行语句 y = x + 5 后,变量 y 的值将表示为符号表达式 x + 5,而非实际的数值。

为正确记录程序执行路径中的分支选择,符号执行维护一个称为路径条件(Path Condition, PC)的逻辑公式。该路径条件是输入符号变量所需满足的布尔约束集合,用于描述当前路径的可达性条件。初始路径条件为真,表示没有任何输入限制。每当程序执行进入条件分支语句,且该条件表达式中含有符号变量时,符号执行引擎会同时探索正反两个逻辑路径。

通过对所有分支路径进行上述分析,符号执行逐步构建出一棵完整的执行树(Execution Tree)。树的每一条从根节点到叶节点的路径代表程序在某类输入下的一条可能执行路径;而每条路径对应的最终路径条件则刻画了触发该路径所需满足的输入约束。通过求解这些路径条件,可以生成能够覆盖特定路径甚至触发潜在漏洞的具体输入。

符号执行不仅提供了一种系统性探索程序行为的能力,还为漏洞检测、路径覆盖分析以及自动化测试生成等场景提供了强有力的理论与实践支撑。尤其在智能合约等对安全性要求极高的场景中,符号执行能够有效识别复杂条件下才会暴露的逻辑漏洞,成为当前主流安全分析工具的关键组成部分。

\subsection{路径爆炸问题}

尽管符号执行在程序分析领域表现出极高的精度与路径覆盖能力,但其最根本的技术瓶颈在于路径爆炸问题(Path Explosion)。该问题源于程序中存在的大量分支结构,尤其是条件语句、循环控制及函数调用中的路径分歧。这些结构会导致程序的潜在执行路径数量呈指数级增长,极大地制约了符号执行技术的可扩展性和应用范围。

路径爆炸的根本原因在于程序结构的组合复杂性。以包含 n 个独立 if 分支的程序为例,其理论上的路径总数为 2 的 n 次方,随着 n 的增长,路径数量迅速激增。更为复杂的情形出现在循环控制结构中,若循环次数依赖于符号输入,其所生成的路径甚至可能趋于无限。路径数量的快速膨胀,使得符号执行在面对中大型程序时,往往难以在可接受的时间与资源成本内完成全路径探索。

路径爆炸所带来的直接后果是分析效率和覆盖能力的严重下降。在实际应用中,即便是规模适中的程序,也可能因路径数量过多而使符号执行引擎陷入“路径爆炸”的计算困境,最终导致路径探索中断、分析中止或覆盖率不足。这一问题极大地限制了符号执行技术在工程实践中的广泛应用,特别是在智能合约、操作系统内核、安全协议等结构复杂、行为丰富的程序分析任务中更为突出。

为缓解路径爆炸问题,研究者提出了多种策略加以应对。首先是路径选择启发式\textsuperscript{\cite{Chen2018dynamic}}(Path Selection Heuristics)方法,它通过启发式规则对路径进行优先级排序,避免遍历所有可能路径,而是选择那些更可能触发异常或提高覆盖率的路径继续执行。例如,优先探索尚未覆盖的代码分支(如基于覆盖率的搜索策略)、包含特定敏感操作的路径、或基于随机抽样选择路径等方式,均可有效提升分析效率。本研究第四章中所设计的动态分析模块也采用了此类路径调度机制,以在控制路径爆炸的同时提升漏洞挖掘的有效性。

另一种有效策略是状态合并\textsuperscript{\cite{zhang2024compiler}}(State Merging)技术。当多个不同路径在某程序点收敛时,若它们的程序状态(除路径条件外)相似,可以尝试将其合并为一个统一的状态,并以更复杂的条件表达式来表示其执行上下文,从而减少待维护的执行状态数。这一方法在一定程度上降低了内存占用与路径数量,但也会增加路径条件的逻辑复杂度,提升对约束求解器的压力。

此外,还可以通过执行限制与剪枝策略对分析过程施加控制。例如,强制设置路径探索的最大深度、循环展开的次数上限、分析运行时长或单次求解超时时间。当某条路径超出限制时,系统即放弃其后续探索,以保障整体分析的可控性和资源消耗的可预测性。

最后,近年来越来越多的研究关注于将符号执行与其他程序分析技术结合,以互补各自的优势\textsuperscript{\cite{liang2025vulseye}}\textsuperscript{\cite{wu2024we}}\textsuperscript{\cite{han2024ethchecker}}。特别是与模糊测试结合的混合测试(Hybrid Testing)方法,能够利用符号执行精确引导模糊测试覆盖稀有路径,或用模糊测试触发新的符号路径,从而实现更高效的漏洞检测与路径探索机制。本研究亦采纳了类似的设计理念,在第五章提出了融合符号执行与动态模糊测试的分析框架,以提高分析效率并突破路径爆炸的瓶颈。

因此,路径爆炸是符号执行的关键技术挑战,也是制约其工程化落地的主要障碍之一。尽管如此,符号执行所具备的精确分析能力和理论完备性,仍使其在漏洞挖掘、安全验证等领域具有不可替代的价值。

\subsection{约束求解}

在符号执行的过程中,约束求解器\textsuperscript{\cite{barrett2024satisfiability}}(Constraint Solver)承担着极其关键的角色。它负责处理执行路径中生成的逻辑条件,决定路径的可达性,生成具体的触发输入,并辅助识别程序中的潜在漏洞。约束求解器的能力与效率直接影响符号执行的覆盖深度、分析速度与漏洞检测的有效性\textsuperscript{\cite{shuai2024partial}}。

具体而言,在符号执行引擎的运行过程中,每当遇到条件跳转(如 JUMPI)、状态断言、或漏洞检测中的分支判断时,引擎都会生成或更新路径条件(Path Condition, PC),并将其交由约束求解器进行可满足性检查。路径条件本质上是一个逻辑公式,表示到达当前程序点所需满足的符号变量之间的约束关系。求解器需判断该公式是否存在满足解,即是否存在一组输入能沿当前路径到达此程序状态。

当前主流的符号执行工具普遍采用可满足性模理论(Satisfiability Modulo Theories, SMT)求解器来完成上述任务。SMT 求解器是传统布尔可满足性(SAT)求解器的扩展,能够在布尔逻辑基础上引入多种数学理论支持,如线性与非线性整数/实数算术、位向量(Bitvectors)、数组(Arrays)、字符串(Strings)等,从而处理更加复杂和现实的路径约束。

在符号执行分析中,最常使用的求解器是Z3,由微软研究院开发。Z3 具有高性能、强表达能力及良好的可扩展性,广泛用于学术研究和工业实践中。其对位向量和数组理论的优化支持,使其尤其适用于面向 EVM 的合约分析场景。此外,Z3 提供了功能完备的 Python API,便于集成到各种分析框架中。除 Z3 之外,CVC4,及其后续版本 CVC5、Yices 等求解器也常见于符号执行相关工具链中。

符号执行引擎与约束求解器之间的典型交互流程如下:首先,引擎将当前路径条件 PC 及可能的附加断言 A 转换为 SMT 求解器可识别的逻辑格式(如 SMT-LIBv2 语法,或通过 API 进行建模);其次,调用求解器的求解方法,判断逻辑公式的可满足性;最后,根据求解器返回的结果做出响应:

若返回 SAT(Satisfiable),说明存在某些输入值能够满足当前路径条件及断言,表明该路径是可行的,或漏洞条件可能被触发。此时可进一步调用 model() 方法获取满足条件的输入实例,即所谓“模型生成”。

若返回 UNSAT(Unsatisfiable),则表明路径条件与断言之间存在逻辑冲突,路径不可达,或漏洞条件无法在该路径下成立。此信息可用于剪枝,避免无效路径的进一步探索。

若返回 UNKNOWN,则表示在当前资源或时间限制下,求解器无法判定路径条件的可满足性。此时分析框架通常采取保守策略,决定是否继续探索或标记该路径为不确定状态。

尽管 SMT 求解器为符号执行提供了强大的支持能力,其性能瓶颈仍是整个分析流程中的关键挑战之一。特别是当路径条件涉及大量符号变量、复杂表达式组合或深层嵌套结构时,求解器可能面临求解时间过长或资源耗尽的问题。这种瓶颈在实际应用中会显著影响符号执行的路径覆盖率与整体运行效率。

因此,约束求解器不仅是符号执行的基础支撑模块,更是决定其性能上限的重要因素。针对求解器性能与表达能力的优化,以及与符号执行引擎的高效集成方式,已成为当前程序分析研究中的热点方向之一。

\subsection{符号执行在智能合约中的应用}

符号执行技术近年来被广泛应用于智能合约的安全分析之中,其有效性已在多个研究与实用工具中得到充分验证。与传统软件相比,智能合约的特性为符号执行技术的应用提供了诸多便利条件。

首先,以太坊虚拟机具备确定性执行模型,即相同的输入总会产生相同的执行结果。这种确定性极大地降低了环境建模的复杂性,使符号执行能够专注于分析合约逻辑本身,而不必过多处理系统调用、线程调度或其他外部不确定性因素所引发的路径分歧。虽然仍需建模如区块属性、外部账户行为等部分环境状态,但相较于传统操作系统级程序的符号分析任务,环境建模难度明显降低。

其次,以太坊平台采用字节码中立架构,使得即便缺乏合约源代码,分析者仍可直接对 EVM 字节码进行符号执行。由于 EVM 字节码结构规范、形式明确,并且大量合约在主网上以字节码形式公开可访问,因此无需源代码依赖即能开展符号分析,大大提升了技术的实用性。

此外,智能合约的高价值特性也赋予了符号执行在该领域中的独特地位。合约漏洞常直接导致数字资产的流失或锁定,造成严重经济损失。因此,即便符号执行存在计算资源开销大、路径爆炸等问题,其“高成本高收益”的特征在智能合约分析中依然具有极高的投入回报比。


进一步地,符号执行还可支持合约级的程序验证。研究人员可通过符号路径覆盖的方式,对某些关键性安全属性进行验证,通过在符号路径上插入断言并检验其在所有路径上是否恒为真,可实现对合约行为的形式化验证。

目前,多个主流智能合约分析工具均已集成符号执行模块,代表性工具包括:
Mythril:最早应用广泛的合约安全分析工具,主要基于符号执行引擎进行漏洞检测;
Manticore:由 Trail of Bits 开发的通用二进制符号执行平台,支持对 EVM 字节码进行分析;
Oyente:较早提出的符号执行工具,具备基础漏洞识别能力;
Maian:专注于检测特定类型合约脆弱性(如 Prodigal、Suicidal、Greedy 合约)的专用符号执行工具。

尽管符号执行在实际应用中仍面临路径爆炸、约束求解瓶颈等问题,其深度、系统性、可扩展的分析能力依然使其成为智能合约安全研究中的核心方法之一。
\section{本章小结}
本章系统地梳理了进行智能合约漏洞检测研究所需的核心背景知识,首先从智能合约入手,阐述了其基本概念、核心组成与工作方式、主流的 Solidity 编程语言及其开发流程,并强调了智能合约所面临的严峻安全挑战。随后,介绍了智能合约的执行环境——以太坊虚拟机,详细解析了其基于栈的核心架构、字节码的基本结构、原子性的交易执行模型以及对其运行至关重要的 Gas 机制。接着,本章介绍了静态分析技术的基本原理、常用方法及其在智能合约安全领域的应用与局限。最后,本章重点介绍了本研究采用的核心分析技术——符号执行,系统地阐述了其基本原理、面临的主要挑战、对关键组件约束求解器的依赖,以及该技术在智能合约安全领域的广泛应用和价值。
\chapter{漏洞原理分析及其检测方法}
本章将深入剖析智能合约中四种典型且危害严重的安全漏洞:重入漏洞、整数溢出漏洞、时间依赖漏洞和拒绝服务攻击漏洞。对于每种漏洞,本章将详细阐述其产生原理,通过具体的 Solidity 代码示例进行说明,并着重分析其在 EVM 字节码层面的表现特征,并最终提出并论述本研究设计的针对性检测策略。本文漏洞的选取基于以下标准:

(1)现实危害性:这四类漏洞的选择参考了 OWASP智能合约漏洞榜单\textsuperscript{\cite{owasp}}、DASP安全排行榜\textsuperscript{\cite{dasp}}以及 SWC安全漏洞目录\textsuperscript{\cite{swc}},以确保所研究问题的现实影响力及经济价值,从而提升研究的实际应用性。

(2)字节码分析可行性:由于以太坊上的大部分智能合约未公开源代码,本研究将重点关注字节码层面的漏洞检测。因此,所选漏洞应具有较强的字节码分析可行性。重入漏洞、整数溢出漏洞、时间戳依赖漏洞和拒绝服务攻击漏洞均符合这一标准,这一点将在本章节的后续分析中详细阐述。


\section{重入漏洞}

% ...existing code...
\subsection{重入漏洞原理}

重入漏洞源于智能合约在执行外部调用时,其正常的执行流程被中断并“重入”的安全问题。具体而言,当合约 A 调用外部合约 B(例如,通过 call 发送以太币)时,控制权会暂时转移给合约 B。如果合约 B 被设计为(或本身就是攻击合约)在接收到调用或以太币后,通过其 fallback 或 receive 函数回调合约 A 中的某个函数,而此时合约 A 的状态(如账户余额)尚未更新,那么这次重入的调用就可能再次通过初始的检查,执行关键操作,导致状态异常或资产被重复提取。

该漏洞的核心在于未能遵循“检查-生效-交互”(Checks-Effects-Interactions)的最佳实践模式。安全的做法应是在执行任何外部调用(交互)之前,完成所有必要的条件检查(检查)并更新合约状态(生效)。然而,在存在重入漏洞的合约中,状态更新操作(如扣减余额的 SSTORE 指令)被置于外部调用(如 CALL 指令)之后。这就为攻击者创造了机会:在状态更新之前,通过外部调用触发回调,重新进入函数,利用未更新的状态再次通过检查,从而反复执行恶意操作。Solidity 中的 call 函数因其允许发送以太币并能触发接收方的代码执行(通过 fallback 或 receive),是导致重入攻击的主要途径。相比之下,delegatecall 通常不涉及此类重入风险,而 staticcall 则禁止状态更改,从而天然免疫此类攻击。因此,不当的外部调用机制、可被利用的回调函数以及错误的状态更新时机共同构成了重入漏洞的成因。

% ...existing code...
\subsection{重入漏洞实例}

图\ref{fig:重入漏洞合约}展示了一个包含重入漏洞的简化合约示例,包括目标合约 EtherStore 和攻击合约 Attack。EtherStore 允许用户存入和取出以太币,其状态变量 balance 记录各账户余额。关键在于 withdraw 函数:它先检查用户余额是否足够,然后通过 msg.sender.call 向用户发送以太币,最后才更新用户的余额记录,这种“检查-交互-生效”的顺序存在重入风险。
\begin{figure}[h]
    \raggedbottom
    \centering
    \begin{minipage}{0.9\textwidth}
    \begin{minted}[
        fontsize=\large,
        linenos,
        numbersep=5pt,
        xleftmargin=2em,
        baselinestretch=1.0
    ]{solidity}
contract EtherStore {
    mapping(address => uint256) public balance;
    function deposit() external payable {
        balance[msg.sender] += msg.value;
    }
    function withdraw(uint256 _amount) external {   // 漏洞点
        require(balance[msg.sender] >= _amount);
        (bool result,) = msg.sender.call{value: _amount}("");
        require(result, "Failed to withdraw Ether");
        balance[msg.sender] -= _amount;
    }
}
contract Attack {
    EtherStore public etherstore;
    constructor(address _etherStoreAddress) public {
        etherstore = EtherStore(_etherStoreAddress);
    }
    fallback() external payable {
        if (address(etherstore).balance >= 1 ether) {
            etherstore.withdraw(1 ether);
        }
    }
    function attack() external payable {
        require(msg.value >= 1 ether);
        etherstore.deposit{value: 1 ether}();
        etherstore.withdraw(1 ether);
    }
}
\end{minted}
\end{minipage}
\caption{重入漏洞合约}
\label{fig:重入漏洞合约}
\end{figure}


图\ref{fig:重入攻击流程示意}演示了攻击流程,具体步骤如下:
\begin{figure}[h]
    \centering
    \includegraphics[width=1.0\linewidth]{重入攻击流程示意}
    \caption{重入攻击流程示意}
    \label{fig:重入攻击流程示意}
\end{figure}

Attack 合约用于演示如何利用此漏洞。其构造函数接收 EtherStore 的地址。核心在于 fallback 函数,当 Attack 合约接收到以太币时,它会检查 EtherStore 的余额,如果足够,就再次调用 EtherStore 的 withdraw 函数。attack 函数则是攻击的入口点,它先向 EtherStore 存入 1 Ether,然后立即调用 withdraw(1 ether) 尝试取款,从而启动攻击流程。


1) 攻击者调用 Attack 合约的 attack 函数,并发送至少 1 ether。

2)  Attack 合约调用 EtherStore 的 deposit 函数,存入 1 ether。此时 Attack 合约在 EtherStore 中的余额为 1 ether。

3) Attack 合约接着调用 EtherStore 的 withdraw(1 ether) 函数。

4)  EtherStore 的 withdraw 函数检查余额(1 ether >= 1 ether,通过),然后执行 \texttt{msg.sender.call{value: 1 ether}("")}  向 Attack 合约发送 1 ether。

5) 发送以太币触发 Attack 合约的 fallback 函数。此时 EtherStore 中 Attack 合约的余额尚未更新(仍为 1 ether)。

6)  Attack 的 fallback 函数检查 EtherStore 余额(假设大于 1 ether),再次调用 EtherStore 的 withdraw(1 ether) 函数。

7) EtherStore 的 withdraw 函数再次被调用。由于余额未更新,检查再次通过(1 ether >= 1 ether)。EtherStore 再次向 Attack 合约发送 1 ether。

8)  步骤 5-7 重复进行,直到 EtherStore 余额不足或 Gas 耗尽。攻击者在此过程中提取了远超其初始存款的以太币。

总结该实例,重入漏洞的发生通常依赖以下四个条件:

1.  合约通过外部调用转移控制权或价值。

2.  接收方合约能够通过回调重新调用发起方合约。

3.  发起方合约在外部调用完成前未更新关键状态。

4.  重入调用能够再次通过初始检查并执行敏感操作,利用未更新的状态获利。

\subsection{重入漏洞在字节码层面的表征}

由上一节分析可知,重入漏洞的发生在于合约发生外部调用将控制权转移到攻击合约。在 EVM 执行模型中,指令 CALL、DELEGATECALL,STATICCALL 三个指令用于发起外部调用,三个指令中 DELEGATECALL 调用过程继承了调用者的上下文,并且不能用于发送以太币,因此通常不会引起重入攻击,STATICCALL 用于执行静态调用,其主要特点是禁止任何状态修改,这意味着被调用的合约函数不能更改存储变量,也不能进行以太币转账,进一步降低了重入攻击的风险。相比之下,CALL 指令在 EVM 执行过程中具有较高的灵活性,不仅可以用于调用外部合约,并且可以修改存储数据或发送以太币,正是由于这种可变性,攻击者可以利用 CALL 触发合约的回调函数,在余额更新之前反复执行资金转移操作,进而造成重入攻击的安全漏洞。因此接下来本节将从 CALL 指令的执行状态入手提出识别重入漏洞的方案:

如图\ref{fig:重入gadget}所示,通过研究 EtherStore 合约字节码,左边图片是经过简化的重入漏洞在 CALL 指令执行前后的特征片段,右边则是在 CALL 指令调用前的栈状态,综合 CALL 指令前后的特征和栈状态,可以识别出 CALL 指令的执行状态,进而判断是否存在重入漏洞。综合来看,重入漏洞的核心是在状态更新 SSTORE 之前执行了可能导致回调的外部调用 CALL。在 EVM 字节码,可以寻找以下特征模式:

1. 状态读取 SLOAD:首先从存储中读取需要检查的状态变量值,通常是余额或其他关键状态变量;

2. 条件检查 JUMPI:基于读取的状态进行条件判断,如 require(balance >= amount),如果条件不满足则跳转到 revert 路径;

3. 外部调用 CALL:执行外部调用,通常是向攻击者控制的合约发送以太币或调用其函数。同时 CALL 指令的调用栈满足 value 字段不为零,gas 字段大于 2300,前者是保证重入攻击的经济意义,后者是保证攻击者合约的回调函数能够执行成功;

4. 状态写入 SSTORE:在外部调用指令执行完成之后,才执行更新状态变量的 SSTORE 指令。

识别出这种 SLOAD ->  JUMPI -> CALL -> SSTORE 的字节码执行顺序,同时检测 CALL 指令发生时栈参数中的 value 和 gas 是否满足条件是检测重入漏洞的关键。


\begin{figure}[h]
    \centering
    \includegraphics[width=1.0\linewidth]{gadget-Gadget+call栈}
    \caption{重入漏洞字节码特征}
    \label{fig:重入gadget}
\end{figure}


\subsection{重入漏洞检测方法}



前述分析表明,重入漏洞的检测需识别 SLOAD -> JUMPI -> CALL -> SSTORE 的指令执行序列,并确认 CALL 指令发生时的上下文条件。仅进行静态模式匹配可能因无法确认路径可行性及调用参数而产生误报。

为此,本文提出一种基于符号执行的重入漏洞检测方法。该方法在符号执行过程中,利用状态机模型识别字节码特征序列 SLOAD -> JUMPI -> CALL -> SSTORE 在实际执行路径上的出现。状态机在遇到 SLOAD 和 JUMPI 后依次推进。当遇到 CALL 指令时,若其符号化的 value 参数大于 0 且 gas 参数大于 2300,状态机进入预警状态。若在此状态下接着执行了 SSTORE 指令,则认为发现了一个潜在的重入漏洞路径。

通过结合状态机模式识别与符号执行的路径探索、状态分析能力,可以识别出那些在指令序列、执行上下文和路径可达性上均构成威胁的重入漏洞。算法 \ref{alg:reentrancy-detection} 详细描述了这一检测逻辑。

\begin{algorithm}[H]
    \KwIn{字节码指令序列 $I = \{i_1, i_2, \dots, i_n\}$, 符号执行栈快照序列 $S = \{s_1, s_2, \dots, s_n\}$}
    \KwOut{是否存在潜在重入漏洞路径}
    
    状态 $state \gets S0$\;
    
    \For{$k \gets 1$ \KwTo $n$}{
      $instr \gets i_k$\;
      $stack \gets s_k$ \tcp*{当前指令执行前的栈状态}
    
      \uIf{$state = S0$ \textbf{and} $instr$ is \texttt{SLOAD}}{
        $state \gets S1$\;
      }
    
      \uElseIf{$state = S1$ \textbf{and} $instr$ is \texttt{JUMPI}}{
        $state \gets S2$\;
      }
    
      \uElseIf{$state = S2$ \textbf{and} $instr$ is \texttt{CALL}}{
        \If{$value > 0$ \textbf{and} $gas > 2300$}{
          $state \gets S3$\;
        }
        \Else{
          $state \gets S0$ \tcp*{条件不满足,重置}
        }
      }
    
      \uElseIf{$state = S3$ \textbf{and} $instr$ is \texttt{SSTORE}}{
        \Return \textbf{True} \tcp*{匹配成功,存在重入风险}
      }
    
      \Else{
        $state \gets S0$ \tcp*{任意不匹配则重置状态机}
      }
    }
    \Return \textbf{False}
    \caption{重入漏洞检测算法}
    \label{alg:reentrancy-detection}
\end{algorithm}
    
\section{整数溢出漏洞}
\subsection{整数溢出漏洞原理}



整数溢出漏洞是指由于智能合约编程语言中整数类型的固定位宽限制,当对一个整数变量执行如加法、减法、乘法等算术运算时,其结果超出了该类型能够表示的范围,导致结果被截断或回绕到一个不正确的值,从而引发不可预测的执行逻辑或安全漏洞的问题。

以 Solidity 为例,整型变量的递增规律以  1 字节 64 位为步长递增,具体支持的范围如表\ref{tab:solidity-integer-types}所示:

\begin{table}[htbp]
    \centering
    \caption{Solidity 中常见整数类型的取值范围}
    \label{tab:solidity-integer-types}
    \begin{tabular}{|c|l|l|}
        \hline
        \textbf{类型} & \textbf{有符号范围(int)} & \textbf{无符号范围(uint)} \\
        \hline
        \texttt{8位} & $-2^7$ 到 $2^7 - 1$ & $0$ 到 $2^8 - 1$ \\
        \hline
        \texttt{16位} & $-2^{15}$ 到 $2^{15} - 1$ & $0$ 到 $2^{16} - 1$ \\
        \hline
        \texttt{32位} & $-2^{31}$ 到 $2^{31} - 1$ & $0$ 到 $2^{32} - 1$ \\
        \hline
        \texttt{64位} & $-2^{63}$ 到 $2^{63} - 1$ & $0$ 到 $2^{64} - 1$ \\
        \hline
        \texttt{128位} & $-2^{127}$ 到 $2^{127} - 1$ & $0$ 到 $2^{128} - 1$ \\
        \hline
        \texttt{256位} & $-2^{255}$ 到 $2^{255} - 1$ & $0$ 到 $2^{256} - 1$ \\
        \hline
    \end{tabular}
\end{table}

% \begin{table}[htbp]
%     \centering
%     \renewcommand{\arraystretch}{1.3}
%     \begin{tabular}{|c|l|l|}
%     \hline
%     \textbf{类型} & \textbf{有符号范围(int)} & \textbf{无符号范围(uint)} \\
%     \hline
%     \texttt{8位} & $-2^7$ 到 $2^7 - 1$ & $0$ 到 $2^8 - 1$ \\
%     \hline
%     \texttt{16位} & $-2^{15}$ 到 $2^{15} - 1$ & $0$ 到 $2^{16} - 1$ \\
%     \hline
%     \texttt{32位} & $-2^{31}$ 到 $2^{31} - 1$ & $0$ 到 $2^{32} - 1$ \\
%     \hline
%     \texttt{64位} & $-2^{63}$ 到 $2^{63} - 1$ & $0$ 到 $2^{64} - 1$ \\
%     \hline
%     \texttt{128位} & $-2^{127}$ 到 $2^{127} - 1$ & $0$ 到 $2^{128} - 1$ \\
%     \hline
%     \texttt{256位} & $-2^{255}$ 到 $2^{255} - 1$ & $0$ 到 $2^{256} - 1$ \\
%     \hline
%     \end{tabular}
%     \caption{Solidity 中常见整数类型的取值范围对比}
%     \label{tab:solidity-integer-types}
%     \end{table}
    
整数溢出漏洞分为上溢和下溢,整数上溢是指数值超过其能存储的最大值,例如 uint256 类型的变量,其表示的范围为 \(0\sim (2^{256}-1)\),如果将 \((2^{256}-1)\) 与 \(1\) 相加,其结果将由于环回变为 \(0\)。整数下溢的情况相反,当变量是无符号类型时,最小值为0,如果递减将会下溢,从而得到可能的最大值。

尽管以太坊社区针对智能合约整数溢出漏洞提出了 SafeMath 库用于避免整数漏洞的产生,并且从 Solidity 0.8.0 版本开始内置了算术溢出检查机制,但由于编程过程中的隐式类型转换、使用不安全的第三方库、内联汇编等依然可能导致智能合约面临整数溢出风险。在智能合约开发过程中,隐式类型转换可能发生在不同精度的数据类型之间,尤其是在低位宽整数与高位宽整数之间的运算。例如,当 uint8 类型的变量与 uint256 变量进行算术运算时,Solidity 可能执行自动类型转换,从而使得溢出检查失效。此外,某些第三方库仍然基于旧版本的 Solidity 进行编写,未必实现完整的溢出保护机制,这使得开发者在使用这些库时可能无意间引入安全漏洞。内联汇编允许开发者直接编写低级 EVM 代码,以提高执行效率或实现特殊功能。然而,由于汇编代码绕过 Solidity 的算术安全检查,完全依赖开发者手动执行溢出保护,很容易导致意外行为。
    
    

\subsection{整数溢出漏洞实例}


% 本节将以一个以太坊上经典的溢出攻击作为案例,分析溢出漏洞的形成机制。

% 2018年4月,BEC智能合约操作黑客攻击,黑客凭空取出大量的BEC代币,并在市场上进行抛售,BEC随即急剧贬值,价值几乎跌至零\textsuperscript{\cite{becattack}},这个攻击中攻击者使用的漏洞正是整数溢出漏洞。

% BEC 合约的地址为 0xC5d105E63711398aF9bbff092d4B6769C82F793D,通过以太坊的区块链浏览器可以提取到 BEC 合约的源代码,BEC 合约整数溢出漏洞点在其中一个用于批量转账的函数上,代码如图\ref{fig:整数溢出漏洞合约}所示:

% 为了更具体地阐述整数溢出漏洞的形成机制及其潜在危害,本节将深入分析一个在以太坊历史上具有标志性意义的真实攻击案例——BEC 智能合约攻击事件。通过剖析这次攻击,我们可以清晰地看到理论上的算术溢出风险如何转化为实际的、灾难性的经济损失。

% 该事件发生在 2018 年 4 月,当时名为 Beauty Chain (BEC) 的代币智能合约遭到了攻击。攻击者利用合约中存在的整数溢出漏洞,成功绕过了转账金额的校验逻辑,凭空铸造并提取了天量级的 BEC 代币。这些非法获取的代币随后被迅速抛售到市场上,引发了恐慌性抛售潮,导致 BEC 代币的价格在极短时间内暴跌,其市值几乎瞬间蒸发归零。这一事件不仅给 BEC 代币的持有者带来了毁灭性的打击,也给整个区块链社区敲响了警钟,凸显了智能合约安全审计,特别是针对算术运算安全的重要性。它生动地证明了,看似微小的编程疏忽,在承载巨大价值的智能合约世界中,可能被无限放大,造成不可挽回的后果。

% 遭受攻击的 BEC 合约部署在以太坊主网上,其地址为 0xC5d105E63711398aF9bbff092d4B6769C82F793D。通过 Etherscan 等区块链浏览器,我们可以获取并审查该合约的源代码。经过分析发现,导致此次攻击的关键漏洞点,正位于其一个名为 batchTransfer 的函数中。该函数原本设计用于实现代币的批量转账功能,但在处理转账总额计算时,未能充分考虑整数溢出的可能性。接下来的代码片段(如图\ref{fig:整数溢出漏洞合约}所示)精确地展示了 batchTransfer 函数中存在整数溢出漏洞的具体位置。

为了更具体地阐述整数溢出漏洞的形成机制及其潜在危害,本节将深入分析一个在以太坊历史上具有标志性意义的真实攻击案例——BEC 智能合约攻击事件。
该事件发生在 2018 年 4 月,当时名为 Beauty Chain (BEC) 的代币智能合约遭到了攻击。攻击者利用合约中存在的整数溢出漏洞,成功绕过了转账金额的校验逻辑,凭空铸造并提取了天量级的 BEC 代币。这些非法获取的代币随后被迅速抛售到市场上,引发了恐慌性抛售潮,导致 BEC 代币的价格在极短时间内暴跌,其市值几乎瞬间蒸发归零\textsuperscript{\cite{becattack}}。

遭受攻击的合约地址为 0xC5d105E63711398aF9bbff092d4B6769C82F793D。通过 Etherscan 等区块链浏览器,可以获取并审查该合约的源代码。经过分析发现,导致此次攻击的关键漏洞点,正位于其一个名为 batchTransfer 的函数中。该函数原本设计用于实现代币的批量转账功能,但在处理转账总额计算时,未能充分考虑整数溢出的可能性。如图\ref{fig:整数溢出漏洞合约}的代码片段展示了 batchTransfer 函数中存在整数溢出漏洞的具体位置。


\begin{figure}[H]
    \centering
    \begin{minipage}{0.9\textwidth}
    \begin{minted}[fontsize=\large, linenos, numbersep=5pt, xleftmargin=0em, baselinestretch=1.0]{solidity}
function batchTransfer(
    address[] _receivers, 
    uint256 _value
) public whenNotPaused returns (bool) {
    uint cnt = _receivers.length;
    uint256 amount = uint256(cnt) * _value; //溢出漏洞
    require(cnt > 0 && cnt <= 20);
    require(_value > 0 && balances[msg.sender] >= amount);
    balances[msg.sender] = balances[msg.sender].sub(amount);
    for (uint i = 0; i < cnt; i++) {
        balances[_receivers[i]] = balances[_receivers[i]].add(_value);
        Transfer(msg.sender, _receivers[i], _value);
    }
    return true;
}
    \end{minted}
    \end{minipage}
    \caption{整数溢出漏洞合约}
    \label{fig:整数溢出漏洞合约}
\end{figure}


% 在 BEC 智能合约的 batchTransfer 函数中,整数溢出漏洞源于 uint256 amount = uint256(cnt) * \_value;这一运算,该语句未包含溢出检查。当 \texttt{cnt} 和 \texttt{\_value} 的乘积超出 \texttt{uint256} 类型的存储范围时,数值将发生溢出,导致 \texttt{amount} 变为非预期值。由于 \texttt{cnt} 可由 \texttt{receivers} 参数控制,而 \texttt{\_value} 可直接指定,攻击者可精确构造输入,使其发生溢出。具体而言,假设 $\texttt{cnt} = 2$ 且 $\texttt{\_value} = 2^{255}$,则 $\texttt{amount} = 2 \times 2^{255} = 2^{256}$,超出 \texttt{uint256} 的最大表示范围,导致数值回绕为 0。

% 溢出的直接影响是绕过第 X 行的安全检查。代码 require(\_value > 0 \&\& balances[msg.sender] >= amount); 旨在确保转账金额合法且账户余额充足。然而,当 amount 溢出为 0 后,balances[msg.sender] >= 0 的条件总是成立,攻击者可在账户余额不足的情况下成功执行大额转账。此漏洞可能导致智能合约中的代币流失,引发经济损失。


在 BEC 智能合约的 \texttt{batchTransfer} 函数中,整数溢出漏洞源于语句将 \texttt{cnt} 和 \texttt{\_value} 相乘并赋值给 \texttt{amount},即 \texttt{amount = uint256(cnt) * \_value;},该语句未包含任何溢出检查。当 \texttt{cnt} 和 \texttt{\_value} 的乘积超出 \texttt{uint256} 类型的最大表示范围时,将发生整数回绕,使得 \texttt{amount} 变为非预期值。由于 \texttt{cnt} 可由 \texttt{receivers} 参数控制,而 \texttt{\_value} 可由攻击者直接指定输入,因此可以构造特定参数触发该漏洞。 例如,若令 \texttt{cnt} 为 $2$, \texttt{\_value} 为 $2^{255}$,则计算结果 \texttt{amount} 为 $2 \times 2^{255} = 2^{256}$,超过 \texttt{uint256} 最大值后回绕为 $0$。

溢出的直接影响是绕过原本用于确保转账安全的检查逻辑。该检查要求转账值大于零,且当前账户余额大于等于转账总额。然而,一旦 \texttt{amount} 回绕为 $0$,\texttt{balances[msg.sender] >= 0} 恒为真,从而绕过检查,使攻击者即使余额不足也能完成大额转账。这将可能导致合约内代币被非法转出,造成严重经济损失。

通过本例可以总结出,整数溢出漏洞的发生通常依赖以下几个必要条件:

1. 算术运算:合约中存在对整数变量的加、减、乘等算术运算;

2. 约束检查缺失:合约在执行算术运算后,缺乏对结果的有效约束检查,尤其是对溢出或下溢的检查;

3. 输入控制:攻击者能够控制参与算术运算的输入参数,尤其是涉及乘法或加法的参数;

4. 逻辑依赖:合约的后续逻辑依赖于算术运算的结果,尤其是涉及余额检查或状态更新的逻辑。 


\subsection{整数溢出漏洞在字节码层面的表征}
整数上溢和下溢漏洞直接关联于 EVM 的算术运算指令:ADD(加法)、SUB(减法)和MUL(乘法)。在字节码层面,对这些指令的执行进行细致分析至关重要,特别是当它们的操作数可能导致结果超出 uint256 范围时。EVM 默认对这些运算执行模运算,即结果会发生回绕而不会报错,这是漏洞产生的根本原因。因此,直接检查字节码中的算术指令及其上下文是发现此类漏洞的有效途径,尤其是在缺乏源代码的情况下。

整数溢出漏洞在字节码层面的表征主要体现在以下几个方面:

1. 操作数加载:通过 SLOAD, CALLDATALOAD,MLOAD 等指令将操作数加载到栈中,可能涉及到合约的状态变量或输入参数;

2. 算术运算:使用 ADD、SUB、MUL 指令进行算术运算,这是潜在的漏洞点;

3. 约束检查缺失:缺少 LT/GT/EQ + JUMPI 的约束判断;

4. 结果使用:运算结果被用于后续的逻辑判断或状态更新,如用 SSTORE 存储到状态变量,用于条件判断,作为外部调用的参数。





\subsection{整数溢出漏洞检测方法}

由于整数溢出漏洞发生于加、减、乘三种算数运算时,因此对于溢出漏洞的检测应该关注 \texttt{ADD}、\texttt{SUB}、\texttt{MUL} 三条指令。具体来说,当执行到这些算术指令时,符号执行不仅计算符号化的结果,还会
额外添加表示 “上溢发生” 或 “下溢发生” 的约束到当前路径条件中,例如,对于 c = a + b,添加约束条件为 solve.add(c < a) 或 solve.add(c < b);对于 c = a - b,添加约束条件为 solve.add(a < b);对于 c = a * b,添加约束条件为 solve.add(c / a != b)。利用约束求解器检查包含溢出/下溢约束的路径条件是否可满足
(SAT)。如果 SAT,则说明存在一组输入可以触发该算术运算发生上溢或下溢。算法\ref{alg:integer-overflow-detection}表示了整数溢出漏洞检测的伪代码,算法的输入为字节码指令序列 I 和符号执行栈快照 S,输出为是否存在潜在整数溢出路径。算法的主要思路是通过符号执行引擎对字节码指令进行遍历,当遇到算术运算指令时,添加相应的约束条件,并使用约束求解器检查路径条件是否可满足。




\begin{algorithm}
    \KwIn{字节码指令序列 $I = \{i_1, i_2, \dots, i_n\}$}
    \KwOut{是否存在潜在整数溢出路径}

    \For{$k \gets 1$ \KwTo $n$}{
        $instr \gets i_k$;

        \uIf{$instr$ is \texttt{ADD}}{
            $(a, b) \gets$ 从符号栈中取出两个操作数;
            $result \gets a + b$; 加入约束 $result < a$ ;
        }
        \uElseIf{$instr$ is \texttt{SUB}}{
            $(a, b) \gets$ 从符号栈中取出两个操作数;
            $result \gets a - b$; 加入约束 $a < b$ ;
        }
        \uElseIf{$instr$ is \texttt{MUL}}{
            $(a, b) \gets$ 从符号栈中取出两个操作数;
            \If{$a \neq 0$}{
                $result \gets a * b$; 加入约束 $result / a \neq b$ ;
            }
        }

        \If{符号路径条件包含任意溢出约束为 \texttt{SAT}}{
            \Return \textbf{True} ;
        }
    }

    \Return \textbf{False};
    \caption{整数溢出检测伪代码}
    \label{alg:integer-overflow-detection}
\end{algorithm}
   
\section{时间依赖漏洞}
\subsection{时间依赖漏洞原理}

在以太坊中,智能合约可以通过 block.timestamp (Solidity) 或 TIMESTAMP (EVM 操作码) 获取当前区块的时间戳。这个时间戳并非一个绝对精确的物理时间,而是由负责打包当前区块的矿工在其本地时钟的基础上设定的。区块时间戳是包含在以太坊区块头中的一个 uint64 值,代表此区块被认为创建时的 UTC 时间。以太坊协议为了容忍网络延迟和节点间时钟的微小差异,对时间戳的设定给予了一定的灵活性:矿工设置的时间戳必须严格大于其父区块的时间戳,并且根据黄皮书规定,该时间戳与验证者自身时钟的偏差理论上不应过大,现行规范中规定了这个时间差不超过 900 秒。然而,在这两个约束条件内,矿工拥有选择具体时间戳的自由度。

这种有限的可操纵性构成了时间依赖漏洞的基础。如果合约的关键决策逻辑,例如决定随机数生成(通过对时间戳取模或哈希)、判断资金是否达到解锁时间、确定游戏或抽奖的赢家、或者控制某些操作的时间窗口,直接或间接地基于 block.timestamp 的精确值或其衍生属性(如奇偶性),那么一个有动机的矿工就有可能通过精心选择提交区块的时间,或者在打包交易时微调所设定的时间戳,来影响合约的执行结果,从而为自己或特定方牟取不正当利益。例如,矿工可以延迟包含某个关键交易的区块,直到时间戳满足某个有利条件,或者优先打包能在特定时间戳下获胜的交易。

即使不考虑矿工的恶意操纵,仅仅依赖 block.timestamp 也可能引入非预期行为。以太坊的出块间隔本身具有一定的随机性,平均出块时间只是一个统计值。因此,依赖时间戳进行精确计时的逻辑,可能会因为实际出块时间的波动而产生偏差,导致合约行为不完全符合预期,尤其是在对时间精度要求较高的场景下,例如金融交易或随机数生成。

\subsection{时间依赖漏洞实例}
本节通过分析一个简单的时间戳依赖智能合约实例,阐述了时间戳漏洞的形成过程及其潜在风险。

如图\ref{fig:timestamp-dependency}所示的智能合约实例 Timegame,其核心逻辑为:玩家每次调用 spin 函数时,需要发送 1 ETH 的参与费。合约通过判断当前区块的时间戳是否能被 7 整除来决定是否将合约余额发放给调用者。具体实现如下:

\begin{figure}[H]
    \centering
    \begin{minipage}{0.9\textwidth}
    \begin{minted}[fontsize=\large, linenos, numbersep=5pt, xleftmargin=2em]{solidity}
contract Timegame {
    uint public lastBlockTime;
    address public luckyUser;
    constructor() payable {}

    function spin() external payable {
        require(msg.value == 1 ether);
        require(block.timestamp != lastBlockTime);
        lastBlockTime = block.timestamp;
        if (block.timestamp % 7 == 0) {
            msg.sender.transfer(address(this).balance);
            luckyUser = msg.sender;
        }
    }
}
    \end{minted}
    \end{minipage}
    \caption{时间戳依赖漏洞}
    \label{fig:timestamp-dependency}
\end{figure}

在此合约中,\texttt{block.timestamp} 被用作判断条件,用以决定是否执行资金转移及更新“幸运用户”状态。 \texttt{lastBlockTime} 用于记录上次合约调用的时间戳,以避免同一区块时间戳重复调用。合约的关键判定逻辑是基于 \texttt{block.timestamp \% 7 == 0},即当当前时间戳是 7 的倍数时,合约将把所有的资金转移给调用者。此处存在一个潜在的漏洞:矿工可以操控区块时间戳,使其满足该条件,从而使得他们能够在特定时间戳下调用 \texttt{spin} 函数,确保自己获得合约中的所有余额。

总过本例可以总结出,时间戳依赖漏洞的发生通常依赖以下两个必要条件:

1. 时间戳获取:合约中存在对时间戳的获取,即通过 block.timestamp 语句获取当前区块的时间戳;

2. 决策参与:合约的关键决策逻辑依赖于时间戳的值或其衍生值,例如通过对时间戳进行取模、加法等操作来判断是否执行某些关键操作。

\subsection{时间依赖漏洞在字节码层面的表征}


时间依赖漏洞的核心是在 EVM 字节码中使用了 TIMESTAMP 指令,并且其结果直接或间接地影响了关键的控制流决策通常通过 JUMPI 指令或状态更新 SSTORE。因此其特征模式可以总结为:

1. 时间戳获取:通过 TIMESTAMP 指令获取当前区块的时间戳并压入栈顶;

2. 时间戳处理与比较:对栈顶的时间戳进行 MOD ADD 等操作,计算出时间戳的某种衍生值,并与其他值进行比较,通常是通过 JUMPI 指令进行条件跳转;

3. 关键逻辑执行:根据时间戳本身或其衍生值,决定是否执行某些关键逻辑,例如资金转移、状态更新等,通常通过 SSTORE 指令进行状态更新。

如图\ref{fig:时间戳依赖特征}是对于前述合约 Timegame 的简化字节码序列示例,展示了时间戳依赖漏洞的特征模式。



\begin{figure}[h]
    \centering
    \includegraphics[width=0.3\linewidth]{时间戳依赖特征}
    \caption{时间戳依赖特征}
    \label{fig:时间戳依赖特征}
\end{figure}

\subsection{时间依赖漏洞检测方法}
对时间戳漏洞的检测的关键在于识别 TIMESTAMP 指令的使用,并追踪其结果如何通过比较和 JUMPI 影响到后续的关键操作。为此可以检测代码中所有使用 TIMESTAMP 指令的地方,并将其标记为潜在风险点。然而,这种方法会产生大量误报,因为很多正常业务逻辑也会使用时间戳,如记录事件发生时间等,需要结合上下文判断其是否影响关键逻辑。

因此检测的方式是:首先检测 TIMESTAMP 指令的使用,标记为潜在风险点,跟踪 TIMESTAMP 指令的结果,在后续的 JUMPI 指令时,检查当前的分支条件是否依赖于 TIMESTAMP 指令及其产生的值,如果是则标记当前分支为受 TIMESTAMP 控制的分支。最后在关键操作如 SSTORE、CALL、SELFDESTRUCT 等指令时,检查当前路径中是否存在受 TIMESTAMP 控制的分支,如果存在则说明存在时间戳依赖漏洞,检测算法如\ref{alg:timestamp-dependency-detection}所示。

% 以下是时间戳依赖漏洞检测的伪代码,算法的输入为字节码指令序列 I,输出为是否存在潜在时间戳依赖路径。算法的主要思路是通过符号执行引擎对字节码指令进行遍历,当遇到 TIMESTAMP 指令时,记录其结果为符号变量 t,并标记当前路径状态依赖于 t;当遇到 JUMPI 指令时,提取条件表达式 cond,如果 cond 中包含 t 的依赖,则标记该分支受 TIMESTAMP 控制;当遇到关键操作如 SSTORE、CALL、SELFDESTRUCT 等指令时,如果当前路径中存在受 TIMESTAMP 控制的分支,则说明存在时间戳依赖漏洞。

\begin{algorithm}[H]
    \KwIn{字节码指令序列 $I = \{i_1, i_2, \dots, i_n\}$}
    \KwOut{是否存在时间戳依赖漏洞}
    
    初始化符号执行器与依赖跟踪器\;
    
    \For{$k \gets 1$ \KwTo $n$}{
        $instr \gets i_k$\;
    
        \If{$instr$ is \texttt{TIMESTAMP}}{
            记录其结果为符号变量 $t$\;
            标记当前路径状态依赖 $t$\;
        }
    
        \If{$instr$ is \texttt{JUMPI}}{
            提取条件表达式 $cond$\;
            \If{$cond$ 中包含 $t$ 的依赖}{
                标记该分支受 \texttt{TIMESTAMP} 控制\;
            }
        }
    
        \If{$instr$ is 关键操作如 \texttt{SSTORE}, \texttt{CALL}, \texttt{SELFDESTRUCT}}{
            \If{当前路径中存在受 \texttt{TIMESTAMP} 控制的分支}{
                \Return \textbf{True} \tcp*{存在时间戳依赖漏洞}
            }
        }
    }
    
    \Return \textbf{False}
    \caption{时间戳依赖漏洞检测伪代码}
    \label{alg:timestamp-dependency-detection}
\end{algorithm}
    
\section{拒绝服务攻击漏洞}
\subsection{拒绝服务攻击漏洞原理}

拒绝服务(Denial of Service, DoS)攻击是一类通过使目标系统资源耗尽或进入异常状态而阻止其继续提供服务的攻击方式。在智能合约领域,DoS 攻击的本质在于通过精心构造的交易或合约交互,使目标合约在某些关键逻辑路径上发生异常、失败或永久性停滞,从而使其无法为后续用户提供正常服务。

DoS 攻击的危害体现在多个维度:一方面,它破坏了合约的可用性,使得正常用户无法访问或使用核心功能;另一方面,它可能造成资金冻结,尤其是在涉及自动化支付、返还机制或时间敏感业务场景中。

导致 DoS 漏洞产生的主要原因包括:依赖外部调用结果、无边界循环处理、资源消耗型逻辑未作限制、区块链系统中 gas 资源限制等。在 Solidity 智能合约开发中,开发者常通过诸如 call、send 等指令向外部地址转账或调用合约函数,这些外部调用可能失败,若未妥善处理返回结果,则易引发回滚进而中止整个交易流程。此外,当合约使用循环结构处理用户数据,而用户数量无法预知时,攻击者可构造极端输入使交易消耗 gas 上限而失败,从而阻断正常用户的操作。本节主要关注与 Gas 耗尽和外部调用相关的 DoS。


\textbf{Gas 耗尽型 DoS (Gas-related DoS):}

如果合约中存在某些操作,通常是循环或对动态数组的遍历,其执行次数或处理的数据量依赖于外部输入或不断增长的存储状态。那么攻击者可能通过构造输入,比如向数组中添加大量元素,使得该操作的 Gas 消耗超过区块 Gas 上限(Block Gas Limit)或交易指定的 Gas Limit。一旦发生这种情况,任何调用该操作的交易都会失败并 revert,导致相关功能无法使用。例如,一个需要遍历所有用户并进行退款的函数,如果用户数量变得非常庞大,这个函数就可能永远无法成功执行。

\textbf{外部调用阻塞型 DoS (DoS via Unexpected Revert):}

如果合约的关键状态转换,如标记拍卖结束、释放锁定资金等依赖于一次外部调用,而该外部调用可能因为外部合约的状态、Bug 或者被攻击者操纵而意外 revert,那么这个关键的状态转换就可能永远无法完成,导致合约功能被阻塞。

% \subsection{拒绝服务攻击漏洞实例}
% 图 X 展示了 Gas 耗尽型 DoS 和外部调用阻塞型 DoS 的具体实例,函数的核心功能是组织拍卖并在结束时退款给所有参与者。合约的设计允许用户通过 bid 函数参与拍卖,并存储其存款。合约的 refundAll 函数用于将所有参与者的存款退还给他们,而 finalizeAuction 函数则用于结束拍卖并释放资金。合约代码如下:
% \begin{figure}[htbp]
%     \centering
%     \begin{minipage}{0.9\textwidth}
%     \begin{minted}[fontsize=\small, linenos, numbersep=5pt, xleftmargin=2em]{solidity}
% contract VulnerableAuction {
%     address payable[] public bidders; // 参与者列表
%     mapping(address => uint) public deposits;
%     address public owner;
%     bool public ended;
%     IExternalCheck externalChecker; 
%     constructor(address _checker) {
%         owner = msg.sender;
%         externalChecker = IExternalCheck(_checker);
%     }
%     function bid() public payable {
%         require(!ended, "Auction already ended");
%         bidders.push(payable(msg.sender));
%         deposits[msg.sender] += msg.value;
%     }
%     function refundAll() public {    // 漏洞1: 无界循环导致的 Gas DoS
%         require(msg.sender == owner, "Only owner");
%         require(ended, "Auction not ended");

%         for (uint i = 0; i < bidders.length; i++) {
%             address payable bidder = bidders[i];
%             uint amount = deposits[bidder];
%             if (amount > 0) {
%                 bidder.transfer(amount); 
%                 deposits[bidder] = 0; 
%             }
%         }
%     }
%     function finalizeAuction() public {     // 漏洞2: 外部调用阻塞导致的 DoS
%         require(msg.sender == owner, "Only owner");
%         require(!ended, "Auction already ended");
%         bool canEnd = externalChecker.check();
%         require(canEnd, "External check failed, cannot end auction");
%         ended = true;
%     }
% }
%     \end{minted}
%     \end{minipage}
%     \caption{拒绝服务攻击漏洞合约}
% \end{figure}

% 在上述合约中,refundAll 函数存在 Gas 耗尽型 DoS 漏洞,漏洞的触发条件是 bidders 数组的长度 bidders.length 变得~非常大。bid 函数允许任何人调用并向 bidders 数组添加元素,没有对数组大小施加任何限制。refundAll 函数需要遍历整个 bidders 数组。循环的每一次迭代都需要执行数组读取、deposits[bidder] 映射读取、条件判断、transfer(包含外部调用和状态检查)以及映射写入 deposits[bidder] = 0,即 SSTORE 操作。这些操作都需要消耗 Gas。当 bidders.length 足够大时,执行整个 for 循环所需的累计 Gas 将超过以太坊的区块 Gas 上限 (Block Gas Limit)一旦累计 Gas 超过区块 Gas 上限,任何调用 refundAll 函数的交易都会因 Gas 耗尽而失败并 revert。由于无法通过提高单笔交易的 Gas Limit 来突破区块 Gas 上限,该函数将变得永久不可用,导致所有投标人的押金被永久锁定在合约中。

% 而 finalizeAuction 函数则存在外部调用阻塞型 DoS 漏洞,漏洞的触发条件是外部合约 externalChecker 的 check() 函数调用失败。finalizeAuction 函数的执行流程中,关键的状态转换 ended = true; 发生在对 externalChecker.check() 的调用并检查其返回值之后。如果 externalChecker.check() 由于任何原因,如 externalChecker 合约自身逻辑错误、状态异常、Gas 不足,或者被其所有者恶意设置为 revert 状态而失败并 revert,那么 require(canEnd, ...) 就会失败,导致 finalizeAuction 函数自身也 revert。如果 externalChecker.check() 的失败是持久性的或者可被攻击者重复触发,那么 finalizeAuction 函数将永远无法成功执行,导致状态变量 ended 永远无法变为 true。这会阻塞拍卖的正常结束流程,可能导致后续依赖 ended == true 状态的逻辑,如资金结算、所有权转移等也无法执行。


% 由此我们总结出拒绝服务攻击漏洞的发生通常依赖以下几个必要条件:

% 1. 无界操作:存在一个循环或对动态大小数据结构,通常是动态数组 [], 或 mapping 的迭代操作,该循环的迭代次数或操作的数据量依赖于一个可由外部用户无限制增长的状态变量,如动态数组成员数量或映射大小,并且循环体内部执行了消耗 Gas 的操作。

% 2. 外部依赖:合约执行关键的状态转换或核心功能,如改变核心状态标志、解锁资金、完成阶段性任务,依赖一次外部合约调用的成功执行,外部调用失败会导致包含关键转换/功能的当前函数revert,并且没有提供替代路径或机制来完成该关键转换或功能。

\subsection{拒绝服务攻击漏洞实例}
图\ref{fig:拒绝服务攻击漏洞合约}展示了 Gas 耗尽型 DoS 和外部调用阻塞型 DoS 的具体实例,函数的核心功能是组织拍卖并在结束时退款给所有参与者。合约的设计允许用户通过 bid 函数参与拍卖,并存储其存款。合约的 refundAll 函数用于将所有参与者的存款退还给他们,而 finalizeAuction 函数则用于结束拍卖并释放资金。合约代码如下:


\begin{figure}
    \centering
    \begin{minipage}{0.9\textwidth}
    \begin{minted}[fontsize=\large, linenos, numbersep=5pt, xleftmargin=0em,baselinestretch=1.0]{solidity}
contract VulnerableAuction {
    address payable[] public bidders; // 参与者列表
    mapping(address => uint) public deposits;
    address public owner;
    bool public ended;
    IExternalCheck externalChecker;
    constructor(address _checker) {
        owner = msg.sender;
        externalChecker = IExternalCheck(_checker);
    }
    function bid() public payable {
        require(!ended, "Auction already ended");
        bidders.push(payable(msg.sender));
        deposits[msg.sender] += msg.value;
    }
    function refundAll() public {    // 漏洞1
        require(msg.sender == owner, "Only owner");
        require(ended, "Auction not ended");

        for (uint i = 0; i < bidders.length; i++) {
            address payable bidder = bidders[i];
            uint amount = deposits[bidder];
            if (amount > 0) {
                bidder.transfer(amount);
                deposits[bidder] = 0;
            }
        }
    }
    function finalizeAuction() public {     // 漏洞2
        require(msg.sender == owner, "Only owner");
        require(!ended, "Auction already ended");
        bool canEnd = externalChecker.check();
        require(canEnd, "External check failed, cannot end auction");
        ended = true;
    }
}
    \end{minted}
    \end{minipage}
    \caption{拒绝服务攻击漏洞合约}
    \label{fig:拒绝服务攻击漏洞合约}
\end{figure}

在上述合约中,refundAll 函数存在 Gas 耗尽型 DoS 漏洞,漏洞的触发条件是 bidders 数组的长度 bidders.length 变得非常大。bid 函数允许任何人调用并向 bidders 数组添加元素,没有对数组大小施加任何限制。refundAll 函数需要遍历整个 bidders 数组。循环的每一次迭代都需要执行数组读取、deposits[bidder] 映射读取、条件判断、transfer(包含外部调用和状态检查)以及映射写入 deposits[bidder] = 0,即 SSTORE 操作。这些操作都需要消耗 Gas。当 bidders.length 足够大时,执行整个 for 循环所需的累计 Gas 将超过以太坊的区块 Gas 上限 (Block Gas Limit),一旦累计 Gas 超过区块 Gas 上限,任何调用 refundAll 函数的交易都会因 Gas 耗尽而失败并 revert。由于无法通过提高单笔交易的 Gas Limit 来突破区块 Gas 上限,该函数将变得永久不可用,导致所有投标人的押金被永久锁定在合约中。

而 finalizeAuction 函数则存在外部调用阻塞型 DoS 漏洞,漏洞的触发条件是外部合约 externalChecker 的 check() 函数调用失败。finalizeAuction 函数的执行流程中,关键的状态转换 ended = true; 发生在对 externalChecker.check() 的调用并检查其返回值之后。如果 externalChecker.check() 由于任何原因,如 externalChecker 合约自身逻辑错误、状态异常、Gas 不足,或者被其所有者恶意设置为 revert 状态而失败并 revert,那么 require(canEnd, ...) 就会失败,导致 finalizeAuction 函数自身也 revert。如果 externalChecker.check() 的失败是持久性的或者可被攻击者重复触发,那么 finalizeAuction 函数将永远无法成功执行,导致状态变量 ended 永远无法变为 true。这会阻塞拍卖的正常结束流程,可能导致后续依赖 ended == true 状态的逻辑,如资金结算、所有权转移等也无法执行。


由此可以总结出拒绝服务攻击漏洞的发生通常依赖以下几个必要条件:

1. 无界操作:存在一个循环或对动态大小数据结构,通常是动态数组 [], 或 mapping 的迭代操作,该循环的迭代次数或操作的数据量依赖于一个可由外部用户无限制增长的状态变量,如动态数组成员数量或映射大小,并且循环体内部执行了消耗 Gas 的操作。

2. 外部依赖:合约执行关键的状态转换或核心功能,如改变核心状态标志、解锁资金、完成阶段性任务,依赖一次外部合约调用的成功执行,外部调用失败会导致包含关键转换/功能的当前函数revert,并且没有提供替代路径或机制来完成该关键转换或功能。



\subsection{拒绝服务攻击漏洞在字节码层面的表征}


对于外部调用阻塞型 DoS 漏洞,EVM 字节码中主要关注 CALL 指令的使用。CALL 指令用于向外部合约发送消息或调用函数,其返回值通常会影响后续的逻辑判断和状态更新,由于本文的检测逻辑只能从本合约的字节码进行分析,因此只能检测 CALL 指令的使用,而无法直接分析外部合约的实现逻辑。从这个角度,拒绝服务攻击漏洞和重入攻击漏洞的检测方法类似,都是通过 CALL 指令的使用来判断是否存在潜在的 DoS 漏洞,或者其实很难从字节码层面区分拒绝服务攻击漏洞和重入攻击漏洞。因此,接下来本文从 gas 耗尽型 DoS 漏洞的角度进行字节码层面的表征分析。

图\ref{fig:拒绝服务攻击漏洞字节码表征}是 refundAll 函数对应的字节码序列示例,展示了 Gas 耗尽型 DoS 漏洞的特征模式,循环的次数由 bidder\_length 控制,而由前述分析容易知道 bidder\_length 是受控数据,可以通过反复调用 bid 函数增加 bidder\_length 的值,主循环体中包含了大量显著消耗 Gas 的指令,如 SLOAD、SSTORE、CALL 等,同时在循环体中不对 Gas 消耗进行检查。

\begin{figure}[h]
    \centering
    \includegraphics[width=0.8\linewidth]{拒绝服务攻击漏洞字节码表征}
    \caption{拒绝服务攻击漏洞字节码表征}
    \label{fig:拒绝服务攻击漏洞字节码表征}
\end{figure}    

Gas 耗尽型 DoS 漏洞的字节码特征主要体现在以下几个方面:

1. 循环结构:存在由 JUMPDEST、JUMP、JUMPI 构成的循环结构;

2. 无界条件:循环的终止条件依赖于通过 SLOAD 从存储中读取的、可能无界增长的值,如动态数组的长度;

3. 大循环体:循环体内包含显著消耗 Gas 的指令,特别是 SLOAD、SSTORE、 CALL 等;

4. 退出机制缺失:缺乏在循环内部使用 GAS 指令检查剩余 Gas 并提前安全退出的机制。


\subsection{拒绝服务攻击漏洞检测方法}

由前面两节的分析可知,Gas 耗尽型 DoS 漏洞的核心在于循环结构的存在和循环条件可控性,以及循环体内的高 Gas 消耗指令。检测方法主要是通过对字节码进行静态分析,识别出所有的循环结构,并检查其终止条件是否依赖于外部存储的值,同时检查循环体内是否包含高 Gas 消耗指令,如 SSTORE、CALL 等。最后还需要检查循环体内是否存在 GAS 指令检查剩余 Gas 的机制。如算法\ref{alg:gas_dos_detection}所示:


\begin{algorithm}[H]
    \KwIn{字节码指令序列 $I = \{i_1, i_2, \dots, i_n\}$}
    \KwOut{是否存在可能的 Gas 耗尽型 DoS 漏洞}
    
    识别所有循环结构 $L = \{loop_1, loop_2, \dots\}$,每个 $loop = (entry, body, exit)$\;
    
    \ForEach{$loop \in L$}{
        提取终止条件表达式 $cond$ 对应的指令 $i_{cond}$\;
    
        \If{$cond$ 依赖于 \texttt{SLOAD} 读取的存储值}{
            $terminateByStorage \gets True$\;
        }
        \Else{
            \textbf{continue} \tcp*{终止条件不依赖外部存储,跳过}
        }
    
        $hasGasIntensiveInstr \gets False$\;
        \ForEach{指令 $instr$ in $body$}{
            \If{$instr$ 是高 Gas 指令(如 \texttt{SSTORE}, \texttt{CALL})}{
                $hasGasIntensiveInstr \gets True$\;
                \textbf{break}\;
            }
        }
    
        $hasGasCheck \gets False$\;
        \If{$body$ 中包含 \texttt{GAS} 并结合 \texttt{LT} + \texttt{JUMPI}}{
            $hasGasCheck \gets True$\;
        }
    
        \If{$terminateByStorage$ \textbf{and} $hasGasIntensiveInstr$ \textbf{and} $\lnot hasGasCheck$}{
            \Return \textbf{True} \tcp*{存在 Gas 耗尽型 DoS 漏洞风险}
        }
    }
    
    \Return \textbf{False}
    \caption{Gas 耗尽型 DoS 检测算法}
    \label{alg:gas_dos_detection}
\end{algorithm}
    

\section{本章小结}

本章对智能合约中四种主要的,具有代表性的安全漏洞,即重入漏洞、整数溢出漏洞、时间戳依赖漏洞和拒绝服务攻击漏洞,进行了系统性的、深入的分析。对于每一种漏洞,首先清晰地阐述了其产生的根本原理和潜在危害,并通过具体的 Solidity 代码示例直观地展示了漏洞的形态。随后,本章的核心贡献在于,深入到 EVM 字节码层面,细致地分析了每种漏洞所对应的关键指令序列和特征模式。最后,本文提出了针对每种漏洞的检测方法,给出了相应的伪代码,展示了如何通过静态分析和符号执行等技术手段,自动化地识别和检测这些漏洞。这为后续设计自动化、精准的检测工具奠定了基础。为第四章将要展开的系统设计提供了清晰的技术路线和理论支撑。

\chapter{系统设计与实现}

\section{引言}

本章的核心任务是基于第三章的分析成果,设计并实现一个的智能合约漏洞自动化检测系统。该系统的核心技术路线是基于字节码层面的静态分析和符号执行技术,在不依赖源代码的情况下,实现对四种特定漏洞的有效检测。将重点介绍如何在字节码层面进行静态分析和动态符号执行,结合第三章提炼的字节码特征,快速识别潜在的漏洞风险点,并通过路径探索和约束求解生成验证性输入,最终确认漏洞。

\section{系统总体设计}


有效的智能合约漏洞检测系统的设计,需在准确性、效率及自动化程度等关键维度间寻求最佳平衡。本节将阐述 ByteChecker 系统所遵循的设计原则以及其宏观架构。

\subsection{设计原则}

在设计 ByteChecker 系统时,主要遵循了以下原则:

1. 准确性优先:系统的首要目标是准确地识别目标漏洞。利用符号执行技术对程序路径进行精确分析,结合第三章提炼的字节码特征,最大限度地减少误报和漏报。相比纯粹的模式匹配或启发式方法,符号执行能提供更强的逻辑保证。

2. 效率优化:符号执行面临路径爆炸的固有挑战。因此,在设计中必须考虑效率优化。本系统通过结合轻量级静态分析预筛选、目标导向的启发式路径探索策略、设置合理的分析深度和时间限制等手段,在保证分析深度的同时,力求提高整体分析效率。

3.模块化与可扩展性:系统架构采用模块化设计,将不同的功能解耦,分成如预处理、符号执行、漏洞检测逻辑、报告生成等模块。这便于对单个模块进行独立的改进、测试和替换,也方便未来扩展系统以支持更多类型的漏洞检测或集成其他分析技术。

\subsection{系统架构}

基于上述设计原则,本文提出的 ByteChecker 系统架构如图\ref{fig:系统架构图}所示,主要包括以下六大模块:

\begin{figure}[htbp]
    \centering
    \includegraphics[width=1.0\linewidth]{系统架构图}
    \caption{系统架构图}
    \label{fig:系统架构图}
\end{figure}


\textbf{输入模块:}作为系统的入口,支持三种形式输入,包括 Solidity 源码、EVM 字节码、链上合约地址,并进行初步的格式验证。对于源代码格式的输入,系统调用 Solidity 编译器 solc 经编译后获取 EVM 字节码和 ABI。输入模块将标准化的字节码和 ABI 信息传递给后续的预处理模块。

\textbf{预处理模块:}对输入的字节码进行必要的预处理,为核心分析引擎做准备。主要工作包括: 

\begin{itemize}
    \item 字节码反汇编:将十六进制的字节码串转换为人类可读的 EVM 操作码序列;
    \item 控制流图构建:基于操作码序列,特别是跳转指令,构建合约的 CFG。CFG 是后续静态分析和指导动态符号执行的基础;
    \item 基本块与函数识别:将指令序列划分为基本块,并识别函数入口点;
    \item 元数据提取:结合 ABI,提取函数签名、状态变量信息等元数据。
\end{itemize}
预处理的结果,CFG、基本块列表、指令序列将提供给核心分析引擎。

% TODO:特征库内容结构是什么,怎么被符号执行引擎使用?
\textbf{字节码特征库:}这是一个关键的知识库,存储了第三章中为四种目标漏洞提炼出的 EVM 字节码层面的特征模式。包括特定的指令序列、CFG 结构特征、或与特定指令相关的操作数及状态条件。该库为核心分析引擎提供目标导向信息。

\textbf{核心分析引擎:}这是实现本研究创新点的核心所在,基于符号执行技术。它内部可以进一步细分为几个协同工作的子模块:

\begin{itemize}
    \item 符号执行基础模块:提供符号状态表示、EVM 指令符号化解释、路径条件管理等基础能力。
    \item 静态分析模块:利用符号执行能力和字节码特征库进行轻量级的、目标导向的静态分析,快速识别高风险代码区域。
    \item 动态分析模块:进行深入的、启发式引导的符号执行路径探索,模拟执行,生成测试用例。
    \item 漏洞检测逻辑:在符号执行过程中应用针对四种漏洞的检测规则和断言。
\end{itemize}
% TODO:什么是规则和断言


该引擎与下层的约束求解器 Z3 紧密交互,进行路径可行性检查和模型生成。引擎的最终输出是发现的潜在漏洞信息列表。

\textbf{约束求解器:}使用外部的 SMT 求解器 Z3。核心分析引擎将路径条件发送给求解器,求解器返回结果可满足性和满足条件的变量赋值。

\textbf{漏洞分析模块}:负责对核心分析引擎输出的原始漏洞信息进行后处理。报告应包含漏洞类型、位置字节码偏移量,源代码行号、触发条件描述。


\section{核心分析引擎设计}

核心分析引擎是本系统的技术核心,它整合了字节码分析与符号执行技术,负责执行实际的漏洞检测任务。本节将详细阐述其内部各组件的设计。

\subsection{预处理模块设计}

预处理模块的目标是将输入的原始字节码转化为更有利于后续分析的结构化表示。
预处理模块是连接原始输入与核心分析引擎的关键桥梁,其核心目标是将输入的智能合约转化为一种结构化的、便于后续静态分析和符号执行的中间表示。此过程主要包含以下关键步骤:


\textbf{1. 字节码规范化处理:}
本阶段的目的是对输入模块的原始字节码文件进行规范化处理:首先,区分并提取合约的运行时代码,识别原始字节码包含了部署字节码和运行时字节码,部署代码仅在创建时执行一次,并非后续交互逻辑分析的重点;其次,去除字节码末尾附加的元数据哈希,这部分不属于可执行指令,会干扰分析。经过此步骤,得到纯净的运行时字节码序列。

\textbf{2. 控制流图构建:}
字节码控制流图是理解合约执行逻辑、进行静态模式匹配和指导动态符号执行路径探索的基础结构。构建过程包括:
\begin{itemize}
    \item \textbf{基本块识别:} 遍历规范化后的字节码指令序列,依据跳转指令 (\texttt{JUMP}, \texttt{JUMPI})、跳转目标指令 (\texttt{JUMPDEST}) 以及终止指令 (\texttt{STOP}, \texttt{RETURN}, \texttt{REVERT}, \texttt{INVALID}, \texttt{SELFDESTRUCT}) 来切分基本块,使每个基本块是一段无分支的连续指令序列。
    \item \textbf{边确定:} 根据基本块末尾指令及栈回溯得到的跳转地址解析块之间的控制流关系。顺序执行产生直通边;\texttt{JUMPI} 产生一条条件跳转边和一条直通边;\texttt{JUMP} 产生无条件跳转边。对于跳转目标地址无法在此时静态确定的动态跳转,如目标来自栈顶计算结果或存储读取,预处理阶段会标记此边,其精确目标将在后续的动态符号执行中根据具体路径条件进行解析。
\end{itemize}
最终生成的控制流图以图数据结构表示,节点为基本块对象,边代表控制流转移。

\textbf{3. 函数入口识别:}
为了支持函数级别的分析作为符号执行的起点,需要识别合约中的函数入口点。主要策略是基于 EVM 的函数调用机制:解析合约开头的函数分发器 dispatcher 执行逻辑,分发器设计为根据交易数据的前 4 字节,也即函数选择器,通过一系列比较和跳转,\texttt{EQ}, \texttt{PUSH4}, \texttt{JUMPI},将控制流导向相应的函数体入口。通过分析此分发器执行逻辑,可以识别出大部分公共和外部函数的入口地址及其对应的函数选择器。若输入包含 ABI 文件,则可直接利用 ABI 信息精确识别函数签名和入口。

预处理模块完成后,将输出结构化的合约表示,包括反汇编指令序列、基本块列表、ABI信息以及最重要的控制流图。这些信息为核心分析引擎的静态分析模块提供了模式匹配的基础,并为动态符号执行模块提供了路径探索的地图。



\subsection{符号执行引擎基础}

符号执行引擎是核心分析引擎的基础,负责模拟 EVM 指令的执行,
但操作的是符号值而非具体值,为了模拟 EVM 的执行语义,系统要维护一个符号化的状态表示,将指运行状态和世界状态进行符号化建模。同时对 EVM 指令行为进行模拟,符号化执行。并与约束求解器 Z3 进行交互,检查路径条件的可满足性,生成测试用例。下面是三个关键组件的设计:

\subsubsection{符号化状态表示}
本系统设计了一个数据结构 SymbolicState 来表示 EVM 在某个程序点 pc 处的完整状态,其中一些关键组件需要符号化处理。包括栈 Stack,内存 Memory,存储 Storage,其余信息在符号执行过程中根据分析需要决定是具体化还是符号化。

对于栈来说,系统用一个列表来模拟栈,但列表中的元素是 Z3 的符号表达式对像,如用 BitVecRef 表示一个 256 位的符号值。栈的操作 PUSH/POP/DUP/SWAP 被转化对应的列表操作。

对于内存来说,由于 EVM 内存是一个字节数组。完全的符号化内存模型非常复杂且内存开销大,系统采用了一个简化的模型。利用 Z3 提供的 Array 支持,将内存建模为从地址到字节的映射。MLOAD 和 MSTORE 则转换成对这个 Array 的 select 和 store 操作,这种方法提供了更精确的内存建模。

对于存储来说,其在 EVM 中的模型本来就是映射,因此类似于内存,利用 Z3 的 Array 支持,将存储设计为持久化的键值对。即为从 256 位地址到 256 位值的映射关系 storage\_map。SLOAD(addr) 对应 select(storage\_map, addr),SSTORE(addr, val) 对应 store(storage\_map, addr, val)。

对于路径条件,被符号化为一个 Z3 的布尔表达式对象集合,表示到达当前状态必须满足的所有分支条件的逻辑的与。初始状态为 True,执行过程中每当执行一个条件跳转指令时,都会复制当前状态,分别模拟条件为真和假的两条路径。分别对两条路径执行约束求解,如果满足约束,则将对应的路径条件集合加入路径记录中,如果不满足约束,则弹出对应的分支跳转条件,并丢弃相应的路径,实现对路径的剪枝。

其他状态,包括程序计数器 pc,可用 Gas,msg.value msg.sender calldata 等调用上下文信息,在执行过程中根据指令的语义确定是否需要符号化处理。

\subsubsection{EVM 指令符号化模拟}
%系统为每个 EVM 操作码实现一个处理函数。该函数接收当前的 SymbolicState,根据指令的语义,更新栈、内存、存储、PC、Gas、路径条件等,并返回更新后的一个或多个后继状态,表\ref{tab:evm_symbolic_simulation}是对主要各个指令的模拟逻辑的简要描述,主要选取了栈操作指令 PUSH,运算指令 ADD,存储访问指令 SLOAD,控制流指令 JUMPI,四个具有代表性的指令,对于内存的读写指令 MLOAD/MSTORE 执行逻辑类似于 SLOAD/SSTORE。

为了精确模拟智能合约在 EVM 上的执行行为,本系统为每一个 EVM 操作码精心实现了一个对应的处理函数。这个设计是符号执行引擎的核心,确保了分析过程严格遵循 EVM 的官方规范。每个处理函数接收当前执行点的完整符号状态(SymbolicState),该状态封装了程序计数器(PC)、符号化的栈、内存、存储以及当前的路径条件等关键信息。函数的核心任务是根据特定操作码定义的语义规则,精确地更新这些状态组件。例如,算术指令会创建新的 Z3 位向量表达式来表示计算结果,栈操作指令则修改符号栈的内容和结构,而存储访问指令则通过与 Z3 的数组理论交互来模拟对持久化存储的读写。更新完成后,处理函数会返回一个或多个后继符号状态。对于大多数顺序执行的指令,只返回一个更新后的状态;而对于像 JUMPI 这样的条件分支指令,则会产生两个代表不同执行路径的后继状态。这种对每个操作码的细致模拟是保证分析深度和准确性的基础。

表\ref{tab:evm_symbolic_simulation} 简要展示了部分关键指令的符号化模拟逻辑。为了体现模拟的多样性,我们选取了四种具有代表性的指令进行说明:\texttt{PUSH1} 代表了基本的栈操作和常量加载;\texttt{ADD} 代表了常见的算术运算,其结果需要符号化表示;\texttt{SLOAD} 代表了对合约持久化状态(存储)的访问,涉及符号化的地址和值;而 \texttt{JUMPI} 则代表了核心的控制流转移机制。需要指出的是,内存读写指令 \texttt{MLOAD} 和 \texttt{MSTORE} 的模拟逻辑与存储访问类似,都是通过符号化的地址在相应的状态组件(易失性内存 vs 持久化存储)上进行读写操作,通常也利用 Z3 的数组理论进行建模。除了这些基础指令,对环境信息指令,如 \texttt{TIMESTAMP},系统操作指令,如 \texttt{GAS},的精确模拟同样重要,它们直接关系到特定类型漏洞的检测。

%在众多指令中,有几类值得特别关注。首先是 \texttt{JUMPI} 指令,作为 EVM 中唯一的条件分支指令,它是程序路径分叉的关键所在。如表\ref{tab:evm_symbolic_simulation}所示,遇到 \texttt{JUMPI} 时,符号执行引擎会复制当前状态,并分别创建两个后继状态:一个对应条件为真(跳转发生),另一个对应条件为假(顺序执行)。同时,相应的约束(分别为 $cond \ne 0$ 和 $cond = 0$)会被添加到各自的路径条件中。由于每次分支都会使路径数量加倍,\texttt{JUMPI} 是导致路径爆炸问题的主要根源。因此,一个关键的优化策略是在每次分支后,立即调用 Z3 求解器检查新生成的路径条件的可满足性。如果某个分支的路径条件被判定为不可满足(UNSAT),则意味着该路径在逻辑上不可能到达,可以被安全地剪枝,从而避免无效的探索。其次是 \texttt{CALL} 指令及其变种。在符号执行中,通常不会真正执行对外部合约的调用,因为这会引入巨大的复杂性和状态空间。取而代之的是一种抽象模拟:引擎弹出调用所需的参数(地址、Gas、值、数据等),然后创建符号化的返回值来代表外部调用的结果,主要是调用是否成功的布尔值 (\texttt{success}) 和返回数据 (\texttt{retdata})。这个符号化的 \texttt{success} 值会被压入栈中,而符号化的 \texttt{retdata} 则根据调用约定写入内存模型。这种抽象处理使得分析可以聚焦于当前合约的逻辑,同时保留了对外部调用结果的依赖性分析能力,对于检测重入等涉及外部交互的漏洞至关重要。最后,对于 \texttt{GAS} 指令,一种常见的简化处理是将其返回值模拟为一个具体的、足够大的常量压入栈中,表示剩余 Gas 充足。虽然这简化了分析,但也限制了对精确 Gas 消耗敏感的漏洞(如某些 Gas 耗尽型 DoS)的检测能力。更精细的模型会符号化地跟踪 Gas 消耗,但这会显著增加约束的复杂性。

% ...existing code...
\begin{table}[h] % 使用 [H] 可能需要 float 包
    \centering
    \caption{部分 EVM 指令的符号化模拟逻辑}
    % 使用 tabularx 环境,设置总宽度为 \linewidth
    % 第一列左对齐,第二列使用 X 类型自动换行并左对齐
    \begin{tabularx}{\linewidth}{l >{\RaggedRight}X}
        \toprule
        \textbf{指令} & \textbf{符号化模拟逻辑} \\
        \midrule
        \texttt{PUSH1 val} & 1. 创建一个 256 位的具体位向量值 $symbolic\_val = \text{BitVecVal}(val, 256)$。 \newline % 使用 \newline 强制换行
                           2. 将 $symbolic\_val$ 压入符号栈 $state.stack$。 \newline
                           3. 返回更新后的 $state$。 \\
        \addlinespace % Adds a small vertical space
        \texttt{ADD}       & 1. 从符号栈弹出两个操作数 $op1 = state.stack.pop()$, $op2 = state.stack.pop()$。 \newline
                           2. 计算符号和 $result = op1 + op2$ (使用 Z3 的位向量加法)。 \newline
                           3. 将 $result$ 压入符号栈 $state.stack$。 \newline
                           4. 返回更新后的 $state$。 \\
        \addlinespace
        \texttt{SLOAD}     & 1. 从符号栈弹出一个地址 $addr = state.stack.pop()$。 \newline
                           2. 从符号存储模型 $state.storage$ 中读取值 $val = \text{Select}(state.storage, addr)$。 \newline
                           3. 将 $val$ 压入符号栈 $state.stack$。 \newline
                           4. 返回更新后的 $state$。 \\
        \addlinespace
        \texttt{JUMPI}     & 1. 从符号栈弹出目标地址 $dest = state.stack.pop()$ 和条件 $cond = state.stack.pop()$。 \newline
                           2. 复制当前状态为两个新状态 $state_{true}$ 和 $state_{false}$。 \newline
                           3. 更新 $state_{true}$:设置 $state_{true}.pc = dest$,并将 $cond \ne 0$ 添加到其路径条件。 \newline
                           4. 更新 $state_{false}$:设置 $state_{false}.pc = \text{current\_pc} + 1$,并将 $cond = 0$ 添加到其路径条件。 \newline
                           5. 返回包含两个后继状态的列表 $[state_{true}, state_{false}]$。 \\
        \bottomrule
    \end{tabularx} % 使用 \end{tabularx}
    \label{tab:evm_symbolic_simulation}
\end{table}
% ...existing code...

%需另外指出的是,JUMPI 指令作为唯一分支指令,如算法所示,是产生新路径的关键,也是路径爆炸问题的来源,因此可以在这里进行路径剪枝,即在每次 JUMPI 指令执行后,使用 Z3 检查路径条件的可满足性,如果不可满足,则立即剪掉该路径;CALL 指令的符号执行并不会真正调用外部合约,而是将其参数依次弹出栈,然后分别符号化其返回值 retdata 和 success ,将 retdata 加入 SymbolicState 所管理的内存中,将 success 加入栈中;对于 GAS 指令则是符号化一个常量压入栈中,表示当前的 Gas 量。




\subsubsection{与约束求解器的交互}
在符号执行过程中,符号值的传播和路径条件的积累是核心任务之一。然而,仅有符号表达式的记录还不足以支持漏洞检测、路径可行性分析等更高层次的任务。为此,系统需集成一个强大的约束求解组件,用于判断符号路径是否可行、变量是否满足某些条件、以及在必要时生成具体测试输入。

本系统选用 Z3 作为底层约束求解器,其支持丰富的 SMT 理论,尤其是对位向量(BitVec)和数组理论的良好支持,能够高效处理 EVM 字节码中常见的算术和存储访问模式。

为了屏蔽底层复杂的 API 调用并提高模块化程度,系统设计了一个高层封装的 Solver 模块。该模块提供了一组简洁的功能接口,用于与 Z3 进行交互,具体功能如表 \ref{tab:solver_interface} 所示:


\begin{table}[h]
    \centering
    \caption{Solver 模块接口功能}
    \begin{tabularx}{\linewidth}{ll >{\RaggedRight}X}
        \toprule
        \textbf{接口} & \textbf{功能描述} & \textbf{主要应用场景} \\
        \midrule
        \texttt{is\_sat()} & 检查给定约束集合的可满足性 & 判断 JUMPI 等条件跳转的分支路径是否可行,用于路径剪枝 \\
        \addlinespace % 可选,增加行间距
        \texttt{get\_model()} & 获取满足约束集合的一个模型 & 当检测到漏洞路径时,生成触发漏洞的具体输入(测试用例) \\
        \addlinespace % 可选
        \texttt{push()} & 保存当前约束求解器的上下文状态 & 在探索新的分支路径前调用 \\
        \addlinespace % 可选
        \texttt{pop()} & 恢复到上一次 \texttt{push()} 保存的状态 & 在分支探索完成或回溯时调用 \\
        \bottomrule
    \end{tabularx} % 使用 \end{tabularx}
    \label{tab:solver_interface}
\end{table}
% ...existing code...


通过上述接口封装,核心分析引擎得以以高度模块化的方式与 Z3 解耦,避免了直接依赖复杂的 Z3 API。这种设计不仅简化了引擎的实现逻辑,也为未来替换其他 SMT 求解器或引入并发求解等高级策略提供了良好的扩展性。

% ... (subsequent sections) ...
\subsection{静态分析模块设计}

本模块的设置目的是在进行完整的动态符号执行之前,结合第三章四种漏洞的特征表征,采用轻量级的字节码模式匹配和污点分析,聚焦于识别与已知漏洞模式相关的局部代码结构和数据流。快速筛选高风险代码区域,生成漏洞候选点列表,为后续的动态分析提供引导,提高整体效率。接下来分别介绍静态分析模块的两个关键步骤:

\subsubsection{字节码模式匹配}

这是静态分析的第一步,对预处理模块生成的指令序列进行扫描,匹配第三章中定义的字节码特征模式:

\begin{itemize}
    \item 重入漏洞: 查找 CALL 指令,并检查其后的基本块,检查 CFG 可达的块中是否存在能被 CALL 影响的状态变量的 SSTORE 指令。
    \item 整数溢出漏洞: 查找 ADD, SUB, MUL 指令,检查它们是否被 SafeMath 库的典型字节码模式所保护,即额外的检查和 JUMPI。如果一个算术指令没有被这种模式保护,则将其标记为可疑点。    
    \item 时间依赖漏洞: 查找 TIMESTAMP 指令,并检查其结果是否直接或间接影响了后续 JUMPI 指令的条件。
    \item 拒绝服务攻击漏洞: 通过后向边识别查找控制流图中的循环结构。
\end{itemize}

对字节码进行模式匹配之后,将匹配到这些模式的代码位置标记为可疑点,作为污点分析的起点。  



\subsubsection{污点分析}

这是静态分析的第二步,对于每个可疑点,系统通过污点分析检查其涉及的关键操作数或变量是否被标记为“污点”,从而进一步筛选出高风险的可疑点。定义来自合约外部的、不可信或可被用户控制的数据源标记为污点源。污点源包括 CALLDATALOAD, CALLER, CALLVALUE 等指令的结果,采用简化的 Worklist 的不动点算法从可疑点出发,分别进行反向追溯和有限步数前向探索,追溯或前向探索追置一个明确的步数上限或基本块上限,如果追溯过程中遇到函数入口或无法解析的依赖,则停止该方向的追溯。然后继续沿着CFG 做前向探索,前向符号执行有限步数,观察该指令的结果如何被使用,特别是如何传递到关键指令或影响控制流。四种漏洞的污点检查逻辑如下:

\begin{itemize}
    \item 重入漏洞:检查 CALL 的目标地址 recipient 或发送的 value 是否是污点;检查在 CALL 之前读取的、用于条件判断的状态变量是否是污点;检查 CALL 之后被 SSTORE 更新的状态变量是否是污点;
    \item 整数溢出: 检查未受保护的 ADD/SUB/MUL 的操作数是否至少有一个是污点。
    \item 时间依赖: 检查 TIMESTAMP 的值是否被前向传播到 JUMPI 条件表达式点。
    \item 拒绝服务攻击:检查循环体内的存储变量(如数组长度)是否是污点,检查 CALL 的目标地址或参数是否是污点。
\end{itemize}

在污点分析阶段完成后,只有当特定代码位置同时符合已定义的漏洞模式匹配条件,并且关键变量受到污点传播影响时,该位置才会被标识为潜在漏洞候选点,用以引导后续的动态分析模块。静态分析模块的输出是一个漏洞候选点列表,包含每个候选点的字节码偏移量、函数名、污点源信息。

\subsection{动态分析模块设计}

动态分析模块是系统的核心执行引擎,它通过模拟合约在符号输入下的执行,目的是对合约进行深入的路径分析,确认静态分析发现的候选点,发现新的漏洞,利用约束求解来验证漏洞条件并生成测试用例。

动态分析模块的本质是根据 CFG 进行路径探索,模拟 EVM 指令的执行。其设计围绕以下几个关键方面展开:

\subsubsection{路径探索策略}
在动态分析模块中,本文采用深度优先搜索作为基础的路径探索策略,以其内存效率高、易于与求解器上下文管理(push()/pop()操作)配合的优势,深入模拟合约执行。为避免纯 DFS 在复杂合约中陷入无关路径或路径爆炸,系统引入多维度的启发式引导机制:首先,针对静态分析模块标记的漏洞候选点,系统将“到达候选点”作为搜索目标,借鉴 A*\textsuperscript{\cite{Hart1968}} 算法思想,设计基于控制流图中节点距离的启发函数,使得更接近候选点的状态优先被探索;其次,依据第三章中提炼的字节码特征模式,对每个待探索状态计算与漏洞特征的匹配度,并优先调度匹配度最高的状态;此外,可选地整合运行时的基本块或边覆盖率信息,优先探索能够覆盖新代码或历史覆盖率最低区域的路径,以扩展分析广度。  

在路径剪枝方面,算法首先对搜索过程中每一条边记录访问次数,系统全局变量设置边的访问次数上限,每次访问一条边,则该条边的访问次数加一;当访问次数超过上限时,系统将该条边标记为不可达,避免后续的路径重复探索;其次,在每次分支前调用 SMT 求解器的可满足性检查(is\_sat),若路径条件不可满足,即约束返回 UNSAT,则立即回溯并舍弃该分支。  

为了进一步控制资源开销,对 DFS 的递归深度和单路径长度施加上限,并为整个分析过程或单条路径探索设置时间限制;在与 Z3 的交互中,还为每次 check() 调用配置超时时间,超时结果可视为 UNKNOWN 或根据保守策略继续探索。通过上述多策略的有机结合,动态分析模块既能够优先覆盖高风险路径,又能在可控的资源范围内保证充分的路径探索,从而高效、准确地定位和验证智能合约中的潜在漏洞。


\subsubsection{状态管理与约束求解}
在动态符号执行过程中,符号状态的高效管理与约束求解性能直接影响路径探索的深度和广度。首先,对于状态复制,系统在每个分支点必须生成后继状态;尽管 Python 中可借助 copy.deepcopy() 实现,但对包含大量 Z3 表达式的 SymbolicState 对象而言,这种全量复制开销极大。为此,本文采用写时复制(Copy‑on‑Write)策略,仅在状态发生实质性变更时才复制相关子结构。

其次,在约束求解器交互方面,本文采用了三项优化措施以提升整体性能:查询缓存:针对相同或高度相似的路径条件,将 Z3 的可满足性检查结果缓存起来。当新路径条件与缓存键匹配时,直接复用先前的判定结果,减少对求解器的重复调用;增量求解:利用 Z3 的 \texttt{push()/pop()} 接口,在分支探索时仅向当前上下文增添新增约束,分支结束后通过回滚操作恢复先前状态,从而避免每次都重新提交整条路径的全部约束;约束简化:在将路径条件发送给 Z3 之前,对布尔表达式和算术式进行局部化简和恒真子表达式消除,同时对同一子表达式多次出现进行共识提取,将冗余的线性约束合并。减轻求解器的负担。

这些优化策略可以有效减少状态复制和求解器调用的开销,从而提升符号执行引擎在处理大型合约时的性能和稳定性。

\subsubsection{测试用例生成}
在符号执行过程中,一旦检测逻辑发现某条路径 \(P\) 同时满足路径条件和漏洞触发条件 \(C_{\mathrm{vuln}}\),即 \(\mathrm{is\_sat}(P.\mathrm{path\_condition}\land C_{\mathrm{vuln}})\) 返回 SAT,系统便会调用 solver.model() 提取出一个满足这些约束的模型 \(M\)。模型 \(M\) 中包含了所有符号输入变量的具体赋值,接下来系统从中还原出与合约调用相关的各项输入:首先,如果在符号执行中将调用者地址 msg.sender 和交易附带的以太币数量 msg.value 作为符号变量建模,则直接从 \(M\) 中提取它们的具体值;其次,根据模型中与 CALLDATALOAD 或函数参数相关的符号变量值,逆向构造出完整的 calldata 字节流,包括 4 字节函数选择器和 ABI 编码的参数。若漏洞依赖于区块属性,如 block.timestamp,则同样从模型中获取相应的符号变量值。最后,系统将目标合约地址 msg.sender、msg.value、calldata 以及可选的区块属性等信息组合成可自动化重放的交易脚本,从而生成完整的、可复现的漏洞测试用例。

%此处应该有图片


\subsubsection{环境建模}
为了在字节码层面有效地进行漏洞检测,符号执行引擎需要对合约运行的区块链环境进行精确且高效的建模。这涉及到对交易上下文、区块链状态、账户信息以及外部交互等多个维度的模拟。本系统采用以下具体的环境建模策略,分别针对调用上下文、区块信息、账户状态与存储、以及外部调用进行阐述:

首先,精确模拟合约与外部环境的交互,特别是交易相关的上下文信息至关重要。系统对关键的调用上下文变量采用了如下建模策略,详见表 \ref{tab:call_context_modeling}:

\begin{table}[H] % 使用 [H] 可能需要 float 包,确保表格出现在此处
    \centering
    \caption{调用上下文变量建模策略}
    \label{tab:call_context_modeling}
    \begin{tabular}{lll}
        \toprule
        \textbf{上下文变量} & \textbf{建模策略} & \textbf{主要目的} \\
        \midrule
        \texttt{msg.sender} & 符号地址变量 & 探索不同调用者身份,检测访问控制漏洞 \\
        \texttt{msg.value} & 符号 uint256 位向量 & 分析以太币转移逻辑 \\
        \texttt{tx.origin} & 符号地址变量  & 检测依赖 \texttt{tx.origin} 与 \texttt{msg.sender} 区分的漏洞 \\
        \texttt{tx.gasprice} & 具体常量值 & 对核心逻辑影响小,简化分析 \\
        \texttt{calldata} & 符号字节数组 & 完全符号化以探索所有函数调用和参数组合 \\
        \bottomrule
    \end{tabular}
\end{table}

该建模方案旨在通过符号化调用者、发送金额、调用数据等关键输入来最大化路径覆盖和漏洞发现能力,同时对影响较小的变量,如 Gas 价格,采用具体值以控制分析复杂度。

其次,对于区块相关信息,其建模侧重于捕捉可能影响合约逻辑的时间与区块高度变化,如表 \ref{tab:block_info_modeling} 所示:

\begin{table}[H] % 使用 [H] 可能需要 float 包
    \centering
    \caption{区块链信息建模策略}
    \begin{tabularx}{\linewidth}{ll >{\RaggedRight}X}
        \toprule
        \textbf{组件/指令} & \textbf{建模策略} & \textbf{主要目的} \\
        \midrule
        \texttt{block.timestamp} & 符号 uint256 & 检测时间戳依赖 \\
        \texttt{block.number} & 符号 uint256 & 检测区块号依赖 \\
        \texttt{block.difficulty}, \texttt{gaslimit}, \texttt{basefee} & 具体常量值 & 简化分析,对常见漏洞模式影响较小 \\
        \bottomrule
    \end{tabularx}
    \label{tab:block_info_modeling}
\end{table}



接着,账户状态与合约存储是合约的核心数据载体,对其进行精确建模是分析状态相关漏洞的基础,具体策略见表 \ref{tab:account_state_modeling}:

\begin{table}[H] % 使用 [H] 可能需要 float 包
    \centering
    \caption{账户状态与存储建模策略}
    \begin{tabularx}{\linewidth}{lll}
        \toprule
        \textbf{组件/指令} & \textbf{建模策略} & \textbf{主要目的} \\
        \midrule
        \texttt{address(this)} & 具体常量地址 & 使用被分析合约的已知地址 \\
        \texttt{address(this).balance} & 符号/具体 uint256 & 初始值可选,根据交易和调用更新 \\
        \texttt{address(any).balance} & 独立符号 uint256 & 建模任意账户余额的访问 \\
        \texttt{storage} & 符号映射 uint256 => uint256 & 精确处理 \texttt{SLOAD}/\texttt{SSTORE} \\
        \bottomrule
    \end{tabularx}
    \label{tab:account_state_modeling}
\end{table}

最后,外部调用是合约交互的关键,也是重入等漏洞的常见触发点。对其建模需平衡精度与复杂性,如表 \ref{tab:external_call_modeling} 所述:

\begin{table}[H] % 使用 [H] 可能需要 float 包
    \centering
    \caption{外部调用建模策略}
    \small % 可选,如果表格太宽,缩小字体
    \begin{tabularx}{\linewidth}{lll}
        \toprule
        \textbf{组件/指令} & \textbf{建模策略} & \textbf{主要目的} \\
        \midrule
        目标地址 & 具体常量 / 符号地址 & 处理静态和动态计算的目标 \\
        返回值 (Success) & 符号布尔变量 & 探索成功/失败路径\\
        返回数据 & 符号字节数组 & 支持对返回数据的后续使用分析 \\
        ETH 转账 & 精确更新余额 & 根据调用的 value 更新 \\
        存储修改 (\texttt{CALL}) & 假设无修改 & 简化,避免状态爆炸 \\
        存储修改 (\texttt{DELEGATECALL}) & 精确更新当前存储 & 在当前上下文执行,允许修改 \\
        存储修改 (\texttt{STATICCALL}) & 禁止状态修改 & 严格执行只读调用 \\
        重入风险处理 (\texttt{CALL}) & 条件性激活模型 & 若目标可能为 \texttt{msg.sender},允许回调重入分析 \\
        \bottomrule
    \end{tabularx}
    \label{tab:external_call_modeling}
\end{table}

综上所述,该整体环境建模方案优先考虑了重入、整数溢出、时间戳依赖、拒绝服务攻击直接相关的环境因素的符号化,同时对较少直接关联的因素进行了简化,旨在实现漏洞检测精度和分析效率之间的平衡。

% \subsection{漏洞检测逻辑实现}


% 此模块负责在动态符号执行过程中,实时应用检测规则,判断当前路径是否触发了目标漏洞。这些规则直接源自第三章对漏洞原理和字节码特征的分析。

% 核心思想: 将漏洞定义为符号执行状态(栈、内存、存储、PC、路径条件)需要满足的特定约束或断言。在执行每条指令后或在特定指令(如 CALL, ADD, JUMPI, SSTORE)执行时,检查这些断言。  

% \subsubsection{重入漏洞检测逻辑}
% 针对重入漏洞的检测,本系统利用符号执行深度探索合约执行路径。检测逻辑的关键在于识别并分析潜在的“检查-生效-交互”(Checks-Effects-Interactions, CEI)模式违规。具体流程设计如下:

% 1.  \textbf{识别潜在风险点:} 符号执行引擎首先关注执行外部交互的 \texttt{CALL} 指令。根据重入漏洞的常见特征,系统将重点分析那些 Gas 大于 2300(允许状态修改)且发送的 Value 可能非零(涉及 ETH 转移)的 \texttt{CALL} 指令,同时排除状态只读的 \texttt{STATICCALL}。

% 2.  \textbf{记录调用前状态:} 在执行一个被识别为潜在风险点的 \texttt{CALL} 指令之前,引擎精确记录与该调用相关的关键状态变量(例如,在转账场景中,调用者的余额 \texttt{balances[msg.sender]})的符号表示 \(B_{\mathrm{before}}\),以及到达该 \texttt{CALL} 指令的路径条件 \(PC_{\mathrm{call}}\)。

% 3.  \textbf{模拟回调:} 为了模拟外部调用可能产生的回调,引擎采用创建状态分支的策略。具体地,在模拟的外部 \texttt{CALL} 返回控制权之后,但在合约状态(如余额 \(B_{\mathrm{before}}\) 被更新为 \(B_{\mathrm{after}}\))通过 \texttt{SSTORE} 等指令实际持久化之前,系统创建一个新的符号执行状态分支。这个新的分支代表了攻击者通过回调函数重新进入当前合约的执行路径。

% 4.  \textbf{探索重入路径:} 引擎接着对新创建的状态分支进行符号执行,目标是探索是否存在一条路径能够再次到达原始 \texttt{CALL} 指令点,并记录这条重入路径的条件 \(PC_{\mathrm{reentry}}\) 以及在该路径上再次读取到的关键状态变量的值 \(\text{balance\_read\_in\_reentry}\)。

% 5.  \textbf{核心可满足性检查:} 获取到重入阶段状态变量的值 \(\text{balance\_read\_in\_reentry}\)后,系统构建组合约束 \(PC_{\mathrm{call}} \land PC_{\mathrm{reentry}} \land (\text{balance\_read\_in\_reentry} = B_{\mathrm{before}})\)。此约束结合了到达初始调用的条件、重入路径的条件以及在重入时读取到的状态值等于调用前状态值的条件。

% 6.  \textbf{确认漏洞:} 引擎利用 SMT 求解器检查上述组合约束的可满足性。若求解器返回“可满足”(SAT),则证明存在一条实际可行的路径,使得攻击者能够在关键状态更新之前,利用旧的状态值 (\(B_{\mathrm{before}}\)) 重新进入合约并执行操作。此时,系统确认检测到重入漏洞。



% \subsubsection{整数溢出漏洞检测逻辑}
% 整数溢出与下溢是智能合约中常见的算术风险,本系统通过符号执行对算术指令进行精确分析以检测此类漏洞。具体检测逻辑遵循以下步骤:

% 1.  \textbf{聚焦关键指令:} 检测逻辑首先识别出执行算术运算的关键 EVM 指令,主要关注 \texttt{ADD}、\texttt{SUB} 和 \texttt{MUL},因为它们是整数溢出或下溢最常发生的地方。

% 2.  \textbf{上下文感知与风险识别:} 系统会考虑重要的上下文信息,特别是 Solidity 编译器版本(0.8.0 前后行为不同)以及是否存在 \texttt{unchecked} 块。通过分析字节码特征或利用编译信息,判断当前的算术操作是否受到内置溢出检查(如 SafeMath 或 Solidity 0.8+ 默认行为)的保护。若无法确定或明确处于无保护环境中,系统将保守地将该算术指令视为一个潜在的溢出风险点。

% 3.  \textbf{符号化操作数:} 当符号执行引擎遇到一个被视为潜在风险点的算术指令时,它会将其操作数从符号栈中弹出,并表示为符号变量(例如,对于二元操作,表示为 \(a\) 和 \(b\)),运算结果也表示为符号变量(例如 \(c\))。




% 4.  \textbf{生成精确的溢出/下溢约束:} 针对具体的操作指令,引擎生成相应的数学约束来精确描述溢出或下溢发生的条件 \(C_{\mathrm{overflow}}\)。本系统利用 Z3 求解器提供的内置位向量算术谓词来实现此功能。对于加法 (\texttt{ADD}),使用 \texttt{BVAdd\_Overflow(a, b)};对于减法 (\texttt{SUB}),使用 \texttt{BVSub\_Overflow(a, b)}(通常表示下溢);对于乘法 (\texttt{MUL}),使用 \texttt{BVMul\_Overflow(a, b)}。
% 这种方法直接利用了求解器优化过的高效实现,确保了约束生成的准确性和效率。


% 5.  \textbf{结合路径条件:} 生成溢出/下溢约束后,引擎获取到达当前指令的完整路径条件 \(PC\)(即到达该点必须满足的所有分支条件的逻辑合取)。然后,将溢出/下溢约束与路径条件结合,形成组合条件 \(PC \land C_{\mathrm{overflow}}\)。

% 6.  \textbf{SMT 求解验证:} 系统将此组合条件提交给 SMT 求解器进行可满足性检查。如果求解器返回“可满足”(SAT),则证明存在一组输入(满足路径条件 \(PC\))能够在当前执行路径上触发该算术指令的整数溢出或下溢(满足 \(C_{\mathrm{overflow}}\))。

% 7.  \textbf{生成完整报告:} 一旦验证成功,系统将判定存在整数溢出/下溢漏洞,并生成详细报告。报告内容包括发生溢出的具体位置(程序计数器)、涉及的操作数的符号表示、触发漏洞的完整路径条件,并且可以进一步调用求解器获取满足条件的模型,生成导致溢出的具体测试用例(输入值)。


% \subsubsection{时间戳依赖漏洞检测逻辑}

% 本系统对时间戳依赖漏洞的检测逻辑,旨在精确识别因错误依赖区块时间戳而导致的可被利用的关键行为差异。具体流程如下:

% 1.  \textbf{识别核心机制:} 检测逻辑首先关注 \texttt{TIMESTAMP} 指令,将其返回值符号化为 \(sym\_ts\)。随后,在遇到条件跳转指令 \texttt{JUMPI} 时,若其跳转条件 \texttt{Cond} 的符号表达式依赖于 \(sym\_ts\),则识别出这是一个潜在的时间戳依赖分叉点。系统认为,此依赖关系是导致关键行为可能因时间戳不同而产生差异的核心机制。

% 2.  \textbf{符号执行探索分支:} 识别分叉点后,系统记录到达该点的基础路径条件 \(PC_{\mathrm{base}}\)。接着,利用符号执行分别探索两条分支路径:条件为真的路径,其路径条件为 \(PC_{\mathrm{true}} = PC_{\mathrm{base}} \land \text{Cond}(sym\_ts)\);以及条件为假的路径,其路径条件为 \(PC_{\mathrm{false}} = PC_{\mathrm{base}} \land \neg \text{Cond}(sym\_ts)\)。这是精确分析不同时间戳下行为差异的基础。

% 3.  \textbf{监测关键行为差异:} 在分别对 \(PC_{\mathrm{true}}\) 和 \(PC_{\mathrm{false}}\) 两条路径进行后续符号执行时,引擎的关键任务是监测并比较预定义的“关键行为”是否出现差异。“关键行为”通常包括执行特定的状态更新操作(如 \texttt{SSTORE} 写入关键变量)、进行重要的外部交互(如 \texttt{CALL} 发送资金或调用特定合约)、或函数返回不同的值等。监测行为差异比仅比较最终状态更能定位有实际影响的依赖。

% 4.  \textbf{路径可行性验证:} 若在两条路径下观察到关键行为差异(例如,一条路径执行了某关键 \texttt{SSTORE} 而另一条没有),系统将进入验证阶段。利用 SMT 求解器分别检查 \(PC_{\mathrm{true}}\) 和 \(PC_{\mathrm{false}}\) 的可满足性。只有当两条路径均被证明是可达的(即求解器对两个路径条件都返回 SAT),才能确认观察到的行为差异并非发生在不可达路径上。

% 5.  \textbf{矿工可操纵性检查:} 在确认两条路径均可行后,系统从 SMT 求解器获取满足条件的模型,并从中提取出对应的时间戳值 \(t_1\)(满足 \(\text{Cond}\))和 \(t_2\)(不满足 \(\text{Cond}\))。最关键的一步是检查这两个时间戳的差值绝对值 \(|t_1 - t_2|\) 是否落在一个被认为是矿工可实际操纵的时间窗口内(例如,小于 900 秒)。此检查旨在过滤掉理论上存在但实际上无法被矿工利用的时间戳依赖,显著提高报告的准确性并减少误报。

% 6.  \textbf{生成完整报告:} 如果上述所有条件(依赖时间戳的分支、关键行为差异、两条路径均可行、时间差在可操纵范围内)均得到满足,系统则最终判定存在时间戳依赖漏洞。此时,将生成详细的报告,包含漏洞发生的具体位置(字节码偏移量)、受影响的关键行为描述、触发漏洞的时间戳条件范围以及相关的执行路径信息,为用户理解和修复漏洞提供充分依据。

% \subsubsection{拒绝服务攻击漏洞检测逻辑}
% 本系统对 Gas 耗尽型拒绝服务(DoS)漏洞的检测逻辑,旨在识别因循环执行次数受外部输入过度影响而可能导致交易无法完成的情况。具体流程遵循以下步骤:

% 1.  \textbf{识别核心机制:} 检测逻辑的核心在于识别可能导致 Gas 耗尽的循环结构。系统首先通过分析控制流图(CFG)中的回边或在符号执行过程中检测到对同一基本块的重复访问来定位循环。其基本假设是,无界或受外部输入显著影响的循环是 Gas 耗尽型 DoS 的主要来源。

% 2.  \textbf{符号执行探索与建模:} 一旦识别出循环,引擎引入一个符号变量 \(N\) 来表示循环的迭代次数。关键步骤是将此符号迭代次数 \(N\) 与控制循环执行的状态变量(例如,存储中的数组长度 \(L\) 或映射大小)关联起来,通过建立约束(如 \(N = L\))来反映这种依赖关系。同时,引擎符号化地执行循环体一次,以估算或精确计算单次迭代所需的 Gas 消耗 \(G_{\mathrm{iter}}\),\(G_{\mathrm{iter}}\) 本身可能是一个依赖于循环内部状态的符号表达式。

% 3.  \textbf{监测关键状态与 Gas 消耗:} 在此阶段,系统关注的是符号迭代次数 \(N\) 与可能受外部控制的状态变量 \(L\) 之间的关联,以及单次迭代的 Gas 消耗 \(G_{\mathrm{iter}}\)。这些是判断循环是否可能被恶意利用以耗尽 Gas 的关键因素。

% 4.  \textbf{构建 Gas 耗尽约束:} 系统构建核心的 DoS 约束,旨在判断是否存在一种情况使得总 Gas 消耗超过区块 Gas 上限 (\texttt{BlockGasLimit})。该约束通常表示为 \(N \times G_{\mathrm{iter}} > \texttt{BlockGasLimit}\)。考虑到精确计算 \(N \times G_{\mathrm{iter}}\) 的复杂性,系统也支持使用简化的启发式约束,例如 \(N > \text{Threshold}\),其中 \(\text{Threshold}\) 是一个预估的、足以引发 Gas 超限的迭代次数阈值。

% 5.  \textbf{路径可行性与 DoS 条件验证:} 引擎将构建的 DoS 约束(精确或启发式)与到达循环入口的路径条件 \(PC\) 相结合。然后,利用 SMT 求解器查询组合条件 \(PC \land (\text{DoS Constraint})\) 的可满足性。只有当求解器返回“可满足”(SAT)时,才证明存在一个可达的执行路径和相应的输入/状态,能够触发足够多的循环迭代以耗尽 Gas。

% 6.  \textbf{生成完整报告:} 如果上述组合条件被证明可满足,系统则判定存在潜在的 Gas 耗尽型 DoS 漏洞。此时,将生成详细报告,包含导致风险的循环结构信息(如入口和出口)、控制循环迭代次数的关键状态变量、触发该场景的路径条件以及相关的执行路径信息,为用户定位和修复问题提供依据。

\section{系统实现细节}

本节的核心目标在于将前述系统设计转化为一个可操作的原型系统——ByteChecker。前一节中已对系统的宏观架构、各模块的功能定位以及核心的混合分析范式进行了详尽阐述,涵盖了静态分析引导符号执行的策略、符号状态的表征方法以及 EVM 指令的模拟机制等关键设计原则。然而,从设计蓝图过渡到实际系统,要求将这些抽象概念具体化为代码实现。此过程涉及一系列关键的技术抉择与工程实践,尤其需关注编程语言与库的选择、关键数据结构的设计范式以及核心算法的实现细节。

在编程语言及库的选择方面,鉴于 Python 语言具备广泛的第三方库生态系统(特别是针对符号执行与约束求解的库,如 Z3)及其在快速原型开发方面的固有优势,本系统主要采用 Python 进行开发。高效地运用 Z3 求解器的 API 进行约束的构建与求解,并选用适宜的库执行字节码反汇编及控制流图(CFG)的构建,是保障系统功能正确性与运行性能的基础。

在数据结构层面,如何高效且精确地表征 EVM 的符号化状态(包括栈、内存、存储)、控制流图,以及分析过程中产生的中间结果与最终漏洞报告,构成了实现过程中的核心挑战。这要求精心设计 Python 类以封装相关概念,并有效利用 Z3 提供的位向量、数组等表达式类型来建模符号值及复杂状态。

更为关键的是核心算法的精确实现。这包括将字节码转换为 CFG 的算法、模拟各条 EVM 指令符号化执行逻辑的转换器、实现启发式引导的深度优先搜索(DFS)路径探索策略,以及嵌入针对四种特定漏洞的精确检测逻辑。这些算法的实现细节直接决定了 ByteChecker 的分析效能与效率。

接下来的小节将对这些实现层面的具体内容进行详细阐述,包括所选用的开发环境与工具库、关键数据结构的具体定义,以及部分核心算法的实现思路与要点,旨在完整呈现 ByteChecker 从设计构思到最终实现的完整过程。

\subsection{开发环境}

ByteChecker 系统的开发与测试环境主要基于以下组件和库,具体配置如表 \ref{tab:dev_environment} 所示:

\begin{table}[H] % 使用 [H] 可能需要 float 包
    \centering
    \caption{开发环境配置}
    \label{tab:dev_environment}
    \begin{tabular}{ll}
        \toprule
        \textbf{组件} & \textbf{版本/工具} \\
        \midrule
        操作系统 & Ubuntu 20.04 LTS \\
        编程语言 & Python 3.8 \\
        约束求解器 & z3-solver 4.14.1.0 \\
        字节码反汇编 & evm-dis 0.1.1 \\
        合约编译管理 & solc-select 1.0.4 \\
        包管理 & pip \\
        版本控制 & Git \\
        \bottomrule
    \end{tabular}
\end{table}



\subsection{关键数据结构}

为了有效地表示和操作智能合约的字节码、存储执行状态以及分析结果,ByteChecker 系统设计并使用了以下关键数据结构,如表 \ref{tab:data_structures} 所示:

% 请确保在 LaTeX 文档导言区加载了 \usepackage{tabularx}, \usepackage{booktabs}, \usepackage{ragged2e}
\begin{table}[H] % 使用 [H] 可能需要 float 包
    \centering
    \caption{关键数据结构}
    \small % 可选,如果表格太宽,缩小字体
    \begin{tabularx}{\linewidth}{@{} >{\RaggedRight\arraybackslash}p{0.28\linewidth} >{\RaggedRight\arraybackslash}p{0.27\linewidth} >{\RaggedRight\arraybackslash}X @{}} % 修改点: l -> p{0.25\linewidth}
        \toprule
        \textbf{数据结构} & \textbf{描述} & \textbf{主要属性/成员} \\
        \midrule
        \texttt{Instruction} & 表示单条 EVM 指令 & \texttt{opcode}, \texttt{operand}, \texttt{pc} \\
        \texttt{BasicBlock} & 表示基本块 & \texttt{start\_pc}, \texttt{end\_pc}, \texttt{instructions} (列表), \texttt{jump\_type} (跳转类型), \texttt{successors} (后继块列表) \\
        \texttt{ControlFlowGraph} & 表示合约的控制流图 & 节点 (\texttt{BasicBlock} 对象), 边 (表示控制流转移) \\
        \texttt{SymbolicState} & 表示符号执行过程中的一个具体状态 & \texttt{pc} (程序计数器), \texttt{stack} (符号栈), \texttt{memory} (符号内存模型), \texttt{storage} (符号存储模型), \texttt{path\_condition} (路径约束), \texttt{gas} (剩余Gas), \texttt{call\_context} (调用上下文), \texttt{block\_info} (区块信息) \\
        \texttt{Z3 Expressions} & Z3 求解器使用的符号表达式对象 & \texttt{BitVecRef} (位向量), \texttt{BoolRef} (布尔表达式), \texttt{ArrayRef} (数组,用于内存/存储) \\
        \texttt{VulnCandidate} & 静态分析识别出的潜在漏洞候选点 & \texttt{vuln\_type} (漏洞类型), \texttt{pc} (代码位置) \\
        \texttt{DetectedVulnerability} & 动态符号执行确认的漏洞实例 & \texttt{vuln\_type}, \texttt{pc}, \texttt{path\_condition} (触发路径), \texttt{model} (可选,触发输入的模型)  \\
        \texttt{Solver} & 对 Z3 求解器接口的封装类 & 提供 \texttt{is\_sat()}, \texttt{get\_model()}, \texttt{push()}, \texttt{pop()} 方法 \\
        \bottomrule
    \end{tabularx}
    \label{tab:data_structures}
\end{table}

这些数据结构的设计旨在清晰地映射 EVM 的概念和符号执行的需求,同时利用现有库的来简化实现。


\subsection{核心算法实现}

% 为了实现 ByteChecker 系统的核心分析能力,需要一系列精心设计的算法协同工作。这些算法涵盖了从字节码的结构化表示到符号执行路径探索,再到具体的漏洞检测和验证的全过程。本节将详细阐述这些关键算法的实现逻辑。

\subsubsection{控制流图构建算法}

控制流图 (CFG) 是理解合约执行逻辑、进行静态分析和指导动态符号执行的基础。其构建过程分为两个主要步骤:基本块划分和控制流关系分析。

\textbf{1. 基本块划分:}
此算法的目标是将输入的反汇编指令序列切分成一系列基本块,每个基本块代表一段无内部跳转的连续指令。算法通过识别特定的指令类型来确定块的边界。如算法 \ref{alg:basic_block_partition_compact} 所示。

% --- 1. 基本块划分 ---
\begin{algorithm}[htbp]
    \caption{基本块划分}
    \label{alg:basic_block_partition_compact}
    \KwIn{反汇编后的操作码文档 $P$}
    \KwOut{从 $P$ 中提取的基本块列表 $BB$}

    $BB \leftarrow []$; $current\_block \leftarrow null$; $is\_new\_block \leftarrow True$; $last\_ins\_address \leftarrow null$; $current\_block\_start\_pc \leftarrow 0$\;

    \While{未到达 $P$ 的末尾}{
        读取 $current\_opcode$ 从 $P$, 提取 $tok\_string$, $current\_ins\_address$\;

        \If{$current\_block$ is null}{ % 处理第一个块
             $B \leftarrow \text{NewBasicBlock}(current\_block\_start\_pc)$; $BB.append(B)$; $current\_block \leftarrow B$; $is\_new\_block \leftarrow False$\;
        }

        \If{$is\_new\_block$}{
             \If{$last\_ins\_address$ is not null and $current\_block.end\_pc$ is null}{ % 标记上一个块结束
                 $current\_block.end\_pc \leftarrow last\_ins\_address$; $current\_block.jump\_type \leftarrow \text{"falls\_to"}$\;
             }
             $B \leftarrow \text{NewBasicBlock}(current\_ins\_address)$; $BB.append(B)$; $current\_block \leftarrow B$; $is\_new\_block \leftarrow False$\; % 创建新块
        }

        $current\_block.add\_instruction(current\_opcode)$; $current\_block.end\_pc \leftarrow current\_ins\_address$\; % 添加指令并更新 PC

        \uIf{$tok\_string == \texttt{"JUMPDEST"}$}{
            \If{$last\_ins\_address$ is not null and $current\_block.start\_pc \ne current\_ins\_address$}{ % JUMPDEST 不是块首
                $current\_block.end\_pc \leftarrow last\_ins\_address$; $current\_block.jump\_type \leftarrow \text{"falls\_to"}$\; % 结束前块
                $B \leftarrow \text{NewBasicBlock}(current\_ins\_address)$; $BB.append(B)$; $current\_block \leftarrow B$\; % JUMPDEST 开始新块
                $current\_block.add\_instruction(current\_opcode)$; $current\_block.end\_pc \leftarrow current\_ins\_address$\;
            }
            $is\_new\_block \leftarrow False$\; % JUMPDEST 自身不结束块
        }
        \uElseIf{$tok\_string == \texttt{"JUMP"}$}{ $current\_block.jump\_type \leftarrow \text{"unconditional"}$; $is\_new\_block \leftarrow True$\; }
        \uElseIf{$tok\_string == \texttt{"JUMPI"}$}{ $current\_block.jump\_type \leftarrow \text{"conditional"}$; $is\_new\_block \leftarrow True$\; }
        \uElseIf{$tok\_string \in \{\texttt{"STOP"}, \texttt{"RETURN"}, \texttt{"REVERT"}, \texttt{"INVALID"}, \texttt{"SELFDESTRUCT"}\}$}{ $current\_block.jump\_type \leftarrow \text{"terminal"}$; $is\_new\_block \leftarrow True$\; }

        $last\_ins\_address \leftarrow current\_ins\_address$\;
    }
    \If{$current\_block$ is not null and $current\_block.jump\_type$ is null}{ $current\_block.jump\_type \leftarrow \text{"terminal"}$\; } % 处理最后块
    \Return{$BB$}
\end{algorithm}
% ...existing code...

% --- 1. 基本块划分 ---
% \begin{algorithm}htbp] % 允许浮动
%     % \footnotesize % 可选
%     \caption{基本块划分}
%     \label{alg:basic_block_partition} % 添加 label
%     \KwIn{反汇编后的操作码文档 $P$}
%     \KwOut{从 $P$ 中提取的基本块列表 $BB$}

%     初始化 $BB \leftarrow []$, $current\_block \leftarrow null$, $is\_new\_block \leftarrow True$\; % 合并初始化
%     $last\_ins\_address \leftarrow null$, $current\_block\_start\_pc \leftarrow 0$\; % 合并初始化

%     \While{未到达 $P$ 的末尾}{
%         读取 $current\_opcode$ 从 $P$\; 提取 $tok\_string$, $current\_ins\_address$\; % 合并读取和提取

%         \If{$current\_block$ is null}{ % 处理第一个块
%              $B \leftarrow \text{NewBasicBlock}(current\_block\_start\_pc)$\; 
%              $BB.append(B)$, $current\_block \leftarrow B$\; 
%              $is\_new\_block \leftarrow False$\; % 合并第3、4行
%         }

%         \If{$is\_new\_block$}{
%              % 标记上一个基本块的结束
%              \If{$last\_ins\_address$ is not null and $current\_block.end\_pc$ is null}{
%                  $current\_block.end\_pc \leftarrow last\_ins\_address$\; $current\_block.jump\_type \leftarrow \text{"falls\_to"}$\; % 合并块结束标记
%              }
%              % 创建新块
%              $B \leftarrow \text{NewBasicBlock}(current\_ins\_address)$\; $BB.append(B)$,$current\_block \leftarrow B$\; 
%              $is\_new\_block \leftarrow False$\; % 合并第3、4行
%         }

%         将 $current\_opcode$ 添加到 $current\_block$ 的指令列表\; $current\_block.end\_pc \leftarrow current\_ins\_address$\; % 合并指令添加和 PC 更新

%         \uIf{$tok\_string == \texttt{"JUMPDEST"}$}{ % 使用 \uIf 减少首行缩进
%             \If{$last\_ins\_address$ is not null and $current\_block.start\_pc \ne current\_ins\_address$}{ % JUMPDEST 不是块首
%                 % 结束 JUMPDEST 之前的块
%                 $current\_block.end\_pc \leftarrow last\_ins\_address$\; $current\_block.jump\_type \leftarrow \text{"falls\_to"}$\;
%                 % JUMPDEST 开始新块
%                 $B \leftarrow \text{NewBasicBlock}(current\_ins\_address)$\; 
%                 $BB.append(B)$,  $current\_block \leftarrow B$\; 
%                 $current\_block.add\_instruction(current\_opcode)$\; % 合并
%                 $current\_block.end\_pc \leftarrow current\_ins\_address$\; % 保持独立或与上一行合并
%             }
%             $is\_new\_block \leftarrow False$\; % JUMPDEST 自身不结束块
%         }
%         \uElseIf{$tok\_string == \texttt{"JUMP"}$}{ $current\_block.jump\_type \leftarrow \text{"unconditional"}$\; $is\_new\_block \leftarrow True$\; } % 合并
%         \uElseIf{$tok\_string == \texttt{"JUMPI"}$}{ $current\_block.jump\_type \leftarrow \text{"conditional"}$\; $is\_new\_block \leftarrow True$\; } % 合并
%         \uElseIf{$tok\_string \in \{\texttt{"STOP"}, \texttt{"RETURN"}, \texttt{"REVERT"}, \texttt{"INVALID"}, \texttt{"SELFDESTRUCT"}\}$}{ $current\_block.jump\_type \leftarrow \text{"terminal"}$\; $is\_new\_block \leftarrow True$\; } % 合并
%         % 其他指令不改变块结构

%         $last\_ins\_address \leftarrow current\_ins\_address$\;
%     }
%     % 处理最后一个块的结束类型
%     \If{$current\_block$ is not null and $current\_block.jump\_type$ is null}{ $current\_block.jump\_type \leftarrow \text{"terminal"}$\; } % 合并 If 条件和赋值
%     \Return{$BB$}
% \end{algorithm}


\textbf{2. 控制流关系分析:}
在基本块划分完成后,此算法负责分析块之间的跳转关系,构建 CFG 的边。如算法 \ref{alg:cfg_analysis_compact} 所示,它遍历所有基本块。对于非终止块,根据其 jump\_type(条件跳转、无条件跳转或顺序执行)来确定其后继基本块。对于 JUMPI,会产生两条边:一条指向紧随其后的指令(fall-through),另一条指向跳转目标地址。对于 JUMP,只产生一条指向跳转目标的边。对于顺序执行的块,产生一条指向下一块的边。跳转目标的地址解析可能需要静态分析(如常量传播)或标记为动态跳转,留待符号执行时解析。

\begin{algorithm}[htbp]
    \caption{控制流关系分析}
    \label{alg:cfg_analysis_compact}
    \KwIn{基本块列表 $BB$}
    \KwOut{基本块之间的边列表 $Edges$}

    Initialize $Edges$, $BlockMap$\;
    \ForEach{block in $BB$}{ $BlockMap[\texttt{block.start\_pc}] = \texttt{block}$\; }

    \For{\texttt{block} in $BB$}{
        \If{\texttt{block.jump\_type} == "terminal"}{ continue\; }

        \texttt{exit\_instr} = \texttt{block} 的最后一条指令; \texttt{fall\_pc} = \texttt{exit\_instr.pc + 1}\;

        \uIf{\texttt{block.jump\_type} == "conditional"}{ % JUMPI
            \If{\texttt{fall\_pc} in $BlockMap$}{ % 添加直通边
                 $B_{fall} = BlockMap[\texttt{fall\_pc}]$; $Edges.append((\texttt{block}, B_{fall}, \text{"fall"}))$; $\texttt{block.successors.append(B\_{fall})}$\;
            }
            \texttt{jump\_pc} = \texttt{resolve\_static\_jump\_target(block, exit\_instr)}$\;$
            \uIf{\texttt{jump\_pc} is not null and \texttt{jump\_pc} in $BlockMap$}{ % 添加条件跳转边 (静态)
                 $B_{jump} = BlockMap[\texttt{jump\_pc}]$; $Edges.append((\texttt{block}, B_{jump}, \text{"cond"}))$; $\texttt{block.successors.append(B\_{jump})}$\;
            } \Else { % 标记动态条件跳转
                 $Edges.append((\texttt{block}, \text{DYN}, \text{"cond"}))$; $\texttt{block.successors.append(DYN)}$\;
            }
        }
        \uElseIf{\texttt{block.jump\_type} == "unconditional"}{ % JUMP
            \texttt{jump\_pc} = \texttt{resolve\_static\_jump\_target(block, exit\_instr)}$\;$
            \uIf{\texttt{jump\_pc} is not null and \texttt{jump\_pc} in $BlockMap$}{ % 添加无条件跳转边 (静态)
                 $B_{jump} = BlockMap[\texttt{jump\_pc}]$; $Edges.append((\texttt{block}, B_{jump}, \text{"uncond"}))$; $\texttt{block.successors.append(B\_{jump})}$\;
            } \Else { % 标记动态无条件跳转
                 $Edges.append((\texttt{block}, \text{DYN}, \text{"uncond"}))$; $\texttt{block.successors.append(DYN)}$\;
            }
        }
        \uElseIf{\texttt{block.jump\_type} == "falls\_to"}{ % 顺序执行
             \If{\texttt{fall\_pc} in $BlockMap$}{ % 添加直通边
                 $B_{fall} = BlockMap[\texttt{fall\_pc}]$; $Edges.append((\texttt{block}, B_{fall}, \text{"fall"}))$; $\texttt{block.successors.append(B\_{fall})}$\;
            }
        }
    }
    \Return{$Edges$}
\end{algorithm}


% \begin{algorithm}
%     \caption{控制流关系分析}
%     \KwIn{基本块列表 $BB$}
%     \KwOut{基本块之间的边列表 $Edges$}

%     Initialize $Edges$\;
%     foreach block in $BB$ do{
%         $BlockMap[\texttt{block.start\_pc}] = \texttt{block}$\;
%     }


%     \For{\texttt{block} in $BB$}{
%         \If{\texttt{block.jump\_type} == "terminal"}{
%             continue\;
%         }

%         \texttt{exit\_instruction} = \texttt{block} 的最后一条指令,
%         \texttt{fall\_through\_pc} = \texttt{exit\_instruction.pc + 1}\; % 直通目标 PC

%         \If{\texttt{block.jump\_type} == "conditional"}{ % JUMPI
%             % 添加直通边
%             \If{\texttt{fall\_through\_pc} in $BlockMap$}{
%                  $B_{fall} = BlockMap[\texttt{fall\_through\_pc}]$\;
%                  $Edges.append((\texttt{block}, B_{fall}, \text{"fall-through"}))$\;
%                  \texttt{block.successors.append(B\_{fall})}\;
%             }
%             % 尝试静态解析跳转目标 (简化:假设目标是 PUSH 的常量)
%             \texttt{jump\_target\_pc} = \texttt{resolve\_static\_jump\_target(block, exit\_instruction)}\;
%             \If{\texttt{jump\_target\_pc} is not null and \texttt{jump\_target\_pc} in $BlockMap$}{
%                  $B_{jump} = BlockMap[\texttt{jump\_target\_pc}]$\;
%                  $Edges.append((\texttt{block}, B_{jump}, \text{"conditional"}))$\;
%                  \texttt{block.successors.append(B\_{jump})}\;
%             } \Else {
%                  % 标记动态跳转
%                  $Edges.append((\texttt{block}, \text{DYNAMIC\_TARGET}, \text{"conditional"}))$\;
%                  \texttt{block.successors.append(DYNAMIC\_TARGET)}\;
%             }
%         }
%         \ElseIf{\texttt{block.jump\_type} == "unconditional"}{ % JUMP
%             % 尝试静态解析跳转目标
%             \texttt{jump\_target\_pc} = \texttt{resolve\_static\_jump\_target(block, exit\_instruction)}\;
%             \If{\texttt{jump\_target\_pc} is not null and \texttt{jump\_target\_pc} in $BlockMap$}{
%                  $B_{jump} = BlockMap[\texttt{jump\_target\_pc}]$\;
%                  $Edges.append((\texttt{block}, B_{jump}, \text{"unconditional"}))$\;
%                  \texttt{block.successors.append(B\_{jump})}\;
%             } \Else {
%                  % 标记动态跳转
%                  $Edges.append((\texttt{block}, \text{DYNAMIC\_TARGET}, \text{"unconditional"}))$\;
%                  \texttt{block.successors.append(DYNAMIC\_TARGET)}\;
%             }
%         }
%         \ElseIf{\texttt{block.jump\_type} == "falls\_to"}{ % 块末尾不是跳转或终止指令
%              \If{\texttt{fall\_through\_pc} in $BlockMap$}{
%                  $B_{fall} = BlockMap[\texttt{fall\_through\_pc}]$\;
%                  $Edges.append((\texttt{block}, B_{fall}, \text{"fall-through"}))$\;
%                  \texttt{block.successors.append(B\_{fall})}\;
%             }
%         }
%     }
%     \Return{$Edges$}
%     \label{alg:cfg_analysis}
% \end{algorithm}


\subsubsection{基于 DFS 的符号执行路径探索}

此算法是动态分析的核心驱动力,它利用构建好的 CFG 和指令符号化模拟逻辑,系统地探索合约的执行路径。如算法 \ref{alg:dfs_symbolic_execution_part1} 和 \ref{alg:dfs_symbolic_execution_part2} 所示,该过程采用深度优先搜索 (DFS) 策略,通过一个工作列表(Worklist,作为栈使用)来管理待探索的状态。算法 \ref{alg:dfs_symbolic_execution_part1} 描述了主循环,它从 Worklist 中弹出状态,执行基本块内的指令符号化模拟,更新当前路径的符号化栈、内存、存储状态以及累积的路径条件,并进行资源限制(如循环次数、Gas、深度)检查以防止无限执行。在块内执行过程中,它会调用 UpdateAnalysisResults 来实时应用漏洞检测逻辑。当遇到 JUMPI 时,它处理分支,将其中一个分支状态压回 Worklist,继续执行另一个。块执行完毕后,调用算法 \ref{alg:dfs_symbolic_execution_part2} 中定义的 HandleBlockExit 过程。该过程负责处理块的退出逻辑,根据跳转类型解析后继块,检查路径可行性,并将可行的后继状态压入 Worklist,从而驱动整个 DFS 探索过程。


% --- DFS 符号执行 Part 1: 路径探索与块内执行 ---
\begin{algorithm}[htbp] % 允许浮动
    \caption{基于 DFS 的符号执行 (Part 1: 路径探索与块内执行)}
    \label{alg:dfs_symbolic_execution_part1}
    \KwIn{初始符号状态 $params_{init}$, 起始基本块 $block_{start}$}
    \KwOut{发现的漏洞列表 $Vulnerabilities$}
    初始化 $Vulnerabilities$, $visited\_edges$, $Worklist$\;
    $Worklist.push((params_{init}, block_{start}, null, 0))$\; % (状态, 当前块, 前驱块, 深度)

    \While{$Worklist$ is not empty}{
        $(params, block, prev\_block, depth) \leftarrow Worklist.pop()$\; % LIFO 实现 DFS

        \If{$block$ is invalid or $depth > DEPTH\_LIMIT$ or $gas\_used(params) > GAS\_LIMIT$}{ continue }

        $current\_edge \leftarrow (prev\_block, block)$\;
        \If{$prev\_block$ is not null and $visited\_edges.get(current\_edge, 0) > LOOP\_LIMIT$}{ continue } % 循环限制
        % \If{$gas\_used(params) > GAS\_LIMIT$}{ continue } % Gas 限制

        \If{$prev\_block$ is not null}{ $visited\_edges[current\_edge] = visited\_edges.get(current\_edge, 0) + 1$\; }

        $current\_state \leftarrow params.copy()$\; % 复制状态以进行块内执行
        $path\_terminated\_in\_block \leftarrow False$\;
        \For{每条指令 $instr$ in $get\_instructions(block)$}{
            $next\_states \leftarrow \text{SymbolicExecute}(instr, current\_state)$\; % 调用指令模拟
            \If{length($next\_states$) == 0}{ % 指令执行终止了当前路径
                 $path\_terminated\_in\_block \leftarrow True$\; break\;
            } \ElseIf{length($next\_states$) > 1}{ % JUMPI 产生分支
                 $Worklist.push(($next\_states[1]$, block, prev\_block, depth))$\; % false 分支压栈
                 $current\_state \leftarrow next\_states[0]$\; % 继续 true 分支
            } \Else { % 单一后继状态
                 $current\_state \leftarrow next\_states[0]$\;
            }
            UpdateAnalysisResults(current\_state, instr, Vulnerabilities)\; % 调用漏洞检查
            current\_state.gas -= calculate\_gas\_cost(instr, current\_state)\; % 更新 Gas
            \If{$current\_state.gas < 0$}{ % 检查块内 Gas 耗尽
                 $path\_terminated\_in\_block \leftarrow True$\; break\;
            }
        }

        \If{$path\_terminated\_in\_block$}{ % 路径在块内终止
            \If{$prev\_block$ is not null}{ $visited\_edges[current\_edge] -= 1$\; } % 回溯边计数
            continue\;
        }

        % 块执行完毕,调用 Part 2 处理块退出和跳转
        HandleBlockExit(current\_state, block, prev\_block, depth, visited\_edges, Worklist, Vulnerabilities)\;
    }
    \Return{$Vulnerabilities$}
\end{algorithm}

% --- DFS 符号执行 Part 2: 块间跳转处理 ---
\begin{algorithm}[htbp] % 允许浮动
    \caption{基于 DFS 的符号执行 (Part 2: 块间跳转处理)}
    \label{alg:dfs_symbolic_execution_part2}
    \SetKwProg{Proc}{Procedure}{:}{}
    \Proc{HandleBlockExit(params, block, prev\_block, depth, visited\_edges, Worklist, Vulnerabilities)}{
        $current\_edge \leftarrow (prev\_block, block)$\;
        $depth \leftarrow depth + 1$\; % 增加深度

        \If{$block.jump\_type$ == "terminal"}{
            GenerateTestCase(params, Vulnerabilities)\; % 生成测试用例
            \If{$prev\_block$ is not null}{ $visited\_edges[current\_edge] -= 1$\; } % 回溯边计数
            \Return % 终止路径,从此过程返回
        }

        % 获取后继块 (需要根据 params 解析动态跳转)
        $successors \leftarrow resolve\_successors(block, params)$\;

        \For{每个后继块 $succ$ in $successors$}{
             % 应用启发式策略决定是否/何时探索 succ (简化:直接加入)
             $new\_params \leftarrow params.clone()$\; % 写时复制优化
             $Worklist.push((new\_params, succ, block, depth))$\; % 加入栈顶,实现 DFS
        }

        % 正常路径(非终止)的回溯边计数
        \If{$prev\_block$ is not null}{ $visited\_edges[current\_edge] -= 1$\; }
    } % EndProc
\end{algorithm}


\subsubsection{静态分析算法}

% 为了提高效率,在进行耗时的动态符号执行之前,可以先进行轻量级的静态分析来识别潜在的高风险区域。

\textbf{1. 字节码模式匹配:}
此算法扫描指令序列,查找与已知漏洞相关的特定指令模式或 CFG 结构。如算法 \ref{alg:pattern_matching} 所示。

% --- 5. 字节码模式匹配 ---
\begin{algorithm}[H]
    \caption{字节码模式匹配 (静态分析)}
    \label{alg:pattern_matching}
    \KwIn{指令列表 $Instructions$, CFG}
    \KwOut{可疑代码位置列表 $SuspiciousPCs$}

    初始化 $SuspiciousPCs$\;

    \For{每条指令 $instr$ in $Instructions$}{
        $pc \leftarrow instr.pc$\;
        % 重入模式: CALL 后紧跟状态修改? (简化示例)
        \If{$instr.opcode == \texttt{"CALL"}$ and instr.gas > 2300}{ % 考虑 Gas
             $next\_block \leftarrow \text{get\_fallthrough\_block}(pc, CFG)$\;
             \If{$next\_block$ is not null and contains\_state\_write($next\_block$)}{
                  $SuspiciousPCs.add((pc, \text{"Reentrancy Candidate"}))$\;
             }
        }
        % 整数溢出模式: 未保护的算术运算?
        \If{$instr.opcode \in \{\texttt{"ADD"}, \texttt{"SUB"}, \texttt{"MUL"}\}$}{
             \If{not is\_safemath\_protected($instr$, $Instructions$, CFG)}{ % 检查保护模式
                  $SuspiciousPCs.add((pc, \text{"Integer Overflow Candidate"}))$\;
             }
        }
        % 时间戳依赖模式: TIMESTAMP 结果影响 JUMPI?
        \If{$instr.opcode == \texttt{"TIMESTAMP"}$}{
             \If{result\_influences\_conditional\_jump($instr$, $Instructions$, CFG)}{ % 数据流分析
                  $SuspiciousPCs.add((pc, \text{"Timestamp Dependency Candidate"}))$\;
             }
        }
        % DoS 模式: 识别循环入口
        \If{is\_loop\_header($pc$, CFG)}{ % 基于 CFG 的循环检测
             $SuspiciousPCs.add((pc, \text{"DoS Loop Candidate"}))$\;
        }
    }
    \Return{$SuspiciousPCs$}
\end{algorithm}
\needspace{3\baselineskip}
\textbf{2. 污点分析:}
污点分析检查关键操作数是否受到外部不可信输入的“污点”影响。如算法 \ref{alg:taint_analysis} 所示,它首先定义污点源(如 CALLER, CALLDATALOAD)。然后使用 Worklist 算法在 CFG 上传播污点信息,直到达到不动点。最后,检查每个可疑点的关键操作数是否被标记为污点。只有同时满足模式匹配和污点条件的点才被认为是高风险的漏洞候选点,用于指导后续的动态分析。

% --- 6. 污点分析 ---
\begin{algorithm}[H]
    \caption{污点分析}
    \label{alg:taint_analysis}
    \KwIn{可疑代码位置列表 $SuspiciousPCs$, CFG, 指令列表 $Instructions$}
    \KwOut{漏洞候选点列表 $VulnCandidates$}

    初始化 $VulnCandidates$, $TaintedState$\; % TaintedState[pc] -> set of tainted stack/memory locations
    定义污点源指令 $SourceOpcodes = \{\texttt{"CALLER"}, \texttt{"CALLDATALOAD"}, \texttt{"CALLVALUE"}, ...\}$\;
    初始化 $Worklist$ (包含所有指令地址)\;

    \While{$Worklist$ is not empty}{ % 不动点计算
        $pc \leftarrow Worklist.pop()$\;
        $instr \leftarrow \text{get\_instruction}(pc, Instructions)$\;
        $old\_taint\_at\_pc \leftarrow TaintedState.get(pc, \emptyset)$\;
        $current\_taint \leftarrow \emptyset$\;

        % 聚合前驱节点的污点状态
        \For{前驱块 $prev\_block$ of $get\_block(pc, CFG)$}{
             $last\_instr\_pc \leftarrow prev\_block.end\_pc$\;
             $current\_taint \leftarrow current\_taint \cup TaintedState.get(last\_instr\_pc, \emptyset)$\;
        }

        % 根据当前指令传播污点
        $output\_taint \leftarrow \emptyset$\;
        \If{$instr.opcode \in SourceOpcodes$}{
             $output\_taint \leftarrow \{\text{taint\_source}(instr)\}$\; % 产生新污点
        } \Else {
             $output\_taint \leftarrow propagate\_taint(instr, current\_taint)$\; % 根据指令语义传播
        }

        % 更新污点状态并检查是否需要加入 Worklist
        \If{$output\_taint \not\subseteq old\_taint\_at\_pc$}{
             $TaintedState[pc] \leftarrow old\_taint\_at\_pc \cup output\_taint$\;
             \For{后继块 $succ\_block$ of $get\_block(pc, CFG)$}{
                  $Worklist.add(succ\_block.start\_pc)$\;
             }
        }
    }

    % 检查可疑点是否被污染
    \For{$(pc, type)$ in $SuspiciousPCs$}{
         $instr \leftarrow \text{get\_instruction}(pc, Instructions)$\;
         $input\_taint\_state \leftarrow \emptyset$\; % 获取指令执行前的污点状态
         \For{前驱块 $prev\_block$ of $get\_block(pc, CFG)$}{
             $input\_taint\_state \leftarrow input\_taint\_state \cup TaintedState.get(prev\_block.end\_pc, \emptyset)$\;
         }
         \If{is\_tainted(instr, type, input\_taint\_state)}{ % 检查关键操作数是否污点
              $VulnCandidates.add((pc, type, \text{taint\_details}))$\;
         }
    }
    \Return{$VulnCandidates$}
\end{algorithm}


% \subsubsection{漏洞检测逻辑}

% 为了在动态符号执行过程中实时识别特定的漏洞模式,系统为每种目标漏洞设计了专门的检测逻辑。这些逻辑以独立检查函数的形式嵌入到符号执行引擎(具体在 DFS 算法的 `UpdateAnalysisResults` 步骤中被调用)。当执行到特定指令或状态发生变化时,相应的检查函数会被触发,用于判断当前路径和状态是否满足该漏洞的触发条件。表 \ref{tab:vuln_detection_logic} 概述了这些关键的检测逻辑接口及其主要功能。


% \begin{table}[H] % 使用 [H] 可能需要 float 包
%     \centering
%     \caption{漏洞检测逻辑接口}
%     \label{tab:vuln_detection_logic}
%     \begin{tabularx}{\linewidth}{@{} l >{\RaggedRight\arraybackslash}X @{}}
%         \toprule
%         \textbf{检测逻辑接口} & \textbf{功能描述} \\
%         \midrule
%         \texttt{CheckReentrancy()} & 检查 \texttt{CALL} 指令是否存在重入风险,重点分析状态更新与外部调用之间的顺序关系,以及是否存在不安全的回调路径。 \\
%         \addlinespace
%         \texttt{CheckIntegerOverflow()} & 检查 \texttt{ADD}, \texttt{SUB}, \texttt{MUL} 等算术指令的结果是否可能超出 256 位整数的表示范围,导致上溢或下溢。 \\
%         \addlinespace
%         \texttt{CheckTimestampDependency()} & 检查 \texttt{JUMPI} 指令的跳转条件是否直接或间接受到 \texttt{TIMESTAMP} 指令返回值的影响,判断是否存在可被矿工操纵的执行路径分叉。 \\
%         \addlinespace
%         \texttt{CheckDoS()} & 检查循环结构(特别是迭代次数受外部输入影响的循环)或可能消耗大量 Gas 的外部调用是否存在导致交易因 Gas 耗尽而失败的风险。 \\
%         \bottomrule
%     \end{tabularx}
% \end{table}



% \subsubsection{测试用例生成}

% 当漏洞检测逻辑确认某条路径满足漏洞触发条件(即相关约束可满足)时,此算法负责从 SMT 求解器返回的模型中提取具体的输入值,生成可复现的测试用例。如算法 \ref{alg:test_case_generation} 所示,它获取满足条件的模型,然后从中提取与 `msg.sender`, `msg.value`, `calldata` 等输入相关的符号变量的具体值,最终组合成一个可执行的交易参数。

% % --- 8. 测试用例生成 ---
% \begin{algorithm}[H]
%     \caption{测试用例生成}
%     \label{alg:test_case_generation} % 添加 label
%     \KwIn{触发漏洞的路径条件 $PC_{vuln}$, 相关的符号输入变量 $SymbolicInputs$, 求解器 $solver$}
%     \KwOut{具体的测试用例数据 $TestCase$ 或 None}

%     \If{not $solver.is\_sat(PC_{vuln})$}{
%         \Return{None} % 路径不可满足
%     }

%     $model \leftarrow solver.get\_model(PC_{vuln})$\;
%     $TestCase \leftarrow \{\}$\;

%     \For{每个符号输入变量 $sym\_var$ in $SymbolicInputs$}{
%         % 尝试从模型中获取具体值,处理可能不存在的情况
%         $concrete\_value = model.eval(sym\_var, model\_completion=True)$\; % model_completion=True 处理未完全约束的变量
%         $input\_name \leftarrow \text{get\_input\_name}(sym\_var)$\; % e.g., "msg.sender", "calldata_param_1"
%         $TestCase[input\_name] \leftarrow concrete\_value$\;
%     }

%     % 可能需要根据 calldata 相关的变量值重构 ABI 编码的 calldata
%     $TestCase[\text{"calldata"}] \leftarrow \text{reconstruct\_calldata}(TestCase)$\;

%     \Return{$TestCase$}
% \end{algorithm}

% 通过以上算法的协同工作,ByteChecker 系统能够有效地在字节码层面进行静态预筛选和深入的动态符号执行分析,最终实现对目标漏洞的自动化检测和验证。


\section{系统工作流程}
ByteChecker 系统各模块间工作流程如图\ref{fig:系统工作流程}所示,具体步骤如下:

\begin{figure}
    % \centering
    \hspace*{4.5cm}
    \includegraphics[width=0.7\linewidth]{系统工作流程}
    \caption{系统工作流程}
    \label{fig:系统工作流程}
\end{figure}


\begin{enumerate}
    \item \textbf{输入接收与处理:} 用户通过命令行或接口提供智能合约的源代码文件、字节码字符串或链上地址。输入模块负责接收并验证输入。如果是源代码,调用 \texttt{solc} 编译器获取字节码和 ABI。
    \item \textbf{预处理:} 预处理模块对获取到的字节码进行反汇编,构建控制流图 (CFG),识别基本块和潜在的函数入口。
    \item \textbf{静态分析与候选点生成:} 静态分析模块(如果启用)快速扫描字节码和 CFG,利用第三章的漏洞特征模式和轻量级符号执行,识别出潜在的高风险代码位置,生成漏洞候选点列表。
    \item \textbf{动态符号执行分析:}
        a.  初始化符号状态,通常从合约的构造函数或指定的公共/外部函数入口点开始。
        b.  根据路径探索策略(DFS + 启发式,优先处理候选点)选择下一个要探索的状态/路径。
        c.  调用 EVM 指令符号化转换器,模拟指令执行,更新符号状态(栈、内存、存储、PC)和路径条件。
        d.  在执行过程中,实时应用漏洞检测逻辑,检查当前状态和路径条件是否满足某种漏洞的触发条件。
        e.  如果检测到满足漏洞条件:记录漏洞信息(类型、位置等),并调用约束求解器获取模型,生成对应的测试用例。
        f.  如果遇到分支点 (`JUMPI`):调用约束求解器检查两个分支路径的可行性。对于可行的分支,创建新的符号状态并加入待探索列表(或根据 DFS 策略立即递归探索)。
        g.  应用资源限制(循环、深度、时间)和状态剪枝策略。
        h.  重复步骤 b-g,直到没有更多路径可探索或达到全局时间/资源限制。
    \item \textbf{结果处理:} 动态符号执行结束后,结果处理模块收集所有记录的潜在漏洞信息,进行去重(例如,同一漏洞在不同路径上被多次发现)和过滤(例如,去除一些低置信度的发现)。
    \item \textbf{报告生成:} 报告生成模块将经过处理的漏洞列表格式化为用户友好的报告,包含必要的细节和证据。
    \item \textbf{输出:} 系统输出最终的检测报告文件。
\end{enumerate}

这个流程强调了从静态预筛选到动态深度分析的结合,旨在平衡分析的效率和深度。

\section{本章小结}


本章详细设计并阐述了 ByteChecker 系统,该系统利用字节码静态分析和符号执行技术来自动化检测智能合约中四种典型漏洞,重入、整数溢出、时间依赖、拒绝服务。本章首先明确了系统的设计原则,强调了准确性、效率优化,可扩展性。接着,提出了一个包含输入处理、预处理、核心分析引擎、结果处理与报告等模块的系统架构。

本章的核心在于对核心分析引擎的深入设计。为此详细阐述了预处理模块如何构建 CFG,符号执行引擎如何表示符号状态、符号化解释 EVM 指令以及与 Z3 求解器交互。尤为关键的是,系统设计了两个协同工作的分析模块:一个静态分析模块,利用轻量级符号执行和第三章的字节码特征快速识别漏洞候选点;以及一个动态分析模块,采用启发式引导的深度优先搜索策略进行深入的路径探索和验证。还进一步明确了在动态符号执行过程中嵌入的、针对四种目标漏洞的具体检测逻辑,这些逻辑直接源自第三章的分析成果。

此外,本章还讨论了系统实现的关键方面,包括开发环境、关键数据结构的设计以及核心算法,如 CFG 构建、指令符号化、路径探索优化、测试用例生成的实现要点。最后,通过流程图和文字描述了系统内各模块之间工作流程。

\chapter{实验与结果分析}


本章通过一系列严谨的实验对其性能进行全面评估。核心目标是定量分析 ByteChecker 在检测四种类型智能合约漏洞方面的准确性(RQ1)和效率(RQ2),并验证其关键创新点——静态分析引导机制对提升符号执行效率的实际效果(RQ3)。为此,本章将首先介绍统一的实验环境和评测设置,包括所采用的评估指标、精心构建的三个数据集,标准数据集、有源码真实世界数据集和无源码真实世界数据集,以及选定的对比工具。随后,将详细呈现三个针对性实验的设计方案、执行结果,并对结果进行深入分析与讨论,最终对 ByteChecker 的整体性能、优势与局限性进行总结。

\section{实验环境}

本文实验均在 Ubuntu 20.04 LTS 系统上进行,使用机器的硬件配置为 Intel Core i7-12700h 处理器,16GB 内存,512GB SSD 存储。

实验中使用到的主要工具和软件如下表所示:

\begin{table}[H]
    \centering
    \caption{实验环境配置}
    \label{tab:env_config}
    \begin{tabular}{|c|c|c|}
        \hline
        \textbf{组件} & \textbf{版本} & \textbf{功能描述}\\ \hline
        Python & 3.8 & 主程序开发\\ \hline
        solc-Select & 1.0.4 & 合约编译\\ \hline
        z3-solver & 4.14.1.0 & 约束求解\\ \hline
        evm-dis & 0.1.1 & 反汇编\\ \hline
    \end{tabular}
    \caption*{注:以上软件版本均为实验时使用的最新稳定版本。}
\end{table}


\section{实验目的}

本章实验的核心目的是全面评估本文设计和实现的 ByteChecker 系统的性能和效果。回答以下研究问题:

RQ1. ByteChecker 在检测重入、整数溢出、时间依赖、拒绝服务这四种漏洞方面的准确性如何?

RQ2. ByteChecker 分析智能合约漏洞方面效率如何?

RQ3. ByteChecker 的静态分析引导机制是否确实提升了动态符号执行的效率?

\section{实验设置}

为了科学地评估 ByteChecker 并回应上一节提出的问题,设计了如下实验。

\subsection{评估指标}

为了评估模型对于智能合约漏洞的检测效果,需要相关指标来进行评价。混淆矩阵\textsuperscript{\cite{sathyanarayanan2024confusion}}(Confusion Matrix)是用于评估分类模型性能的重要工具,它能够直观地展示模型的预测结果与真实标签之间的关系。以二分类问题为例,混淆矩阵的表示如表\ref{tab:confusion_matrix}所示:

\begin{table}[h]
\centering
\renewcommand{\arraystretch}{1.4}
\begin{tabular}{|c|c|c|}
\hline
\diagbox{实际类别}{预测类别} & 正例 & 反例 \\ \hline
正例 & TP & FN \\ \hline
反例 & FP & TN \\ \hline
\end{tabular}
\caption{混淆矩阵用于分类模型的性能评估}
\label{tab:confusion_matrix}
\end{table}

其中 TP (True Positive)表示模型正确预测为正例的数量,FP(False Positive)表示模型错误预测为正例的数量,TN(True Negative)表示模型正确预测为反例的数量,FN(False Negative)表示模型错误预测为反例的数量。基于混淆矩阵,可以计算出多种评估指标来全面评估模型的性能。

在本实验中,将使用以下指标来评估 ByteChecker 的检测性能:
    
准确率 (Accuracy):
表示模型预测正确的样本数占总样本数的比例,其中 TP 是真正例,TN 是真反例,FP 是假正例,FN 是假反例:
\begin{equation}
Accuracy = \frac{TP + TN}{TP + TN + FP + FN} 
\end{equation}

精确率 (Precision)
衡量被预测为正例的样本中,实际为正例的样本的比例。如果精确率较高,意味着模型在预测正例时的准确性较高,即“假警报”少,计算公式如下:
\begin{equation}
Precision = \frac{TP}{TP + FP}
\end{equation}

召回率 (Recall)
衡量实际正例样本中被正确预测为正例的比例。如果召回率较高,意味着模型能够较好地识别出实际的正例样本,即“漏报”少。以下是召回率的计算公式:
\begin{equation}
Recall = \frac{TP}{TP + FN}
\end{equation}

F1 分数 (F1 Score)
精确率和召回率的调和平均数,目的是寻找一个平衡点,以同时考虑精确率和召回率的重要性。F1 分数越高,表明模型的性能越优秀:
\begin{equation}
F1\text{-}score = \frac{2 \times Precision \times Recall}{Precision + Recall}
\end{equation}

这些指标从不同角度评估模型的分类性能,有助于全面理解分类模型的表现。

\subsection{实验数据集}

为了全面评估并在不同场景下验证 ByteChecker,构建了一个包含三类合约的数据集:

1. 标准数据集:这个数据集由 112 个人工审阅的智能合约组成,包含了186 个漏洞点,合约来自以往智能合约安全研究者释出的数据集\textsuperscript{\cite{zheng2024dappscan}}\textsuperscript{\cite{zheng2023turn}},在此基础上筛选出包含本研究关注的四种漏洞类型的合约。由于这个数据集经过了反复的人工审阅和标注,因此可以认为其 Ground Truth 是可靠的。此数据集主要用于验证 Bytecheck 的功能性,即对合约漏洞检测的准确性。数据集中的漏洞类型包括重入、整数溢出、时间依赖和拒绝服务漏洞,经统计各种类型类型漏洞分布如表\ref{tab:vulnerability_distribution}:

\begin{table}[h]
    \centering
    \renewcommand{\arraystretch}{1.3}
    \begin{tabular}{|c|c|}
    \hline
    \textbf{漏洞类型} & \textbf{漏洞数量} \\ \hline
    重入漏洞 & 46 \\ \hline
    整数溢出漏洞 & 74 \\ \hline
    时间依赖漏洞 & 36 \\ \hline
    拒绝服务漏洞 & 30 \\ \hline
    \textbf{总计} & 186 \\ \hline
    \end{tabular}
    \caption{数据集中四种漏洞类型的分布}
    \label{tab:vulnerability_distribution}
\end{table}
    
2. 有源码的真实世界合约数据集:这个数据集来自 Thomas\textsuperscript{\cite{Durieux_2020}}在研究中使用的以太坊链上数据集,包含了共 47587 个智能合约源代码,本实验根据合约版本号和合约长度筛选了 5000 份作为实验的数据集。由于这份数据集并没有提供明显的标注,因此其 Ground Truth 来自于几种漏洞检测工具之间的交叉验证,以及人工审阅,此数据集主要用与横向对比 Oyente , Mythril 和 Slither\textsuperscript{\cite{Feist_2019}},评估 ByteChecker 的准确性和效率。

3. 无源码的真实世界合约数据集:这个数据集包含了 1000 个以太坊公链上的真实合约,使用 Etherscan\textsuperscript{\cite{etherscan}}提供的 API 接口从以太坊区块链上抓取的合约字节码,这个数据集的目的是为了评估 ByteChecker 在使用和不适用静态分析模块情况下检测的效率,因此,为了让实验更具有普遍性,抓取过程中,将字节码长度均匀分布在 1KB 到 24KB 之间的合约进行抓取,用字节码长度近似合约复杂度。
\subsection{实验设计}

为系统性地回答第 5.1 节提出的三个研究问题,本节设计了以下三个实验,分别侧重于准确性评估、效率与对比分析、以及静态分析引导机制的有效性验证。

\textbf{实验一:ByteChecker 准确性评估 (对应 RQ1)}

本实验目的是定量评估 ByteChecker 在检测四种目标漏洞方面的精确率、召回率和 F1 分数。使用数据集是数据集 1。该数据集包含 112 个合约和 186 个经过人工确认的漏洞点,提供了可靠的 Ground Truth,是评估准确性的理想选择。

实验流程为:在实验环境上运行 ByteChecker,输入为数据集 1 中的所有合约。为每个合约设置统一的分析超时阈值为 15 分钟,以确保实验的可重复性。收集 ByteChecker 对每个合约输出的漏洞报告,包含漏洞类型和位置信息。将 ByteChecker 的报告与数据集的 Ground Truth 进行逐一比对。根据比对结果,统计四种漏洞类型各自的真阳性 (TP)、假阳性 (FP) 和假阴性 (FN) 数量。依据 5.3.1 节定义的公式,计算每种漏洞类型以及总体的精确率、召回率和 F1 分数。

最后对结果进行分析,讨论 ByteChecker 在不同类型漏洞上的表现差异及其原因,并结合具体案例分析典型的 FP 和 FN 情况。

\textbf{实验二:ByteChecker 效率及对比分析 (对应 RQ2)}

本实验目的是评估 ByteChecker 在处理大规模真实世界智能合约时的分析效率,并将其与三种具有代表性的现有工具——Mythril(符号执行)、Slither(静态分析)和 Oyente(早期基于字节码的符号执行工具)进行横向比较。实验核心关注指标为分析时间与峰值内存消耗,以量化各工具的性能开销和可扩展性。

为模拟真实应用场景并评估工具处理大规模数据的能力,本实验选用数据集 2,该数据集包含大量从以太坊区块链收集的真实合约源代码。考虑到数据集规模庞大,实验将在整个数据集或一个具有代表性、覆盖不同复杂度合约的子集上进行。

实验流程设计如下:首先,在统一的硬件与软件环境下,为每个合约设置相同的分析超时阈值 15 分钟。随后,运行 ByteChecker 及三种对比工具,Mythril, Slither 和 Oyente,各工具使用其标准或默认配置,对其支持的全部漏洞类型进行检测。需要强调的是,由于各工具设计目标与技术实现不同,其检测的漏洞集合并不完全一致;本实验旨在比较各工具完成其各自扫描任务的整体效率。在分析过程中,将精确记录每个工具处理单个合约所需的执行时间与运行期间达到的峰值内存使用量。

实验数据的分析将聚焦于效率指标。计算并比较 ByteChecker 与各对比工具在处理数据集时的平均及中位数分析时间和峰值内存消耗,结果将以表格形式呈现。为进一步评估 ByteChecker 的可扩展性,本实验将绘制其分析时间随合约复杂度(以字节码大小为代理变量)变化的趋势图。分析讨论部分将结合各工具的核心技术(静态分析、符号执行、混合方法)对其效率表现差异进行解释。

最后,必须明确指出,鉴于数据集 2 缺乏精确的、逐一标注的 Ground Truth,本实验不进行严格的准确性评估,即不计算 Precision, Recall, F1-Score 等指标,相关准确性验证见实验一。本实验中可能报告的漏洞发现数量仅作为效率比较的辅助信息,不用于直接衡量检测准确度。

\textbf{实验三:静态分析引导机制有效性验证 (对应 RQ3)}

为验证本文提出的静态分析引导机制对提升动态符号执行效率的实际效果,本实验旨在量化评估该机制在减少分析时间和降低路径探索复杂度方面的贡献。实验的核心在于对比包含静态分析引导的完整版 ByteChecker,后文称之 ByteChecker-Full, 与禁用该引导、仅依赖底层符号执行引擎的版本,后文称之 ByteChecker-SE-Only,在处理相同任务时的性能差异。

本实验选用数据集 3,该数据集包含 1000 个从以太坊公链获取、字节码大小分布均匀(1KB-24KB)且无对应源代码的真实合约。选择此数据集的原因在于:其无源码特性符合 ByteChecker 的主要应用场景;合约复杂度的多样性有助于观察优化机制在不同难度下的表现;规模适中,便于进行全面的对比测试。

实验流程如下:首先,在统一的实验环境下, ByteChecker-Full 和 ByteChecker-SE-Only 设置成相同的分析超时阈值,都为  15 分钟。随后,分别使用这两个版本处理数据集 3 中的全部 1000 个合约。在执行过程中,精确记录每个版本分析每个合约所需的执行时间和峰值内存使用量。为更深入地评估效率提升,若工具支持,强烈建议同时记录两个版本在分析过程中探索的路径数量、访问的状态数量以及调用 SMT 求解器的次数。

数据分析阶段,本实验将重点比较 ByteChecker-Full 和 ByteChecker-SE-Only 在各项效率指标上的表现。具体而言,将计算两个版本在处理整个数据集时的平均及中位数时间、内存消耗、路径/状态数量,分析统计数据,评估差异的显著性。此外,可选择性地比较两个版本发现的漏洞总数,以初步判断效率提升是否以牺牲显著的检测能力为代价。


\section{实验结果与分析}

本节将展示 ByteChecker 在上述数据集上的实验结果,并进行深入分析实验数据。实验结果将分为三个部分:首先是准确性评估,展示 ByteChecker 在标准数据集上的检测性能;其次是效率评估与对比分析,比较 ByteChecker 与其他工具在真实世界合约上的性能;最后是静态分析引导机制的有效性验证,分析该机制对符号执行效率的提升效果,这三个部分将分别对应于三个研究问题。

\subsection{检测准确性分析}

本实验在标准数据集上评估 ByteChecker 检测四种目标漏洞的准确性。表\ref{tab:accuracy_results_ch5}汇总了主要的有效性指标。

\begin{table}
    \centering
    \caption{ByteChecker 在标准数据集上的检测准确性结果}
    \begin{tabular}{lccccc}
    \toprule
    指标 & 重入 & 整数溢出 & 时间依赖 & 拒绝服务 & \textbf{总体} \\ 
    \midrule
    TP (检测到的真实漏洞数) & \textbf{35} & \textbf{56} & \textbf{28} & \textbf{24} & \textbf{143} \\ 
    FP (误报数) & \textbf{15} & \textbf{22} & \textbf{10} & \textbf{9} & \textbf{56} \\ 
    FN (漏报数) & \textbf{11} & \textbf{18} & \textbf{8} & \textbf{6} & \textbf{43} \\ 
    \midrule
    Precision (\%) & \textbf{70.0\%} & \textbf{71.8\%} & \textbf{73.7\%} & \textbf{72.7\%} & \textbf{72.1\%} \\ 
    Recall  (\%) & \textbf{76.1\%} & \textbf{75.7\%} & \textbf{77.8\%} & \textbf{80.0\%} & \textbf{77.1\%} \\ 
    F1-Score (\%) & \textbf{72.9\%} & \textbf{73.7\%} & \textbf{75.7\%} & \textbf{76.2\%} & \textbf{74.6\%} \\ 
    \bottomrule
    \end{tabular}
    \label{tab:accuracy_results_ch5}
\end{table}

    
从实验的结果看,ByteChecker 在检测四种类型的智能合约漏洞方面整体表现较为出色,体现了其静态分析与动态符号执行结合策略的有效性。具体分析如下:

从宏观平均指标来看,ByteChecker 的整体精确率为 72.1\%,召回率为 77.1\%,F1 值达到 74.6\%,说明工具在减少误报的同时也保持了较高的检出能力,检测性能较为均衡。

在具体漏洞类型上,拒绝服务漏洞的检测性能最为突出,精确率高达 72.7\%,召回率达到 80.0\%,F1 值为 76.2\%。这表明 ByteChecker 能够较好地识别此类漏洞,并且漏报较少,适合用于对拒绝服务风险的重点审查。

整数溢出和时间依赖类漏洞的表现也较为优秀,F1 值分别为 73.7\% 和 75.7\%。这表明在处理复杂的合约路径与时间条件相关的漏洞场景时,ByteChecker 的符号执行策略具有较强的能力。

重入漏洞的精确率略低,为 70.0\%,但召回率达到 76.1\%,说明其在识别重入模式方面仍有一定的优势,尤其是在防止漏检方面表现良好。

总体来看,ByteChecker 在四类漏洞检测任务中的表现较为均衡,表明其分析方法在识别不同类型漏洞方面具有一定的适用性。召回率较高,显示其在发现真实漏洞方面具备较强的检测能力,这对于提高漏洞检测覆盖率具有一定价值。

在此基础上本实验进一步分析误报和漏报的具体情况,相对于误报,本实验更加关注漏报的具体情况,除去因为编译器版本问题导致的误报,漏报产生的原因都因为复杂的控制流形态,导致 ByteChecker 在符号执行中未能探索到所有可能的路径,或是路径条件过于复杂,导致符号执行引擎无法求解出路径条件。 为此本实验在此处选取了一个合约进行分析,展示了漏报的具体情况。

\begin{figure}[H]
    \centering
    \begin{minipage}{0.9\textwidth}
    \begin{minted}[fontsize=\large, linenos, numbersep=5pt, xleftmargin=0em,baselinestretch=1.0]{solidity}
contract SavingsManager {
function claim() external {
    address payable recipient = payable(tx.origin);
    uint256 payout = _calculatePayout(recipient);

    if (_validate(recipient, payout)) {
        _dispatch(recipient, payout);
    }
}
function _calculatePayout(address user) 
        internal view returns (uint256) {
    return uint256(accounts[user].balance);
}
function _validate(address user, uint256 amount) 
        internal view returns (bool) {
    return amount > 1 ether; 
}
function _dispatch(address payable user, uint256 amount) internal {
    (bool ok, ) = user.call{value: amount}("");
    require(ok);
    accounts[user].balance = 0;
}
}
    \end{minted}
    \end{minipage}
    \caption{重入漏报}
    \label{fig:reentrancy_miss}
\end{figure}

如图\ref{fig:reentrancy_miss}的简化合约所示,合约存在一个隐藏的重入漏洞,用户通过 claim 函数发起提现操作,提现金额 amount 通过结构体字段间接获得,完成 \_validate 函数校验后,最终通过 \_dispatch() 进行外部调用。经过分析,该漏洞点未被 ByteChecker 检测出来,主要原因为:ByteChecker 对当前合约进行字节码分析,当重入的关键 CALL 指令位于多层内部函数,而调用入口又包含了多层逻辑分支与路径控制时,工具很难在字节码层面还原函数之间的真实控制流关系。同时,CALL 转账金额并非直接来自 msg.value ,而是通过中间变量传递、结构体字段索引等形式实现,这进一步增加了数据追踪的困难。

类似这样的漏报其实本质是一个安全和效率的权衡问题,对于非常不典型的重入漏洞场景,对其特征进行兼容性建模会严重影响系统的运行效率。


\subsection{效率评估与对比分析}

为评估 ByteChecker 的分析效率并进行横向对比,本实验在数据集 2 的一个代表性子集(包含 5000 个不同复杂度的合约)上运行了 ByteChecker、Mythril、Slither 和 Oyente。所有实验均在统一环境下进行,单个合约分析超时阈值设为 15 分钟 (900 秒)。核心效率指标——分析时间和峰值内存消耗的统计结果如表\ref{tab:efficiency_comparison}所示。

\begin{table}[H]
    \centering
    \caption{ByteChecker 与对比工具在数据集 2 上的效率比较}
    \begin{tabularx}{\linewidth}{lccccc}
        \toprule
        \textbf{工具} & \textbf{主要技术} & \textbf{平均时间 (s)} & \textbf{中位数时间 (s)} & \textbf{最短时间 (s)} & \textbf{最长时间 (s)} \\
        \midrule
        Slither & 静态分析 & 8.5 & 4.2 & 0.5 & 112.3 \\
        Oyente & 符号执行  & 65.2 & 38.1 & 1.8 & 900.0  \\
        Mythril & 符号执行 & 215.8 & 155.6 & 2.5 & 900.0 \\
        ByteChecker & 静态+符号执行 & 148.3 & 95.7 & 1.2 & 900.0 \\
        \bottomrule
    \end{tabularx}
    \label{tab:efficiency_comparison}
\end{table}

\textbf{结果分析:}

从表\ref{tab:efficiency_comparison}的数据可以看出不同工具在分析效率上的显著差异,这与其采用的核心技术密切相关。

1.  Slither 作为纯静态分析工具,其分析速度最快,平均时间和中位数时间均远低于其他工具。其分析时间主要取决于合约结构复杂度和需要检查的模式数量,不受路径爆炸等动态执行问题的影响,因此最短和最长时间跨度相对较小,且没有出现超时情况。这符合静态分析通常具有较高效率的预期。

2.  Mythril 和 Oyente 作为基于符号执行的工具,其平均和中位数分析时间显著长于 Slither。这主要是由于符号执行需要探索程序路径、进行约束求解,面临路径爆炸和求解器性能瓶颈的挑战。两者都出现了大量合约分析超时的情况(最长时间达到 900 秒),表明它们在处理复杂合约时遇到了困难。Mythril 的平均时间长于 Oyente,可能反映了其更深入的分析策略或不同的实现细节。

3.  ByteChecker 采用了静态分析与符号执行相结合的方法。其平均和中位数分析时间介于纯静态分析工具 (Slither) 和纯符号执行工具 (Mythril) 之间。相较于 Mythril,ByteChecker 的平均分析时间缩短了约 31\%,中位数时间缩短了约 38\%。这表明静态分析的预处理和引导作用,在一定程度上提高了符号执行的效率,例如通过剪枝不相关路径或优先探索潜在漏洞区域。尽管如此,ByteChecker 仍然依赖符号执行进行深度分析,因此其效率仍低于 Slither,并且在处理部分极其复杂的合约时同样会超时。

4.  所有工具的最短分析时间都相对较低,说明对于简单合约,各种技术都能快速完成分析。


综上所述,实验结果在效率层面验证了不同技术路线的特点。ByteChecker 的混合方法在效率上相较于纯符号执行工具展现出一定的改进,但仍需消耗比纯静态分析更多的时间和资源。这种效率表现是在追求比静态分析更深层次的漏洞发现能力(如实验一所示的准确性)下的一种权衡。


    
\subsection{静态分析引导机制有效性分析}
    
为验证静态分析引导机制对提升动态符号执行效率的实际效果,在实验 3 中对比了包含静态分析引导的完整版 ByteChecker,后文称之为 ByteChecker-Full, 与禁用该引导,仅依赖底层符号执行引擎的版本,后文称之为  ByteChecker-SE-Only。实验在数据集 3 上进行,超时阈值设为 15 分钟。核心对比指标包括分析时间、探索的路径数量以及检测到的漏洞数量。实验结果汇总于表 X。


    
\begin{table}[H]
    \centering
    \caption{静态分析引导机制对 ByteChecker 效率影响的比较 (数据集 3)}
    \begin{tabular}{lcccr}
        \toprule
        \textbf{指标} & \textbf{ByteChecker-SE-Only} & \textbf{ByteChecker-Full} & \textbf{改进率 (\%)} \\
        \midrule
        平均分析时间 (s) & 245.6 & 178.8 & 27.2\% \\
        中位数分析时间 (s) & 180.2 & 115.5 & 35.9\% \\
        平均探索路径数 (千条) & 14.8 & 8.9 & 39.9\% \\
        平均检测漏洞数 (个/合约) & 1.45 & 1.51 & -4.1\% \\
        超时合约比例 (\%) & 18.5\% & 11.2\% & 39.5\% \\
        \bottomrule
    \end{tabular}
    \caption*{注:改进率计算方式为 (SE-Only - Full) / SE-Only。负改进率表示 Full 版本指标更高。路径数和漏洞数为平均每个合约的值。}
    \label{tab:static_guidance_comparison}
\end{table}

\textbf{结果分析:}
    
表\ref{tab:static_guidance_comparison}的数据清晰地展示了静态分析引导机制对 ByteChecker 性能的积极影响。

1. 分析时间显著减少: ByteChecker-Full 的平均分析时间相较于 ByteChecker-SE-Only 减少了 27.2\%,中位数时间减少了 35.9\%。这表明静态分析模块通过预先识别潜在的漏洞区域和剪枝明显不相关的代码路径,有效减少了后续符号执行所需的时间。超时合约的比例也从 18.5\% 显著下降到 11.2\%,进一步证明了效率的提升。

2. 路径探索复杂度降低: ByteChecker-Full 探索的路径数量比 ByteChecker-SE-Only 减少了近 40\%。这直接印证了静态分析在引导符号执行引擎方面的作用,使其能够更聚焦于与潜在漏洞相关的路径,避免了大量冗余或低价值路径的探索,从而显著降低了分析的复杂度。

3. 检测能力未受影响甚至略有提升: 对比两个版本平均检测到的漏洞数量,ByteChecker-Full (1.51个/合约) 略高于 ByteChecker-SE-Only (1.45个/合约)。这初步表明,效率的提升并非以牺牲检测能力为代价。静态分析的引导可能帮助符号执行引擎优先探索到一些更深层次或更难触发的漏洞路径,从而在减少总体路径的同时,略微提高了漏洞发现的数量。

综上所述,实验结果有力地支持了静态分析引导机制的有效性(RQ3)。该机制通过优化路径选择和剪枝策略,在不牺牲甚至略微提升漏洞检测能力的前提下,显著降低了符号执行所需的时间和探索复杂度,提高了 ByteChecker 的整体分析效率和可扩展性。
    




\subsection{实验总结}


本章通过在标准数据集和真实世界合约数据集上进行的一系列实验,对本文提出的基于字节码静态分析与符号执行的漏洞检测系统 ByteChecker 进行了全面评估,旨在回答引言中提出的三个研究问题 (RQ1-RQ3)。主要实验结论总结如下:

1.  准确性 (RQ1): 在包含 186 个已知漏洞的标准数据集上,ByteChecker 展现了良好的整体检测能力。其宏平均 F1 分数达到 74.6\%,精确率为 72.1\%,召回率为 77.1\%。在四种目标漏洞类型中,对拒绝服务(F1: 76.2\%)和时间依赖(F1: 75.7\%)的检测效果尤为突出,对整数溢出(F1: 73.7\%)和重入(F1: 72.9\%)也表现出有效的检测能力。对漏报案例的分析表明,复杂的控制流和难以求解的路径约束是当前方法面临的主要挑战。

2.  效率与可扩展性 (RQ2): 在包含 5000 个真实世界合约的数据集子集上,ByteChecker 的平均分析时间为 148.3 秒,中位数时间为 95.7 秒。其效率显著优于纯符号执行工具 Mythril(平均 215.8 秒),但低于纯静态分析工具 Slither(平均 8.5 秒)。这一定位反映了其混合方法在效率与分析深度之间的权衡。虽然对于部分极其复杂的合约仍会超时,但静态分析引导机制显示出缓解路径爆炸、提升可扩展性的潜力。

3.  静态分析引导机制有效性 (RQ3): 对比实验明确验证了静态分析引导机制的积极作用。启用该机制后,平均分析时间减少了 27.2\%,中位数时间减少了 35.9\%,平均探索路径数量减少了 39.9\%,并且超时合约比例降低了 39.5\%。重要的是,效率的提升并未以牺牲检测能力为代价,平均检测到的漏洞数量甚至略有增加(4.1\%),表明引导机制有助于聚焦于更有可能存在漏洞的路径。

4.  局限性: 实验也揭示了 ByteChecker 的局限性。对于依赖复杂外部交互或精确资源建模的漏洞(如某些时间依赖和 Gas 相关的 DoS),字节码层面的分析仍存挑战,可能导致误报或漏报。同时,对于规模极大或状态空间极其庞大的合约,分析时间和内存消耗依然是瓶颈,尽管静态引导有所缓解。

综上所述,实验结果表明,本文提出的 ByteChecker 系统作为一种结合字节码静态分析与符号执行的混合方法,在检测特定类型的智能合约漏洞方面是有效且具有竞争力的。它在准确性和效率之间取得了较好的平衡,并通过静态分析引导显著提升了符号执行的可行性。然而,在处理特定复杂漏洞场景和极端规模合约方面仍存在改进空间,为未来的研究指明了方向。

% ...existing code...
\chapter{总结与展望}

\section{全文总结}

随着区块链技术的蓬勃发展,智能合约已成为构建去中心化应用的核心基石。然而,其承载高价值资产且部署后通常不可更改的特性,使得安全问题尤为突出。现有的智能合约漏洞检测方法,如静态分析、动态分析、符号执行和形式化验证等,虽然取得了一定进展,但在检测精度、效率、自动化程度以及对复杂漏洞模式的处理能力上仍存在各自的局限性,难以完全满足实际需求。特别是在缺乏源代码、只能分析 EVM 字节码的场景下,现有工具面临更大挑战。

针对上述问题,本文聚焦于 EVM 字节码层面,旨在设计并实现一种更有效、更高效的自动化漏洞检测方法。研究工作主要围绕以下几个方面展开:首先,深入分析了四种典型且危害严重的智能合约漏洞——重入、整数溢出、时间依赖和拒绝服务攻击——的产生原理、攻击模式及其在 EVM 字节码层面的具体指令序列和状态转换特征,为后续的自动化检测奠定了理论基础。

其次,基于对漏洞字节码特征的理解,本文提出了一种结合轻量级静态分析与符号执行的混合式漏洞检测方法。该方法的核心创新在于利用静态分析快速识别潜在的漏洞候选点,并以此引导后续的动态符号执行,使其能够更聚焦于高风险的代码路径,从而在保证分析深度的同时,有效缓解符号执行固有的路径爆炸问题,提升整体分析效率。

再次,基于所提出的混合分析方法,本文设计并实现了一个名为 ByteChecker 的原型系统。详细阐述了系统的整体架构,包括输入处理、预处理(字节码反汇编、CFG 构建)、核心分析引擎(包含静态分析模块、动态符号执行模块、漏洞检测逻辑)以及结果处理与报告等关键模块的设计细节。重点介绍了符号执行引擎如何进行状态建模、指令符号化模拟、与 Z3 约束求解器的交互,以及静态分析模块如何通过模式匹配和污点分析生成候选点,动态分析模块如何采用启发式引导的 DFS 策略进行路径探索和验证。

最后,通过在标准数据集和真实世界合约数据集上进行的一系列实验,对 ByteChecker 系统的性能进行了全面评估。实验结果表明:在准确性方面,ByteChecker 对四种目标漏洞展现了良好的检测能力(总体 F1 分数 74.6\%);在效率方面,其分析速度显著优于纯符号执行工具 Mythril,验证了混合方法的优势;关键的静态分析引导机制被证明能够有效降低分析时间和路径探索复杂度(平均时间减少 27.2\%,路径数减少 39.9\%),且未牺牲检测能力。

\section{后续工作展望}

尽管 ByteChecker 在智能合约漏洞检测方面取得了一定成效,但仍存在显著的改进空间和未来研究方向。首先,在提升分析精度与广度方面,当前方法在处理极端复杂合约逻辑或依赖精细外部环境建模(如多合约交互、精确 Gas 消耗)的漏洞时仍面临挑战。未来的研究可以探索更高级的程序分析技术,例如将抽象解释与符号执行更紧密地结合,或者开发专门的多合约交互分析方法,以更准确地捕捉复杂场景下的漏洞。同时,ByteChecker 目前专注于四种典型漏洞,未来可扩展其字节码特征库和检测逻辑,以支持更广泛的漏洞类型检测,例如访问控制缺陷、业务逻辑错误、代理合约相关问题等,从而构建更全面的安全分析能力。

其次,在优化分析效率与可扩展性方面,虽然静态分析引导机制已证明有效,但效率仍有提升潜力。可以研究更先进的路径选择启发式策略,例如结合机器学习预测高风险路径或基于信息增益的探索策略,进一步聚焦于最可能包含漏洞的代码区域。探索并行或分布式符号执行技术,将路径探索任务分解到多个计算单元,有望大幅缩短对大型或复杂合约的分析时间。此外,将符号执行与模糊测试(Fuzzing)、机器学习等其他技术进行更深层次的融合,例如利用 Fuzzing 快速覆盖浅层路径并生成种子输入,再由符号执行进行深度探索,也是一个充满前景的方向。

最后,提升工具的易用性与实际集成度对于推动研究成果落地至关重要。未来的工作可以着力于优化漏洞报告的详细程度和可理解性,例如提供更清晰的触发路径、变量值示例,甚至结合反编译结果进行可视化展示。将 ByteChecker 集成到常见的智能合约开发框架(如 Hardhat, Truffle)或持续集成/持续部署(CI/CD)流水线中,使其成为开发和审计过程中的自动化安全检查环节,实现“安全左移”。随着区块链生态的发展,将本文提出的混合分析思想和技术扩展应用于新兴的区块链平台和虚拟机(如基于 WASM 的 VM),以应对其特有的安全挑战,也是未来值得努力的方向。总之,智能合约安全是一个持续演化的领域,未来的研究需在技术深度和广度上不断突破,并更加注重工具的实用性和生态融合。


\bibliography{reference}
\end{document}